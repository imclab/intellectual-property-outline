\section{Introduction}

\subsection{Philosophy}

\subsubsection{Natural Rights: Locke, \emph{Two Treatises on Government}}

\begin{enumerate}
    \item Property derives from labor of the body and work of the 
    hands---as long as he leaves ``enough and as good'' in the commons for 
    others.\footnote{Casebook p. 2.}
    \item How does Locke's theory apply to intellectual (non-tangible) 
    property?
    \item Europeans emphasize natural rights---e.g., reputation and 
    noneconomic aspects.\footnote{Casebook p. 5.}
\end{enumerate}

\subsubsection{Personhood: Radin, \emph{Property and Personhood}}

\begin{enumerate}
    \item Two views of property: personal and fungible.\footnote{Casebook p. 
    7.}
    \item Hegel: property is embodied will.\footnote{Casebook p. 8.}
\end{enumerate}

\subsubsection{Utilitarian/Economic Incentive}

\begin{enumerate}
    \item The dominant justification for American IP law.
    \item Two functions:\footnote{Casebook p. 11.}
    \begin{enumerate}
        \item Provide incentives to create.
        \item Ensure the integrity of the marketplace.
    \end{enumerate}
\end{enumerate}

\subsection{Overview of Intellectual Property}

\begin{enumerate}
    \item See pp. 25--31.
    \item See especially the table on pp. 26-28.
\end{enumerate}
