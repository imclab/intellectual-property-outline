\section{Trademark}

\subsection{Introduction}

\subsubsection{Basic Goals of Trademarks}

\begin{enumerate}
    \item Protect consumers.
    \item Protect the integrity of the marketplace.
\end{enumerate}

\subsubsection{Background}

\begin{enumerate}
    \item Trademarks ``help to reduce information and transaction costs by 
    allowing customers to estimate the nature and quality of goods before 
    purchase.''\footnote{Casebook p. 763.}
    \item 1870: first federal trademark statute. Struck down as beyond 
    Congress's powers under the patent and copyright clause.\footnote{Casebook 
    p. 764.}
    \item 1881: second federal statute, passed under the Commerce Clause. 
    Significantly modified in 1905 and 1920.
    \item 1946: Lanham Act, 15 U.S.C. \S\S\ 1051 et seq.
    \item Trademark protections have generally expanded.\footnote{Casebook p. 
    765.} For instance, they now include trade dress (in addition to marks) 
    and intent to use (in addition to actual use).
\end{enumerate}

\subsubsection{A Brief Overview of Trademark Theory}

\begin{enumerate}
    \item Trademarks do not protect or reward novelty or invention. They 
    reward the first user of a mark.\footnote{Casebook p. 765.}
    \item Traditional trademark principles resemble tort law: preventing 
    unfair competition and consumer deception.
    \item Infringement actions can protect consumers.\footnote{Casebook p. 
    766.}
    \item Trademarks protect three kinds of investment:
    \begin{enumerate}
        \item Creation of the mark.
        \item Advertising.
        \item Product creation.
    \end{enumerate}
    \item \textbf{Guiding principle}: customer perception as a test for 
    validity and infringement.
\end{enumerate}

\subsubsection{The Basic Economics of Trademark and Advertising}

\begin{enumerate}
    \item No consensus on the economic function of 
    trademarks.\footnote{Casebook p. 766.}
    \item Does advertising help or manipulate consumers? Does it communicate 
    valuable price or quality information, or does it artificially create 
    demand for nonessential features?
    \item Early commentators--``product differentiation theory'':
    \begin{enumerate}
        \item Trademarks are bad.
        \item Advertising unnaturally stimulates demand.
        \item Advertising perpetuates oligopoly. For instance, brand-name 
        drugs can sell for twice as much as chemically identical generic 
        drugs. Arguably, this hurts consumers.\footnote{Casebook pp. 766--67.}
    \end{enumerate}
    \item Now--``product information theory'':
    \begin{enumerate}
        \item Trademarks are good.
        \item Consensus that advertising cheaply conveys information to 
        consumers.
        \item Product's ``search characteristics'': price, color, shape, etc.
        \item Product's ``experience characteristics'': taste, long-term 
        durability. These are only apparent after purchase. Advertising and 
        trademarks can help identify these characteristics and encourage 
        repeat purchases.
    \end{enumerate}
    \item \emph{Consumer protection theory}: advertising cheaply conveys 
    valuable information to consumers.
    \item \emph{Producer incentive theory}: trademarks are ``essential 
    shorthand'' for consumers' positive associations with a 
    product.\footnote{Casebook p. 768.}
    \item Copyright protections have expanded beyond protecting against 
    consumer deception to a broader range of property-like 
    protections.\footnote{Casebook pp. 769--70.}
\end{enumerate}

\subsection{What Can Be Protected as a Trademark? (Subject Matter)}

\subsubsection{Trademarks, Trade Names, and Service Marks}

\begin{enumerate}
    \item \textbf{Trademark}: ``word, name, symbol, or device'' to 
    \textbf{identify and distinguish} goods and to \textbf{indicate the 
    source} of goods. The source can be unknown; what matters is that the 
    trademark is a unique identifier.\footnote{15 U.S.C. \S\ 1127; casebook p. 
    771.}
    \item \textbf{Service mark}: ``word, name, symbol, or device'' to identify 
    \emph{services}. Generally subject to the same rules as trademarks. E.g., 
    ``Hyatt hotel services.''
    \item \textbf{Trade names}: can only be registered if they \emph{identify 
    the source of particular goods}, rather than a company 
    alone.\footnote{Casebook p. 772.}
    \item \textbf{Slogans}: ``the greatest show on earth.''
\end{enumerate}

\subsubsection{Certification and Collective Marks}

\begin{enumerate}
    \item \textbf{Certification mark}: ``word, name, symbol, or device'' to 
    certify characteristics of a product---i.e., a seal of approval. Used by 
    trade associations and commercial groups---e.g., ``Good Housekeeping,'' 
    the city of Roquefort.\footnote{Casebook pp. 772--73.}
    \item \textbf{Collective mark}: trademark or service mark adopted by a 
    collective. It can be (1) used by its members to distinguish its products 
    from non-member products, or (2) indicating membership in a collective 
    group, like a union.\footnote{Casebook p. 773.} Marks of the first type 
    are useful in franchising arrangements.\footnote{Casebook p. 774.}
\end{enumerate}

\subsubsection{Trade Dress and Product Configuration}

\begin{enumerate}
    \item Design and packaging---and sometimes the design of the product 
    itself.\footnote{Casebook p. 774.}
\end{enumerate}

\subsubsection{Color, Fragrance, and Sounds: \emph{Qualitex Co. v. Jacobson 
Products Co., Inc.}}

Colors can be trademarked, but like other trademarks, they require secondary 
meaning. The secondary meaning allows consumers to identify the product's 
source.

\begin{enumerate}
    \item Qualitex manufactured drycleaning pads with a distinctive green-gold 
    color.
    \item Circuit courts were split on whether colors could be trademarked.
    \item ``Symbol'' or ``device'' can refer to ``almost anything at all that 
    is capable of carrying meaning~.~.~.~''\footnote{Casebook p. 775.}
    \item Colors can signify brands. When a color signifies a product's 
    origin, it has taken on a ``secondary meaning.''\footnote{Casebook p. 776.}
    \begin{enumerate}
        \item \textbf{\enquote{\enquote{secondary meaning} is acquired when 
        \enquote{in the minds of the public, the primary significance of a 
        product feature~.~.~.~is to identify the source of the product rather 
        than the product itself}}}\footnote{Casebook p. 776.}
    \end{enumerate}
    \item Color meets the basic requirements for use as a trademark. It 
    can distinguish goods and identify their source without serving any other 
    function.\footnote{Casebook p. 777.}
    \item Jacobson made four arguments for why color alone should not be 
    granted trademark protections:
    \begin{enumerate}
        \item \emph{First}: it will create uncertainty (``shade confusion'').
        \begin{enumerate}
            \item But courts are skilled at making difficult decisions.
        \end{enumerate}
        \item \emph{Second}: colors are in limited supply.
        \begin{enumerate}
            \item Maybe, but that ``occasional problem'' does not justify a 
            ``blanket prohibition.'' 
            \item The \textbf{functionality doctrine} ``forbids the use of a 
            product's feature as a trademark where doing so will put a 
            competitor at a significant disadvantage because the feature is 
            \enquote{essential to the use or purpose of the article} or 
            \enquote{affects [its] cost or quality.}''\footnote{Casebook p. 
            779.} Here, if a limited supply of colors would harm competitors, 
            courts would not allow exclusive use of one of the colors as a 
            trademark.
        \end{enumerate}
        \item \emph{Third}: many older cases support its position.
        \begin{enumerate}
            \item No---those were pre--Lanham Act.
        \end{enumerate}
        \item \emph{Fourth}: firms can already use color as part of another 
        trademark or trade dress.
        \begin{enumerate}
            \item A company might have a reason to use a color instead of a 
            word or symbol.
        \end{enumerate}
    \end{enumerate}
    \item Justice Breyer, dissenting:
    \begin{enumerate}
        \item Four reasons not to extend trademark protections to colors:
        \begin{enumerate}
            \item Slippery slope.
            \item Depletion.
            \item Old cases.
            \item Companies can incorporate color into their broader mark.
        \end{enumerate}
    \end{enumerate}
\end{enumerate}

\subsection{Establishment of Trademark Rights}

\subsubsection{Distinctiveness}

\paragraph{Classification of Marks and Requirements for Protection}

\begin{enumerate}
    \item When a trademark can identify a \textbf{unique} product source, 
    rights are determined by priority of use.\footnote{Casebook p. 781.} These 
    types of marks are then further subdivided:
    \begin{enumerate}
        \item \textbf{Arbitrary}: Kodak, Exxon, Google.
        \item \textbf{Fanciful}: Apple Computer. The line between arbitrary 
        and fanciful is thin; \emph{Zatarain's} conflates them---p. 784.
    \item \textbf{Suggestive}: Coppertone.
    \end{enumerate}
    \item All other marks---\textbf{non-distinctive}---require 
    \textbf{secondary meaning}. Types:
    \begin{enumerate}
        \item \textbf{Descriptive}: describes something about the 
        goods or services---e.g., ``Tender Vittles'' for cat food or 
        ``Arthriticare'' for arthritis treatment.\footnote{Casebook p. 781.} 
        \item \textbf{Geographic}.
        \item \textbf{Personal names}.
        \item ``Secondary meaning exists when buyers associate a descriptive 
        term with a single source of products.''\footnote{782.} Buyers do not 
        need to know the \emph{identity} of the source; they only need to know 
        that the product comes from a \emph{single} source.
        \item \textbf{Colors, fragrances, sounds}: UPS brown, IBM blue.
    \end{enumerate}
    \item \textbf{Generic}: aspirin, cellophane, thermos, yo-yo, escalator, 
    google, kleenex. These become generic through genericide, i.e, they 
    become associated with a \textbf{category} or type of products, rather 
    than a specific source---so they no longer serve the goals of trademark 
    law (to protect consumers and preserve the integrity of the marketplace).  
    \textbf{Unprotectable even if they have acquired secondary 
    meaning.}\footnote{Casebook p. 785.})
\end{enumerate}

\paragraph{Descriptive Marks and Fair Use: \emph{Zatarain's, Inc. v. Oak Grove 
Smokehouse, Inc.}}
~\\\\
Descriptive terms can be trademarked, but the fair use defense allows 
competitors to use them in their original, descriptive sense. So, Zatarain's 
can have an exclusive right in the phrase ``Fish-Fri'' for its fish frying 
batter, but its competitors can use the words ``fish'' and ``fry'' to describe 
their own products.

Also: consumer surveys can help show secondary meaning.

Also: fair use in trademark law \textbf{applies only to descriptive marks}.

\begin{enumerate}
    \item Zatarain's brought infringement suits for two of its registered 
    trademarks: ``Fish-Fri'' and ``Chick-Fri.''
    \item \textbf{Fair use} ``prevents a trademark registrant from 
    appropriating a descriptive term for its own use to the exclusion of 
    others, who may be prevent thereby from accurately describing their own 
    goods.''\footnote{Casebook p. 785.}
    \item ``Fish-Fri'':
    \begin{enumerate}
        \item Is ``Fish-Fri'' descriptive?
        \begin{enumerate}
            \item \emph{Dictionary}: refers to fried fish, so yes, 
            ``Fish-Fri'' is descriptive of the product (which is used to fry 
            fish).
            \item \emph{``Imagination test}'': if imagination is required to 
            associate the term with the product, the term is suggestive, not 
            descriptive. But here, no imagination is required to associate the 
            two.
            \item \emph{Competitive need}: do competitors need to use the term 
            to describe their products? Here, probably yes, because there is a 
            ``paucity of synonyms'' for ``fish'' and 
            ``fry.''\footnote{Casebook p. 787.}
            \item \emph{Actual use}: have competitors used the term to 
            describe their own products? Here, yes.
            \item So, ``Fish-Fri'' is descriptive.
        \end{enumerate}
        \item Does ``Fish-Fri'' have a secondary meaning?
        \begin{enumerate}
            \item Yes---survey evidence (of Louisiana, at least) and 
            circumstantial evidence showed that consumers associated 
            ``Fish-Fri'' with a specific source.
        \end{enumerate}
        \item Was there fair use?
        \begin{enumerate}
            \item Zatarain's cannot claim an exclusive right in the original, 
            descriptive sense of ``fish fry.'' So, its competitors are free to 
            use them in that sense.\footnote{Casebook p. 789.}
            \item Also, dissimilar trade dress would likely prevent consumer 
            confusion.
        \end{enumerate}
    \end{enumerate}
    \item ``Chick-Fri'':
    \begin{enumerate}
        \item Descriptive? Yes.\footnote{Casebook p. 789.}
        \item Secondary meaning? No (``paltry''---ha).\footnote{Casebook p. 
        790.}
    \end{enumerate}
\end{enumerate}

\paragraph{Genericness}

\begin{enumerate}
    \item A trademarked term must point to a specific source. 
    \item ``Toyota'' is a species. ``Car'' is a genus.\footnote{Casebook p. 
    794.}
    \item Terms are either born generic (``copier'') or become generic through 
    genericide (``Xerox'').
    \item Anybody can petition for \textbf{cancellation} of a name that has 
    become generic. 15 U.S.C. \S\ 1064 (Lanham \S\ 14). Courts can also 
    declare trademarks to be generic.
    \item No protection for ``merely descriptive terms''---\S\ 1052 (Lanham 
    \S\ 2). However, descriptive marks can be registered if they have ``become 
    distinctive of the applicant's goods in commerce.'' \S\ 1052(f), Lanham 
    \S\ 2(f).
    \item Firms use advertising and lawsuits to protect their trademarks 
    against dilution, but these tactics might conflict with competition and 
    the First Amendment.\footnote{Casebook pp. 798--99.}
    \item Generally, generic terms are not trademarkable, but at least one 
    state has held that a generic term in public use can be protected if used 
    uniquely by a product developer (``Anti-Defamation 
    League'').\footnote{Casebook p. 799.}
    \item Brandeis on genericide: \emph{Kellogg Co. v. National Biscuit 
    Co.}\footnote{Casebook p. 799.}
    \item Non-word marks can be generic---e.g., grape 
    leaves.\footnote{Casebook p. 800.}
\end{enumerate}

\paragraph{Genericide: \emph{The Murphy Door Bed Co., Inc. v. Interior Sleep Systems, Inc.}}
~\\\\
Marks that have become generic are no longer protectable as trademarks.

\begin{enumerate}
    \item 1918: Mr. Murphy won a patent for his bed design.\footnote{Casebook 
    p. 794.}
    \item 1925: Murphy incorporated the Murphy Door Bed Company.
    \item 1981--84: The PTO and TTAB denied trademark protection for ``Murphy 
    bed'' because the term had become generic.
    \item 1981: Defendants entered into a distribution agreement with the 
    plaintiffs, requiring it to identify Murphy beds as trademarked. Upon 
    learning of the TTAB's rejection, defendants incorporated a new company 
    with ``Murphy bed'' in the name.
    \item The district court held that ``Murphy bed'' was not generic because 
    it had a secondary meaning.\footnote{Casebook p. 795.}
    \item Rule: trademark protection is denied to a generic term unless there 
    is a secondary meaning.
    \item Who has the burden of proving genericness?
    \begin{enumerate}
        \item If the term was originally generic (e.g., ``Video Buyer's 
        Guide''), the plaintiff has the burden of showing that it's not 
        generic.
        \item If the term was originally specific but the public later 
        expropriated it, the defendant has the burden of showing that it is 
        generic.\footnote{Casebook p. 796.}
    \end{enumerate}
    \item Since this term was originally not generic, the defendants had the 
    burden of showing that it had become generic.
    \item Held: the term was generic because of (1) the TTAB's findings, (2) 
    the presence of ``Murphy bed'' as a generic term in dictionaries, and (3) 
    newspaper and magazine use as a generic term. Reversed.
\end{enumerate}

\paragraph{Genericide, Language, and Policing Costs}

\begin{enumerate}
    \item Genericide takes away some rights that the creator earned through 
    his or her creativity.\footnote{Casebook p. 801.}
    \item Some creative works create \textbf{network externalities}: users 
    benefit from the fact that others use the work (e.g., technical standards 
    like TCP/IP). Generic marks can have network externalities. These 
    externalities derive partly from collective labor, rather than from the 
    labor of an individual creator, so it might make sense not to give the 
    creator exclusive rights.\footnote{Casebook pp. 802--03.}
    \item Should the cost of policing (e.g., ``you can't Xerox a Xerox on a 
    Xerox'') be a factor in the genericness analysis? Policing can indicate an 
    attempt to provide an alternative generic standard. On the other hand, it 
    can also indicate an attempt to maintain barriers to entry and to create 
    free advertising. Policing noncommercial uses also raises First Amendment 
    concerns.\footnote{Casebook p. 803.}
\end{enumerate}

\paragraph{Trade Dress and Secondary Meaning: \emph{Two Pesos, Inc. v. Taco 
Cabana, Inc.}}
~\\\\
Trade dress is the ``total image of the business.'' Proof of secondary meaning 
is \textbf{not required for an inherently distinctive trade dress to be 
protectable}. Trade dress should be treated the same as verbal or symbolic 
trademarks.

Also: functional trade dress (e.g., Korean barbecue) cannot be protected under 
trademark law.

\begin{enumerate}
    \item Two Pesos opened restaurants that mimicked the decor of Taco 
    Cabana's restaurants.
    \item The question was whether inherently distinctive trade dress (i.e., 
    restaurant decor) requires proof of secondary meaning to be 
    protectable.\footnote{Casebook p. 806.}
    \item \S\ 43(a) does not require secondary meaning for inherently 
    descriptive words or symbols. There is no basis in the statute for 
    distinguishing between trade dress and other types of marks. Thus, proof 
    of secondary meaning is not required where the trade dress is inherently 
    distinctive.\footnote{Casebook pp. 807--08.}
\end{enumerate}

\paragraph{Product Design and Inherent Distinctiveness: \emph{Wal-Mart Stores, 
Inc. v. Samara Brothers, Inc.}}
~\\\\
``Symbol'' or ``device'' can mean just about anything. Product design, like 
color, is not inherently distinctive, so it's protectable only on a showing of 
secondary meaning.

On the other hand, project \emph{packaging} can be inherently distinctive, so 
it doesn't require secondary meaning.

\begin{enumerate}
    \item Wal-Mart copied the design of a line of clothing that Samara 
    manufactured.\footnote{Casebook p. 810.}
    \item Trade dress includes product design.\footnote{Casebook p. 811.}
    \item Marks can be distinctive in two ways: (1) inherently distinctive 
    (i.e., arbitrary, fanciful, or suggestive) or (2) secondary meaning.
    \item \emph{Qualitex} (see above) held that colors can never be inherently 
    distinctive.
    \item Similarly, product design can never be protected without a showing 
    of secondary meaning.\footnote{Casebook p. 813.}
    \item The court distinguishes three elements of trade dress:
    \begin{enumerate}
        \item Product \emph{design} (at issue here): not inherently 
        distinctive, so it requires a showing of secondary meaning.
        \item Product \emph{packaging}: \emph{can} indicate origin and 
        therefore can be protectable.
        \item ``Tertium quid'': something ``akin to product 
        packaging.''\footnote{Casebook p. 814.}
    \end{enumerate}
\end{enumerate}

\paragraph{Conflicts among IP Protections}

\begin{enumerate}
    \item The ``principle of election'' allowed only one type of protection 
    per item. Rejected in \emph{Yardley}.\footnote{Casebook p. 815.}
    \item Utility patent protection generally trumps other forms of protection 
    regarding functional features because it has the highest 
    burden.\footnote{Casebook pp. 815--16.}
    \item Trademark law does not protect against copying of works whose 
    copyright protections have expired.\footnote{Casebook p. 816.}
\end{enumerate}

\paragraph{Functionality and the Primacy of Patent: \emph{TrafFix Devices, 
Inc. v. Marketing Displays, Inc.}}
~\\\\
Functional trade dress is unprotected. Utility patents can support a finding 
that trade dress is functional, though courts are split.

\begin{enumerate}
    \item Two distinct issues:
    \begin{enumerate}
        \item Trademark: ``WindMaster'' vs. ``WindBuster.''
        \item Trade dress (i.e., product configuration as an indication of 
        source) in the dual spring design for road signs.
    \end{enumerate}
    \item The district court held that there was no trademark protection here 
    because (1) there was no secondary meaning in the dual spring design, and 
    (2) the design was functional.
    \item Is an expired utility patent relevant to the trademark analysis? 
    Circuits are split.
    \item Unregistered trade dress is \textbf{not protectable if it is 
    functional}. \S\ 1125(a)(3), Lanham \S\ 43(a)(3).
    \item Held: ``A utility patent is strong evidence that the features 
    therein claimed are functional.''\footnote{Casebook p. 818--19.} Here, the 
    elements of the claimed trade dress were patented---so the court found 
    them to be functional, and thus unprotectable as trade dress.
\end{enumerate}

\subsubsection{Priority}

\begin{enumerate}
    \item \S\ 45(a): mark must be (1) used in commerce or (2) registered with 
    a bona fide intent to use in commerce.\footnote{Casebook p. 828.}
    \item The analysis depends on \textbf{how} and \textbf{how much} the mark 
    is used.
\end{enumerate}

\paragraph{Priority and the Use Requirement: \emph{Zazu Designs v. L'Oreal, 
S.A.}}
~\\\\
Knowledge that another person plans to use a mark does not prevent you from 
using it. However, the \textbf{intent to use} rule lets someone ``reserve'' a 
mark if it is actually used within six months (extendable to up to three years 
for good cause). \S\ 1051(b), Lanham \S\ 1(b).

``Use'' means ``used in commerce.'' Sales are probative of use, but neither is 
necessary nor sufficient.

\begin{enumerate}
    \item Timeline:
    \begin{enumerate}
        \item 1985: Zazu met with chemists; made sales in its salon.
        \item 11/85, 2/86: Zazu send small shipments to Texas and Florida. 
        \item 4/86: L'Oreal entered into a covenant with Riviera; made its 
        first interstate shipment.
        \item 6/12/86: L'Oreal registered the mark.
        \item 1987: Zazu brought suit.
    \end{enumerate}
    \item To establish priority, you have to win the race to the 
    marketplace.\footnote{Casebook p. 830.} But there are some rules to the 
    race:
    \begin{enumerate}
        \item The owner of a trademark \textbf{used in commerce} can request 
        registration. \S\ 1051, Lanham \S\ 1.
        \item Caselaw balances two factors:\footnote{Casebook p. 830.}
        \begin{enumerate}
            \item Prevent rent-seeking: entrepreneurs reserve brand names in 
            order to raise rivals' costs.
            \item Allow firms to seek protection for a mark before investing 
            substantial sums in promotion.
        \end{enumerate}
        \item Policy justifications for the use requirement:
        \begin{enumerate}
            \item Furthers the purpose of trademark law.
            \item Prevents warehousing of trademarks.
            \item Provides notice to others.
        \end{enumerate}
        \item Drawbacks:
        \begin{enumerate}
            \item May cause uncertainty about when rights attach.
            \item May result in the loss of preparatory expenses.
        \end{enumerate}
    \end{enumerate}
    \item ``Knowledge that ZHD planned to use the ZAZU mark in the future does 
    not present an obstacle to L'Oreal's adopting it 
    today.''\footnote{Casebook p. 832.}
\end{enumerate}

\paragraph{Geographic Limitations on Trademark Use}

\begin{enumerate}
    \item Registration establishes national priority. For unregistered marks, 
    concurrent users can expand their geographic areas unless it causes 
    confusion.
    \item Two types of \textbf{concurrent use}:
    \begin{enumerate}
        \item Different products in the same market.
        \item Different geographic markets.
    \end{enumerate}
    \item Geographic priority is determined by (1) first use in a given area 
    and (2) customer associations in that area.
    \item If one concurrent user registers the mark, the other can keep using 
    it, but it will be frozen at its current geographic area. The senior party 
    gains constructive national rights.
\end{enumerate}

\subsubsection{Trademark Office Procedures}

\begin{enumerate}
    \item \textbf{Principal register}: allows nationwide constructive notice 
    and use, and incontestable status after five years.
    \item \textbf{Secondary register}: to register in foreign countries, it 
    used to be required for a mark to be registered domestically first.
    \item Marks can be rejected for being immoral, deceptive, or scandalous.
    \item Can also be rejected for being \textbf{merely descriptive}, 
    \textbf{primarily geographically descriptive}, or \textbf{primarily 
    geographically deceptively misdescriptive}---e.g., Wisconsin cheese, 
    Washington apples. Specific exceptions for appellations of origin for wine 
    and spirits (e.g., Bourbon) and certification marks (e.g., Roquefort).
    \item ``Nantucket'' for men's shirts is ok because buyers are unlikely to 
    be deceived about the shirts' origin. \emph{Nantucket}.
\end{enumerate}

\subsubsection{Incontestability: \emph{Park 'N Fly, Inc. v. Dollar Park and 
Fly, Inc.}}

Descriptiveness does not outweigh incontestability. For the for requirements 
to gain incontestable status, see \S\ 1065.

\begin{enumerate}
    \item Can an alleged infringer of an incontestable trademark defend on the 
    grounds that the trademark is descriptive?
    \item Here, Dollar Park counterclaimed for cancellation on the grounds 
    that Park 'N Fly was descriptive.\footnote{Casebook p. 862.}
    \item Held: nothing in the statute supports the view that a trademark, 
    once it has become incontestable, can be cancelled for being descriptive. 
    See Lanham Act \S\ 33(a), 15 U.S.C. \S\ 1115(a).
\end{enumerate}

\paragraph{Defenses that Survive Incontestability}

\begin{enumerate}
    \item \S\ 33(b):
    \begin{enumerate}
        \item Obtainment by fraud.
        \item Abandonment.
        \item Used to misrepresent source or origin.
        \item Fair use (only for descriptive marks).
        \item Prior third-party rights (e.g., concurrent use).
        \item Prior registered mark.
        \item Functional.
    \end{enumerate}
    \item Mark is generic. \S\ 1064(c).
    \item Others: antitrust, laches, other equitable doctrines.
\end{enumerate}

\subsection{Infringement}

\subsubsection{Use as a Trademark}

\paragraph{``Use in Commerce'': \emph{Rescuecom Corp. v. Google, Inc.}}
~\\\\
Infringement under the Lanham Act requires ``use in commerce.'' \S\ 1114. 
Liability arises when the use is ``likely to cause confusion [or mistake or 
deception].'' \S\ 1125(a). 

\begin{enumerate}
    \item Rescuecom offered computer repair services. Google suggested 
    ``Rescuecom'' as a paid search keyword to Rescuecom's competitors.
    \item District court: Google's use was not a ``use in commerce'' because 
    the final ads did not display Rescuecom's trademark.
    \item How does Google's actions affect Rescuecom's trademark interests? 
    According to the Second Circuit, the key is \textbf{customer confusion}. 
    \item We don't know whether Rescuecom can prove that Google's use causes 
    likelihood of confusion or mistake. The case settled after remand.
\end{enumerate}

\subsubsection{Likelihood of Consumer Confusion: \emph{AMF Inc. v. Sleekcraft 
Boats}}

\begin{enumerate}
    \item Is ``Slickcraft'' likely to be confused with ``Sleekcraft''?
    \item Here, there is not high cross-elasticity of demand---i.e., the goods 
    aren't perfect substitutes for each other. But they're close enough that 
    the similar marks might cause \emph{some} competitive harm.
    \item The court considered seven factors in its \textbf{strength of mark 
    analysis}:
    \begin{enumerate}
        \item Strength of the mark.
        \item Proximity or relatedness of the goods.
        \item Similarity of the marks.
        \item Evidence of actual confusion.
        \item Marketing channels used.
        \item Degree of customer care in purchase.
        \item Defendant's intent in selecting the mark.
        \item Likelihood of expansion into other marks.
    \end{enumerate}
    \item This is separate from the \textbf{invalidity analysis}---e.g., 
    \emph{Park n' Fly.}
    \item Courts often use the ``sight, sound, and meaning'' test.
    \item Disclaimers can play a role.
\end{enumerate}

\subsubsection{Dilution}

\begin{enumerate}
    \item Two kinds: \textbf{blurring} and \textbf{tarnishment} (see below).
    \item Courts can find infringement without confusion if dilution is 
    likely.\footnote{Casebook p. 889.}
    \item Federal Trademark Dilution Act of 1995 (FTDA), 15 U.S.C. 1125(c), 
    Lanham Act 43(c):\footnote{Casebook pp. 890--91.}
    \item Five elements (\S\ 1125(c)):
    \begin{enumerate}
        \item Mark must be \textbf{famous}. Factors: 42(c)(2)(A).
        \item Mark must be \textbf{distinctive} (spectrum: generic to 
        arbitrary/fanciful).
        \item Junior user makes commercial use of the mark in commerce...
        \item ...after the senior mark has become famous.
        \item Two kinds:
        \begin{enumerate}
            \item \textbf{Blurring}: ``impairs the distinctiveness.'' Six 
            factors. \S\ 42(c)(2)(B).
            \item \textbf{Tarnishment}: ``harms the reputation.'' \S\ 
            43(c)(3)(A).
        \end{enumerate}
    \end{enumerate}
    \item Limiting factors (43(c)):
    \begin{enumerate}
        \item Can only win an injunction.
        \item Registration of the defendant's mark is a complete defense.
        \item \textbf{Exclusions}: fair use, news, noncommercial use. 
    \end{enumerate}
\end{enumerate}

\paragraph{Parody and Dilution: \emph{Louis Vuitton Malletier S.A. v. Haute 
Diggity Dog, LLC}}
~\\\\
Successful parodies do not dilute by blurring because they do not ``impair the 
distinctiveness'' (43(c)(2)(B)) of the original mark. But parody is a defense 
only if the trademark is not being used qua trademark.

\begin{enumerate}
    \item Haute Diggity Dog made dog toys that pariodied high-fashion brands, 
    including ``Chewy Vuitton.''\footnote{Casebook pp. 892--93.}
    \item LVM argued that HDD's products would blur and tarnish its ``LOUIS 
    VUITTON'' mark.\footnote{Casebook p. 894.}
    \item Dilution by blurring? No.
    \begin{enumerate}
        \item Is the association between HDD's and LVM's marks likely to 
        impair the distinctiveness of LVM's marks?\footnote{Casebook p. 895.}
        \item LVM argued that any imitation of a famous mark dilutes the 
        mark.\footnote{Casebook p. 896.}
        \item After analyzing the six statutory factors (43(c)(2)(B)), the 
        court held that HDD's parody ``will not blur the distinctiveness of 
        the famous mark as a unique identifier of its 
        source.''\footnote{Casebook p. 898.}
        \item No dilution by blurring because HDD's parody was not likely to 
        \textbf{impair the distinctiveness} of LVM's marks.
    \end{enumerate}
    \item Dilution by tarnishment? No.
    \begin{enumerate}
        \item LVM argued that the possibility that a dog could choke on HDD's 
        toys harms the reputation of LVM's mark.\footnote{Casebook p. 898.}
        \item The court dismissed this claim, finding that it was unlikely 
        that a dog would choke ``on such a toy.''\footnote{Casebook p. 899.}
    \end{enumerate}
    \item Affirmed.
\end{enumerate}

\subsubsection{Domain Names and Cybersquatting}

\begin{enumerate}
    \item Goal (as with all of trademark law): prevent the likelihood of 
    confusion.
    \item Two legal regimes:
    \begin{enumerate}
        \item Domestic: Anticybersquatting Consumer Protection Act (ACPA), 15 
        U.S.C. \S\ 1125(d). Creates a civil cause of action 
        when:\footnote{Casebook p. 911.}
        \begin{enumerate}
            \item \textbf{Bad faith intent to profit}.
            \item Name is \textbf{identical or confusingly similar} to a 
            distinctive mark.
            \item ...or \textbf{dilutive of a famous mark}.
            \item \textbf{Legitimate use factors} (\S\ 43(d)(1)(B)): trademark 
            or other IP rights of the person in the domain name; use of the 
            person's legal name; prior bona fide uses; offers to transfer for 
            financial gain; knowing registration of multiple confusingly 
            similar names.
        \end{enumerate}
        \item ICANN/UDRP: WIPO-administered arbitration. Applies when:
        \begin{enumerate}
            \item Name is \textbf{identical or confusingly similar}.
            \item \textbf{No rights} or legitimate interests.
            \item Name is registered and used in \textbf{bad faith}.
        \end{enumerate}
    \end{enumerate}
\end{enumerate}

\paragraph{\emph{PETA v. Doughney}}
~\\\\
Courts don't like bad faith registration.

\begin{enumerate}
    \item Doughney registered peta.org and labeled it ``People Eating Tasty 
    Animals.''\footnote{Casebook p. 921..}
    \item Doughney argued that his use was a parody. The Fourth Circuit found 
    that (1) Doughney had a bad faith intent to profit and (2) peta.org was 
    identical or confusingly similar to, or dilutive of, the PETA mark.
\end{enumerate}

\paragraph{New Domain Name Abuses}

\begin{enumerate}
    \item \textbf{Tasting}: registering domain names and returning them for a 
    full refund within five days.
    \item \textbf{Kiting}: automated tasting and re-tasting.
    \item \textbf{Spying}: squatters see the names you search for and register 
    them first.
\end{enumerate}

\subsection{Defenses}

\subsubsection{Abandonment}

\paragraph{\emph{Major League Baseball Properties, Inc. v. Sed Non Olet 
Denarius, Inc.}}
~\\\\
Abandonment requires (1) discontinuation of use (2) with no intent to resume.

\begin{enumerate}
    \item After the Dodgers moved to LA, SNOD opened restaurants in Brooklyn 
    called the ``Brooklyn Dodger.'' MLB sued.
    \item SNOD argued that MLB had failed to make commercial or trademark 
    use of the ``Brooklyn Dodgers'' mark for at least 25 years.
    \item A mark has been abandoned if its \textbf{``use has been discontinued 
    with no intent to resume.''}\footnote{Casebook p. 956.}
    \item Burden of proof:
    \begin{enumerate}
        \item On the party requesting cancellation.
        \item Registration is prima facie evidence of continuous use.
        \item Must show abandonment by a preponderance of the evidence.
    \end{enumerate}
    \item Los Angeles would have had to continue to use ``Brooklyn Dodgers''
    as the name of its baseball team. Only in this way would the public 
    continue to identify the name with the team. LA tried to ``warehouse'' the 
    mark.
    \item Lanham Act \S\ 45: ``Nonuse for three consecutive years shall be 
    prima facie evidence of abandonment.''
    \item Held for SNOD.
\end{enumerate}

\paragraph{Unsupervised Licenses: \emph{Dawn Donut Company, Inc. v. Hart's 
Food Stores, Inc.}}
~\\\\
Lack of supervision in a licensing arrangement can cause abandonment.

\begin{enumerate}
    \item Dawn was a wholesaler of donut mix. Hart's started using ``Dawn'' to 
    market its donuts.
    \item Hart's counterclaimed, arguing that Dawn had abandoned the mark 
    because of its licensees' \textbf{inadequate quality control and 
    supervision}.\footnote{Casebook p. 960.} The district court rejected the 
    counterclaim.
    \item Majority opinion: no abandonment.
    \item Judge Lumbard, dissenting:
    \begin{enumerate}
        \item Controlled licensing is not abandonment.
        \item However, inadequate supervision could lead to abandonment. 
        \item ``[U]nless the licensor exercises supervision and control over 
        the operations of its licensees the risk that the public will be 
        unwittingly deceived will be increased and this is precisely what the 
        [Lanham] Act is in part designed to prevent.''
        \item Should remand for further factfinding.
    \end{enumerate}
\end{enumerate}

\subsubsection{Nontrademark (or Nominative) Use, Parody, and the First 
Amendment}

\paragraph{Noncommercial Use: \emph{Mattel, Inc. v. MCA Records}}

\begin{enumerate}
    \item Mattel sued MCA over Aqua's ``Barbie Girl.''\footnote{Casebook p. 
    972.}
    \item The district court held that Aqua's use was nominative fair use, 
    there was no likelihood of confusion, and no dilution.
    \item Was there infringement?
    \begin{enumerate}
        \item The likelihood-of-confusion test doesn't accommodate expressive 
        uses. Trademark owners do not have the right to control uses beyond 
        the source-identifying function.\footnote{Casebook pp. 973--74.}
        \item \emph{Rogers}: literary titles do not violate the Lanham Act 
        unless it has ``no artistic relevance'' or ``misleads as to the source 
        or content'' of the work.\footnote{Casebook p. 975.}
        \item Under \emph{Rogers}, MCA's use of ``Barbie'' did not infringe 
        Mattel's trademark.
    \end{enumerate}
    \item Dilution? No.
    \begin{enumerate}
        \item MCA's use here was ``a classic blurring 
        injury.''\footnote{Casebook p. 976.}
        \item However, it falls under the noncommercial use exemption, 15 
        U.S.C. \S\ 1125(c)(4)(B).\footnote{Casebook p. 976.} \emph{Hoffman}: 
        any amount of noncommercial speech entitles the work to full First 
        Amendment protection.\footnote{Casebook p. 978.}
    \end{enumerate}
\end{enumerate}

\subsection{Remedies}

\subsubsection{Injunctions}

\begin{enumerate}
    \item Derives from trademarks' property traits.
\end{enumerate}

\subsubsection{Damages}

\paragraph{Infringer's Gain and Mark Owner's Loss: \emph{Lindy Pen Co., Inc. 
v. Bin Pen Corp.}}

\begin{enumerate}
    \item % TODO 990-996
\end{enumerate}

\paragraph{Corrective Advertising: \emph{Big O Tire Dealers, Inc. v. The 
Goodyear Tire \& Rubber Company}}
~\\\\
Courts can award ``corrective advertising'' damages when a junior user co-opts 
a senior user's mark, creating confusion as to the origin of the senior user's 
products.

\begin{enumerate}
    \item Big O, then a small company, developed a tire called ``Big foot.'' 
    Goodyear then launched a massive advertising campaign using the term 
    ``Bigfoot.''
    \item This case involves ``reverse confusion.'' Goodyear didn't trade on 
    Big O's goodwill. Instead, its use of the mark created confusion as to the 
    origin of Big O's products.
    \item The court here held that plaintiffs can recover for reverse 
    confusion.
    \item Goodyear spent about \$10 million on its national advertising 
    campaign. Big O sought to recover \$2.8 million on one of two 
    rationales:\footnote{Casebook p. 999.}
    \begin{enumerate}
        \item Big O operates in 14 states, or 28\% of states. \$2.8M is 28\% 
        of \$10M.
        \item The FTC often awards 25\% for corrective advertising in 
        misleading advertising cases.
    \end{enumerate}
    \item The court here awarded \$678,302, calculated as (1) 28\% of the 
    amount Goodyear spent, (2) reduced by 75\%, following the FTC 
    rule.\footnote{Casebook p. 1000.}
\end{enumerate}
