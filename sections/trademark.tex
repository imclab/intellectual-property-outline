\section{Trademark}

% TODO trademark cases

\subsection{Introduction}

\subsubsection{Background}

\begin{enumerate}
    \item Trademarks ``help to reduce information and transaction costs by 
    allowing customers to estimate the nature and quality of goods before 
    purchase.''\footnote{Casebook p. 763.}
    \item 1870: first federal trademark statute. Struck down as beyond 
    Congress's powers under the patent and copyright clause.\footnote{Casebook 
    p. 764.}
    \item 1881: second federal statute, passed under the Commerce Clause. 
    Significantly modified in 1905 and 1920.
    \item 1946: Lanham Act, 15 U.S.C. \S\S\ 1051 et seq.
    \item Trademark protections have generally expanded.\footnote{Casebook p. 
    765.}
\end{enumerate}

\subsubsection{A Brief Overview of Trademark Theory}

\begin{enumerate}
    \item Trademarks do not protect or reward novelty or invention. They 
    reward the first user of a mark.\footnote{Casebook p. 765.}
    \item Traditional trademark principles resemble tort law: preventing 
    unfair competition and consumer deception.
    \item Infringement actions can protect consumers.\footnote{Casebook p. 
    766.}
    \item Trademarks protect three kinds of investment:
    \begin{enumerate}
        \item Creation of the mark.
        \item Advertising.
        \item Product creation.
    \end{enumerate}
\end{enumerate}

\subsubsection{The Basic Economics of Trademark and Advertising}

\begin{enumerate}
    \item No consensus on the economic function of 
    trademarks.\footnote{Casebook p. 766.}
    \item Does advertising help or manipulate consumers? Does it communicate 
    valuable price or quality information, or does it artificially create 
    demand for nonessential features?
    \item Early commentators--``product differentiation theory'':
    \begin{enumerate}
        \item Trademarks are bad.
        \item Advertising unnaturally stimulates demand.
        \item Advertising perpetuates oligopoly. For instance, brand-name 
        drugs can sell for twice as much as chemically identical generic 
        drugs. Arguably, this hurts consumers.\footnote{Casebook pp. 766--67.}
    \end{enumerate}
    \item Now--``product information theory'':
    \begin{enumerate}
        \item Trademarks are good.
        \item Consensus that advertising cheaply conveys information to 
        consumers.
        \item Product's ``search characteristics'': price, color, shape, etc.
        \item Product's ``experience characteristics'': taste, long-term 
        durability. These are only apparent after purchase. Advertising and 
        trademarks can help identify these characteristics and encourage 
        repeat purchases.
    \end{enumerate}
    \item \emph{Consumer protection theory}: advertising cheaply conveys 
    valuable information to consumers.
    \item \emph{Producer incentive theory}: trademarks are ``essential 
    shorthand'' for consumers' positive associations with a 
    product.\footnote{Casebook p. 768.}
    \item Copyright protections have expanded beyond protecting against 
    consumer deception to a broader range of property-like 
    protections.\footnote{Casebook pp. 769--70.}
\end{enumerate}

\subsection{What Can Be Protected as a Trademark?}

\subsubsection{Trademarks, Trade Names, and Service Marks}

\begin{enumerate}
    \item \textbf{Trademark}: ``word, name, symbol, or device'' to identify 
    goods.\footnote{15 U.S.C. \S\ 1127; casebook p. 771.}
    \item \textbf{Service mark}: ``word, name, symbol, or device'' to identify 
    \emph{services}. Generally subject to the same rules as trademarks.
    \item \textbf{Trade names}: can only be registered if they \emph{identify 
    the source of particular goods}, rather than a company 
    alone.\footnote{Casebook p. 772.}
\end{enumerate}

\subsubsection{Certification and Collective Marks}

\begin{enumerate}
    \item \textbf{Certification mark}: ``word, name, symbol, or device'' to 
    certify characteristics of a product---i.e., a seal of approval. Used by 
    trade associations and commercial groups---e.g., the city of 
    Roquefort.\footnote{Casebook pp. 772--73.}
    \item \textbf{Collective mark}: trademark or service mark adopted by a 
    collective. It can be (1) used by its members to distinguish its products 
    from non-member products, or (2) indicating membership in a collective 
    group, like a union.\footnote{Casebook p. 773.} Marks of the first type 
    are useful in franchising arrangements.\footnote{Casebook p. 774.}
\end{enumerate}

\subsubsection{Trade Dress and Product Configuration}

\begin{enumerate}
    \item Design and packaging---and sometimes the design of the product 
    itself.\footnote{Casebook p. 774.}
\end{enumerate}

\subsubsection{Color, Fragrance, and Sounds: \emph{Qualitex Co. v. Jacobson 
Products Co., Inc.}}
~\\\\
Colors can be trademarked.

\begin{enumerate}
    \item Qualitex manufactured drycleaning pads with a distinctive green-gold 
    color.
    \item Circuit courts were split on whether colors could be trademarked.
    \item ``Symbol'' or ``device'' can refer to ``almost anything at all that 
    is capable of carrying meaning~.~.~.~''\footnote{Casebook p. 775.}
    \item Colors can signify brands. When a color signifies a product's 
    origin, it has taken on a ``secondary meaning.''\footnote{Casebook p. 776.}
    \begin{enumerate}
        \item \textbf{\enquote{\enquote{secondary meaning} is acquired when 
        \enquote{in the minds of the public, the primary significance of a 
        product feature~.~.~.~is to identify the source of the product rather 
        than the product itself}}}\footnote{Casebook p. 776.}
    \end{enumerate}
    \item Color meets the basic requirements for use as a trademark. It 
    can distinguish goods and identify their source without serving any other 
    function.\footnote{Casebook p. 777.}
    \item Jacobson made four arguments for why color alone should not be 
    granted trademark protections:
    \begin{enumerate}
        \item \emph{First}: it will create uncertainty (``shade confusion'').
        \begin{enumerate}
            \item But courts are skilled at making difficult decisions.
        \end{enumerate}
        \item \emph{Second}: colors are in limited supply.
        \begin{enumerate}
            \item Maybe, but that ``occasional problem'' does not justify a 
            ``blanket prohibition.'' 
            \item The \textbf{functionality doctrine} ``forbids the use of a 
            product's feature as a trademark where doing so will put a 
            competitor at a significant disadvantage because the feature is 
            \enquote{essential to the use or purpose of the article} or 
            \enquote{affects [its] cost or quality.}''\footnote{Casebook p. 
            779.} Here, if a limited supply of colors would harm competitors, 
            courts would not allow exclusive use of one of the colors as a 
            trademark.
        \end{enumerate}
        \item \emph{Third}: many older cases support its position.
        \begin{enumerate}
            \item No---those were pre--Lanham Act.
        \end{enumerate}
        \item \emph{Fourth}: firms can already use color as part of another 
        trademark or trade dress.
        \begin{enumerate}
            \item A company might have a reason to use a color instead of a 
            word or symbol.
        \end{enumerate}
    \end{enumerate}
\end{enumerate}

\subsection{Establishment of Trademark Rights}

\subsubsection{Distinctiveness}

\paragraph{Classification of Marks and Requirements for Protection}

\begin{enumerate}
    \item When a trademark can identify a \textbf{unique} product source, 
    rights are determined by priority of use.\footnote{Casebook p. 781.} These 
    types of marks are then further subdivided:
    \begin{enumerate}
        \item Arbitrary (Kodak, Ivory [for soap], Exxon).
        \item Fanciful (similar to arbitrary).
        \begin{enumerate}
            \item TODO difference between arbitrary and fanciful? 
            \emph{Zatarain's} conflates them---p. 784.
        \end{enumerate}
    \item Suggestive (Coppertone).
    \end{enumerate}
    \item All other marks---\textbf{generic trademarks}---require 
    \textbf{secondary meaning}. Types:
    \begin{enumerate}
        \item \textbf{Generic}: aspirin, cellophane. (Unprotectable even if 
        they have acquired secondary meaning.\footnote{Casebook p. 785.})
        \item \textbf{Descriptive}: describes something about the 
        goods or services---e.g., ``Tender Vittles'' for cat food or 
        ``Arthriticare'' for arthritis treatment.\footnote{Casebook p. 781.}
        \item \textbf{Geographic}.
        \item \textbf{Personal names}.
        \item ``Secondary meaning exists when buyers associate a descriptive 
        term with a single source of products.''\footnote{782.} Buyers do not 
        need to know the \emph{identity} of the source; they only need to know 
        that the product comes from a \emph{single} source.
    \end{enumerate}
\end{enumerate}

\paragraph{Descriptive Marks and Fair Use: \emph{Zatarain's, Inc. v. Oak Grove 
Smokehouse, Inc.}}
~\\\\
Descriptive terms can be trademarked, but the fair use defense allows 
competitors to use them in their original, descriptive sense. So, Zatarain's 
can have an exclusive right in the phrase ``Fish-Fri'' for its fish frying 
batter, but its competitors can use the words ``fish'' and ``fry'' to describe 
their own products.

\begin{enumerate}
    \item Zatarain's brought infringement suits for two of its registered 
    trademarks: ``Fish-Fri'' and ``Chick-Fri.''
    \item \textbf{Fair use} ``prevents a trademark registrant from 
    appropriating a descriptive term for its own use to the exclusion of 
    others, who may be prevent thereby from accurately describing their own 
    goods.''\footnote{Casebook p. 785.}
    \item ``Fish-Fri'':
    \begin{enumerate}
        \item Is ``Fish-Fri'' descriptive?
        \begin{enumerate}
            \item \emph{Dictionary}: refers to fried fish, so yes, 
            ``Fish-Fri'' is descriptive of the product (which is used to fry 
            fish).
            \item \emph{``Imagination test}'': if imagination is required to 
            associate the term with the product, the term is suggestive, not 
            descriptive. But here, no imagination is required to associate the 
            two.
            \item \emph{Competitive need}: do competitors need to use the term 
            to describe their products? Here, probably yes, because there is a 
            ``paucity of synonyms'' for ``fish'' and 
            ``fry.''\footnote{Casebook p. 787.}
            \item \emph{Actual use}: have competitors used the term to 
            describe their own products? Here, yes.
            \item So, ``Fish-Fri'' is descriptive.
        \end{enumerate}
        \item Does ``Fish-Fri'' have a secondary meaning?
        \begin{enumerate}
            \item Yes---survey evidence (of Louisiana, at least) and 
            circumstantial evidence showed that consumers associated 
            ``Fish-Fri'' with a specific source.
        \end{enumerate}
        \item Was there fair use?
        \begin{enumerate}
            \item Zatarain's cannot claim an exclusive right in the original, 
            descriptive sense of ``fish fry.'' So, its competitors are free to 
            use them in that sense.\footnote{Casebook p. 789.}
            \item Also, dissimilar trade dress would likely prevent consumer 
            confusion.
        \end{enumerate}
    \end{enumerate}
    \item ``Chick-Fri'':
    \begin{enumerate}
        \item Descriptive? Yes.\footnote{Casebook p. 789.}
        \item Secondary meaning? No (``paltry''---ha).\footnote{Casebook p. 
        790.}
    \end{enumerate}
\end{enumerate}

\paragraph{Genericness}

\begin{enumerate}
    \item A trademarked term must point to a specific source. 
    \item ``Toyota'' is a species. ``Car'' is a genus.\footnote{Casebook p. 
    794.}
    \item Terms are either born generic (``copier'') or become generic through 
    genericide (``Xerox'').
\end{enumerate}

\paragraph{\emph{The Murphy Door Bed Co., Inc. v. Interior Sleep Systems, Inc.}}
~\\\\
Marks that have become generic are no longer protectable as trademarks.

\begin{enumerate}
    \item 1918: Mr. Murphy won a patent for his bed design.\footnote{Casebook 
    p. 794.}
    \item 1925: Murphy incorporated the Murphy Door Bed Company.
    \item 1981--84: The PTO and TTAB denied trademark protection for ``Murphy 
    bed'' because the term had become generic.
    \item 1981: Defendants entered into a distribution agreement with the 
    plaintiffs, requiring it to identify Murphy beds as trademarked. Upon 
    learning of the TTAB's rejection, defendants incorporated a new company 
    with ``Murphy bed'' in the name.
    \item The district court held that ``Murphy bed'' was not generic because 
    it had a secondary meaning.\footnote{Casebook p. 795.}
    \item Rule: trademark protection is denied to a generic term unless there 
    is a secondary meaning.
    \item Who has the burden of proving genericness?
    \begin{enumerate}
        \item If the term was originally generic (e.g., ``Video Buyer's 
        Guide''), the plaintiff has the burden of showing that it's not 
        generic.
        \item If the term was originally specific but the public later 
        expropriated it, the defendant has the burden of showing that it is 
        generic.\footnote{Casebook p. 796.}
    \end{enumerate}
    \item Since this term was originally not generic, the defendants had the 
    burden of showing that it had become generic.
    \item Held: the term was generic because of (1) the TTAB's findings, (2) 
    the presence of ``Murphy bed'' as a generic term in dictionaries, and (3) 
    newspaper and magazine use as a generic term. Reversed.
\end{enumerate}

\paragraph{Notes on Genericness}

\begin{enumerate}
    \item Firms use advertising and lawsuits to protect their trademarks 
    against dilution, but these tactics might conflict with competition and 
    the First Amendment.\footnote{Casebook pp. 798--99.}
    \item Generally, generic terms are not trademarkable, but at least one 
    state has held that a generic term in public use can be protected if used 
    uniquely by a product developer (``Anti-Defamation 
    League'').\footnote{Casebook p. 799.}
    \item Brandeis on genericide: \emph{Kellogg Co. v. National Biscuit 
    Co.}\footnote{Casebook p. 799.}
    \item Non-word marks can be generic---e.g., grape 
    leaves.\footnote{Casebook p. 800.}
\end{enumerate}

\paragraph{Genericide, Language, and Policing Costs}

\begin{enumerate}
    \item Genericide takes away some rights that the creator earned through 
    his or her creativity.\footnote{Casebook p. 801.}
    \item Some creative works create \textbf{network externalities}: users 
    benefit from the fact that others use the work (e.g., technical standards 
    like TCP/IP). Generic marks can have network externalities. These 
    externalities derive partly from collective labor, rather than from the 
    labor of an individual creator, so it might make sense not to give the 
    creator exclusive rights.\footnote{Casebook pp. 802--03.}
    \item Should the cost of policing (e.g., ``you can't Xerox a Xerox on a 
    Xerox'') be a factor in the genericness analysis? Policing can indicate an 
    attempt to provide an alternative generic standard. On the other hand, it 
    can also indicate an attempt to maintain barriers to entry and to create 
    free advertising. Policing noncommercial uses also raises First Amendment 
    concerns.\footnote{Casebook p. 803.}
\end{enumerate}

\paragraph{Trade Dress and Secondary Meaning: \emph{Two Pesos, Inc. v. Taco 
Cabana, Inc.}}
~\\\\
Proof of secondary meaning is not required for an inherently distinctive trade 
dress to be protectable. Trade dress should be treated the same as verbal or 
symbolic trademarks.

\begin{enumerate}
    \item Two Pesos opened restaurants that mimicked the decor of Taco 
    Cabana's restaurants.
    \item The question was whether inherently distinctive trade dress (i.e., 
    restaurant decor) requires proof of secondary meaning to be 
    protectable.\footnote{Casebook p. 806.}
    \item \S\ 43(a) does not require secondary meaning for inherently 
    descriptive words or symbols. There is no basis in the statute for 
    distinguishing between trade dress and other types of marks. Thus, proof 
    of secondary meaning is not required where the trade dress is inherently 
    distinctive.\footnote{Casebook pp. 807--08.}
\end{enumerate}

\paragraph{Product Design and Inherent Distinctiveness: \emph{Wal-Mart Stores, 
Inc. v. Samara Brothers, Inc.}}
~\\\\
Product design is protectable only on a showing of secondary meaning.

\begin{enumerate}
    \item Wal-Mart copied the design of a line of clothing that Samara 
    manufactured.\footnote{Casebook p. 810.}
    \item Trade dress includes product design.\footnote{Casebook p. 811.}
    \item Marks can be distinctive in two ways: (1) inherently distinctive 
    (i.e., arbitrary, fanciful, or suggestive) or (2) secondary meaning.
    \item \emph{Qualitex} (see above) held that colors can never be inherently 
    distinctive.
    \item Similarly, product design can never be protected without a showing 
    of secondary meaning.\footnote{Casebook p. 813.}
    \item The court distinguishes three elements of trade dress:
    \begin{enumerate}
        \item Product \emph{design} (at issue here): not inherently 
        distinctive.
        \item Product \emph{packaging}: \emph{can} indicate origin and 
        therefore can be protectable.
        \item ``Tertium quid'': something ``akin to product 
        packaging.''\footnote{Casebook p. 814.}
    \end{enumerate}
\end{enumerate}

\paragraph{Conflicts among IP Protections}

\begin{enumerate}
    \item The ``principle of election'' allowed only one type of protection 
    per item. Rejected in \emph{Yardley}.\footnote{Casebook p. 815.}
    \item Utility patent protection generally trumps other forms of protection 
    regarding functional features because it has the highest 
    burden.\footnote{Casebook pp. 815--16.}
    \item Trademark law does not protect against copying of works whose 
    copyright protections have expired.\footnote{Casebook p. 816.}
\end{enumerate}

\paragraph{Functionality}

% TODO 817

\paragraph{\emph{TrafFix Devices, Inc. v. Marketing Displays, Inc.}}

% TODO 817-828

\subsubsection{Priority}

\begin{enumerate}
    \item \S\ 45(a): mark must be (1) used in commerce or (2) registered with 
    a bona fide intent to use in commerce.\footnote{Casebook p. 828.}
    \item The analysis depends on \textbf{how} and \textbf{how much} the mark 
    is used.
\end{enumerate}

\paragraph{\emph{Zazu Designs v. L'Oreal, S.A.}}

% TODO 829-38

\paragraph{Geographic Limitations on Trademark Use}

% TODO 838-40

\paragraph{Priority and Trademark Theory}

% TODO 840-45

\paragraph{Secondary Meaning in the Making}

% TODO 845-48

\subsubsection{Trademark Office Procedures}

% TODO 848-861

\subsubsection{Incontestability: \emph{Park 'N Fly, Inc. v. Dollar Park and 
Fly, Inc.}}

\begin{enumerate}
    \item Can an alleged infringer of an incontestable trademark defend on the 
    grounds that the trademark is descriptive?
    \item Here, Dollar Park counterclaimed for cancellation on the grounds 
    that Park 'N Fly was descriptive.\footnote{Casebook p. 862.}
    \item Held: nothing in the statute supports the view that a trademark, 
    once it has become incontestable, can be cancelled for being descriptive. 
    See Lanham Act \S\ 33(a), 15 U.S.C. \S\ 1115(a).
\end{enumerate}

\subsection{Infringement}

\subsubsection{Use as a Trademark}

% TODO 868-76

\subsubsection{Likelihood of Consumer Confusion}

% TODO 876-83

\subsubsection{Dilution}

\begin{enumerate}
    \item Courts can find infringement without confusion if dilution is 
    likely.\footnote{Casebook p. 889.}
    \item Federal Trademark Dilution Act of 1995 (FTDA), 15 U.S.C. 1125(c), 
    Lanham Act 43(c):\footnote{Casebook pp. 890--91.}
    \begin{enumerate}
        \item Applies if unauthorized use would harm the ``distinctiveness'' 
        or ``potency'' of the original mark.
        \item Mark must be \textbf{famous}. 43(c)(2)(A).
        \item \textbf{Blurring}: ``impairs the distinctiveness.'' Six factors. 
        43(c)(2)(B).
        \item \textbf{Tarnishment}: ``harms the reputation.'' 43(c)(2)(C).
        \item \textbf{Exclusions}: fair use, news, noncommercial use. 
        43(c)(3)(A).  \end{enumerate}
\end{enumerate}

\paragraph{Parody and Dilution: \emph{Louis Vuitton Malletier S.A. v. Haute 
Diggity Dog, LLC}}
~\\\\
Successful parodies do not dilute by blurring because they 
do not ``impair the distinctiveness'' (43(c)(2)(B)) of the original mark.

\begin{enumerate}
    \item Haute Diggity Dog made dog toys that pariodied high-fashion brands, 
    including ``Chewy Vuitton.''\footnote{Casebook pp. 892--93.}
    \item LVM argued that HDD's products would blur and tarnish its ``LOUIS 
    VUITTON'' mark.\footnote{Casebook p. 894.}
    \item Dilution by blurring? No.
    \begin{enumerate}
        \item Is the association between HDD's and LVM's marks likely to 
        impair the distinctiveness of LVM's marks?\footnote{Casebook p. 895.}
        \item LVM argued that any imitation of a famous mark dilutes the 
        mark.\footnote{Casebook p. 896.}
        \item After analyzing the six statutory factors (43(c)(2)(B)), the 
        court held that HDD's parody ``will not blur the distinctiveness of 
        the famous mark as a unique identifier of its 
        source.''\footnote{Casebook p. 898.}
        \item No dilution by blurring because HDD's parody was not likely to 
        \textbf{impair the distinctiveness} of LVM's marks.
    \end{enumerate}
    \item Dilution by tarnishment? No.
    \begin{enumerate}
        \item LVM argued that the possibility that a dog could choke on HDD's 
        toys harms the reputation of LVM's mark.\footnote{Casebook p. 898.}
        \item The court dismissed this claim, finding that it was unlikely 
        that a dog would choke ``on such a toy.''\footnote{Casebook p. 899.}
    \end{enumerate}
    \item Affirmed.
\end{enumerate}

\paragraph{Notes on Dilution}

% fixme 899--903

\paragraph{Dilution and ``Search Theory''}

% fixme 903-904

\paragraph{Trademark Preemption}

% fixme 904-06

\subsubsection{Extension by Contract: Franchising and Merchandising}

% fixme 906-11

\subsubsection{Domain Names and Cybersquatting}

% fixme 911-930

\subsection{Defenses}

\subsubsection{Abandonment}

% TODO 953-66

\newpage % TODO remove

\subsubsection{Nontrademark (or Nominative) Use, Parody, and the First 
Amendment}

\paragraph{Noncommercial Use: \emph{Mattel, Inc. v. MCA Records}}

% TODO see kozinski, trademarks unplugged, 68 nyu l rev 960

\begin{enumerate}
    \item Mattel sued MCA over Aqua's ``Barbie Girl.''\footnote{Casebook p. 
    972.}
    \item The district court held that Aqua's use was nominative fair use, 
    there was no likelihood of confusion, and no dilution.
    \item Was there infringement?
    \begin{enumerate}
        \item The likelihood-of-confusion test doesn't accommodate expressive 
        uses. Trademark owners do not have the right to control uses beyond 
        the source-identifying function.\footnote{Casebook pp. 973--74.}
        \item \emph{Rogers}: literary titles do not violate the Lanham Act 
        unless it has ``no artistic relevance'' or ``misleads as to the source 
        or content'' of the work.\footnote{Casebook p. 975.}
        \item Under \emph{Rogers}, MCA's use of ``Barbie'' did not infringe 
        Mattel's trademark.
    \end{enumerate}
    \item Dilution? No.
    \begin{enumerate}
        \item MCA's use here was ``a classic blurring 
        injury.''\footnote{Casebook p. 976.}
        \item However, it falls under the noncommercial use exemption, 15 
        U.S.C. \S\ 1125(c)(4)(B).\footnote{Casebook p. 976.} \emph{Hoffman}: 
        any amount of noncommercial speech entitles the work to full First 
        Amendment protection.\footnote{Casebook p. 978.}
    \end{enumerate}
\end{enumerate}

% TODO remove
\newpage

\subsection{Remedies}

\subsubsection{Injunctions}

\begin{enumerate}
    \item Derives from trademarks' property traits.
\end{enumerate}

\subsubsection{Damages}

\paragraph{Infringer's Gain and Mark Owner's Loss: \emph{Lindy Pen Co., Inc. 
v. Bin Pen Corp.}}

\begin{enumerate}
    \item % TODO 990-996
\end{enumerate}

\paragraph{Corrective Advertising: \emph{Big O Tire Dealers, Inc. v. The 
Goodyear Tire \& Rubber Company}}
~\\\\
Courts can award ``corrective advertising'' damages when a junior user co-opts 
a senior user's mark, creating confusion as to the origin of the senior user's 
products.

\begin{enumerate}
    \item Big O, then a small company, developed a tire called ``Big foot.'' 
    Goodyear then launched a massive advertising campaign using the term 
    ``Bigfoot.''
    \item This case involves ``reverse confusion.'' Goodyear didn't trade on 
    Big O's goodwill. Instead, its use of the mark created confusion as to the 
    origin of Big O's products.
    \item The court here held that plaintiffs can recover for reverse 
    confusion.
    \item Goodyear spent about \$10 million on its national advertising 
    campaign. Big O sought to recover \$2.8 million on one of two 
    rationales:\footnote{Casebook p. 999.}
    \begin{enumerate}
        \item Big O operates in 14 states, or 28\% of states. \$2.8M is 28\% 
        of \$10M.
        \item The FTC often awards 25\% for corrective advertising in 
        misleading advertising cases.
    \end{enumerate}
    \item The court here awarded \$678,302, calculated as (1) 28\% of the 
    amount Goodyear spent, (2) reduced by 75\%, following the FTC 
    rule.\footnote{Casebook p. 1000.}
\end{enumerate}
