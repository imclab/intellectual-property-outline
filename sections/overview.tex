\section{Overview}

\subsection{Patent}

\begin{enumerate}
    \item \textbf{Elements of patentability}.
    \begin{enumerate}
        \item \textbf{Subject matter} (35 U.S.C. \S\ 101).
        \begin{enumerate}
            \item Patentable: process, machine, manufacture, composition of 
            matter. \emph{Chakrabarty}.
            \item Not patentable: laws of nature, physical phenomena, abstract 
            ideas. \emph{Bilski}.
        \end{enumerate}
        \item \textbf{Utility} (35 U.S.C. \S\ 101).
        \begin{enumerate}
            \item Not patentable: compounds without known uses.  Courts don't 
            want to grant ``hunting licenses.'' \emph{Brenner}.  \item 
            Patentable: promising clinical results in mice. 
            \emph{Brana}.
            \item Deceptive inventions are patentable. \emph{Juicy Whip}. But 
            malicious inventions may not be.
        \end{enumerate}
        \item \textbf{Written description and enablement} (35 U.S.C. \S\ 112).
        \begin{enumerate}
            \item Must contain a \textbf{written description} that can 
            \textbf{``enable any person skilled in the art''} to make it.
            \item You can't claim more than you enable. \emph{Incandescent 
            Lamp}.
            \item Claims are limited to their written descriptions. 
            \emph{Gentry Gallery}. Applicants must disclose actual structures 
            and actual working examples. \emph{Ariad}.
        \end{enumerate}
        \item \textbf{Novelty} (35 U.S.C. \S\ 102).
        \begin{enumerate}
            \item Novelty: inventor vs. prior art. Is it new?
            \item Priority: inventor vs. inventor. Who invented it first?
            \item 1952 Act: first to invent.
            \begin{enumerate}
                \item ``~.~.~.~\textbf{priority} generally goes to the first 
                inventor to (1) reduce an invention to practice, without (2) 
                abandoning the invention.''\footnote{Casebook p. 248.}
                \item You don't get priority if you unreasonably delayed 
                reduction to practice. \emph{Griffith}.
            \end{enumerate}
            \item AIA (for patents filed on or after March 16, 2013): first to 
            file. 
            \begin{enumerate}
                \item Removed the priority requirement.
                \item ``Prior user rights'' are granted to non-patentees who 
                were using the invention before someone else patented it, with 
                some limitations (e.g., non-transferability).
            \end{enumerate}
            \item Novelty: newness. \S\ 102(a).
            \item Statutory bars: timeliness. \S\ 102(b).
            \item Earlier use, even if not published, establishes prior art. 
            ``Public'' means ``not explicitly private.'' \emph{Rosaire}.
            \item Publication of a single copy in a foreign university library 
            can satisfy the publication limitation. \emph{Hall}.
            \item Hidden use (i.e., ``non-informing public use'') can still be 
            public use. \emph{Egbert}.
            \item Experimental use is not public use. \emph{City of 
            Elizabeth}.
        \end{enumerate}
        \item \textbf{Nonobviousness} (35 U.S.C. \S\ 103).
        \begin{enumerate}
            \item ``Primary gatekeeper of the patent system.''
            \item No patent if the differences from prior art would have been 
            obvious to a PHOSITA.
            \item \S\ 103 requires an ``inventive leap.'' Minor obvious 
            improvements don't qualify. \emph{Graham}.
            \item The Federal Circuit's ``teaching, suggestion, or 
            motivation'' (TSM) test is too restrictive. The standard for 
            nonobviousness is flexible. Courts can use common sense. 
            \emph{KSR}.
            \item AIA: obviousness is determined at the \emph{filing date}, 
            not the invention date.
        \end{enumerate}
    \end{enumerate}
    \item \textbf{PTO administrative procedures}.
    \item \textbf{Infringement}.
    \begin{enumerate}
        \item \textbf{Claim interpretation}: Courts can look to extrinsic 
        sources, like dictionaries, to discern the meaning of claim terms. 
        Patent construction is a matter of law, not fact (i.e., for judges, 
        not juries).  \emph{Phillips}.
        \item \textbf{Literal infringement}: patent owner must ``show the 
        presence of every element'' of the claim. \emph{Larami}.
        \item \textbf{Doctrine of equivalents}: patent owner must show the 
        ``substantial equivalent'' of every element of the claim. 
        \emph{Larami}.
        \item Test for infringement: ``Does the accused product or process 
        contain elements \textbf{identical or equivalent} to each claimed 
        element of the patented invention?'' The new test focuses on 
        individual elements, so it's known as the \textbf{``all elements 
        rule.''} \emph{Warner-Jenkinson}.
        \item \textbf{Prosecution history estoppel}: prevents patent holders 
        from claiming subject matter that they surrendered during prosecution. 
        It's a bar to an infringement claim, but the patentee can rebut it (1) 
        by showing that the infringing equivalent was unforeseeable at the 
        time of application, (2) the rationale for the amendment during 
        prosecution bears only a tangential relationship to the infringing 
        equivalent, or (3) some other reason. \emph{Festo}.
    \end{enumerate}
    \item \textbf{Defenses} (35 U.S.C. 282).
    \begin{enumerate}
        \item \textbf{Inequitable conduct}: To prove inequitable conduct, an 
        accused infringer must show (1) \textbf{materiality} (that the PTO 
        would have thought the omitted fact was important) and (2) 
        \textbf{intent} to deceive, by a clear and convincing evidence 
        standard. Inequitable conduct on one claim will render the entire 
        patent unenforceable. \emph{Therasense}.
        \item \textbf{Exhaustion}.
        \begin{enumerate}
            \item The first sale exhausts a patentee's control over future 
            sales and repairs (but not reconstructions).
            \item Downstream \emph{selling} is permitted, but \emph{copying} 
            is not. \emph{Bowman}.
        \end{enumerate}
    \end{enumerate}
    \item \textbf{Remedies} (35 U.S.C. \S\S\ 283--87).
    \begin{enumerate}
        \item \textbf{Injunctions}: plaintiffs must satisfy a four-part test 
        to win a permanent injunction (irreparable injury, inadequacy of 
        monetary damages, balance of hardships between plaintiff and 
        defendant, and service of the public interest---see \emph{eBay}). 
        Kennedy, dissenting, would not allow injunctions for minor patents.
        \item \textbf{Damages}: (1) lost profits and (2) reasonable royalties.
        \begin{enumerate}
            \item Goal: find ``the difference between [patentee's] pecuniary 
            condition after the infringement, and what his condition would 
            have been if the infringement had not occurred.'' \emph{Yale 
            Lock}. Award the patentee's loss, not the infringer's gain.
            \item \emph{Panduit}: the patentee wants to show that would have 
            had sales that the infringer made. To do so, the patentee must 
            show the ``absence of acceptable noninfringing substitutes.''
            \item Reasonable royalty is a fallback when the patentee can't 
            show lost profits---the ``hypothetical bargain'' principle.
        \end{enumerate}
        \item \textbf{Willful infringement}: requires more than a showing of 
        negligence. \emph{Seagate}.
    \end{enumerate}
\end{enumerate}

\newpage

\subsection{Copyright}

% TODO 

\newpage

\subsection{Trademark}


\newpage
% TODO 

\subsection{Trade Secret}


\newpage
% TODO 

\subsection{State Law and Federal Preemption}
