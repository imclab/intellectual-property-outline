\section{Overview}

\subsection{Patent}

\begin{enumerate}
    \item \textbf{Elements of patentability}.
    \begin{enumerate}
        \item \textbf{Subject matter} (35 U.S.C. \S\ 101).
        \begin{enumerate}
            \item Patentable: process, machine, manufacture, composition of 
            matter. \emph{Chakrabarty}.
            \item Not patentable: laws of nature, physical phenomena, abstract 
            ideas. \emph{Bilski}.
        \end{enumerate}
        \item \textbf{Utility} (35 U.S.C. \S\ 101).
        \begin{enumerate}
            \item Not patentable: compounds without known uses.  Courts don't 
            want to grant ``hunting licenses.'' \emph{Brenner}.  \item 
            Patentable: promising clinical results in mice. 
            \emph{Brana}.
            \item Deceptive inventions are patentable. \emph{Juicy Whip}. But 
            malicious inventions may not be.
        \end{enumerate}
        \item \textbf{Written description and enablement} (35 U.S.C. \S\ 112).
        \begin{enumerate}
            \item Must contain a \textbf{written description} that can 
            \textbf{``enable any person skilled in the art''} to make it.
            \item You can't claim more than you enable. \emph{Incandescent 
            Lamp}.
            \item Claims are limited to their written descriptions. 
            \emph{Gentry Gallery}. Applicants must disclose actual structures 
            and actual working examples. \emph{Ariad}.
        \end{enumerate}
        \item \textbf{Novelty} (35 U.S.C. \S\ 102).
        \begin{enumerate}
            \item Novelty: inventor vs. prior art. Is it new?
            \item Priority: inventor vs. inventor. Who invented it first?
            \item 1952 Act: first to invent.
            \begin{enumerate}
                \item ``~.~.~.~\textbf{priority} generally goes to the first 
                inventor to (1) reduce an invention to practice, without (2) 
                abandoning the invention.''\footnote{Casebook p. 248.}
                \item You don't get priority if you unreasonably delayed 
                reduction to practice. \emph{Griffith}.
            \end{enumerate}
            \item AIA (for patents filed on or after March 16, 2013): first to 
            file. 
            \begin{enumerate}
                \item Removed the priority requirement.
                \item ``Prior user rights'' are granted to non-patentees who 
                were using the invention before someone else patented it, with 
                some limitations (e.g., non-transferability).
            \end{enumerate}
            \item Novelty: newness. \S\ 102(a).
            \item Statutory bars: timeliness. \S\ 102(b).
            \item Earlier use, even if not published, establishes prior art. 
            ``Public'' means ``not explicitly private.'' \emph{Rosaire}.
            \item Publication of a single copy in a foreign university library 
            can satisfy the publication limitation. \emph{Hall}.
            \item Hidden use (i.e., ``non-informing public use'') can still be 
            public use. \emph{Egbert}.
            \item Experimental use is not public use. \emph{City of 
            Elizabeth}.
        \end{enumerate}
        \item \textbf{Nonobviousness} (35 U.S.C. \S\ 103).
        \begin{enumerate}
            \item ``Primary gatekeeper of the patent system.''
            \item No patent if the differences from prior art would have been 
            obvious to a PHOSITA.
            \item \S\ 103 requires an ``inventive leap.'' Minor obvious 
            improvements don't qualify. \emph{Graham}.
            \item The Federal Circuit's ``teaching, suggestion, or 
            motivation'' (TSM) test is too restrictive. The standard for 
            nonobviousness is flexible. Courts can use common sense. 
            \emph{KSR}.
            \item AIA: obviousness is determined at the \emph{filing date}, 
            not the invention date.
        \end{enumerate}
    \end{enumerate}
    \item \textbf{PTO administrative procedures}.
    \item \textbf{Infringement}.
    \begin{enumerate}
        \item \textbf{Claim interpretation}: Courts can look to extrinsic 
        sources, like dictionaries, to discern the meaning of claim terms. 
        Patent construction is a matter of law, not fact (i.e., for judges, 
        not juries).  \emph{Phillips}.
        \item \textbf{Literal infringement}: patent owner must ``show the 
        presence of every element'' of the claim. \emph{Larami}.
        \item \textbf{Doctrine of equivalents}: patent owner must show the 
        ``substantial equivalent'' of every element of the claim. 
        \emph{Larami}.
        \item Test for infringement: ``Does the accused product or process 
        contain elements \textbf{identical or equivalent} to each claimed 
        element of the patented invention?'' The new test focuses on 
        individual elements, so it's known as the \textbf{``all elements 
        rule.''} \emph{Warner-Jenkinson}.
        \item \textbf{Prosecution history estoppel}: prevents patent holders 
        from claiming subject matter that they surrendered during prosecution. 
        It's a bar to an infringement claim, but the patentee can rebut it (1) 
        by showing that the infringing equivalent was unforeseeable at the 
        time of application, (2) the rationale for the amendment during 
        prosecution bears only a tangential relationship to the infringing 
        equivalent, or (3) some other reason. \emph{Festo}.
    \end{enumerate}
    \item \textbf{Defenses} (35 U.S.C. 282).
    \begin{enumerate}
        \item \textbf{Inequitable conduct}: To prove inequitable conduct, an 
        accused infringer must show (1) \textbf{materiality} (that the PTO 
        would have thought the omitted fact was important) and (2) 
        \textbf{intent} to deceive, by a clear and convincing evidence 
        standard. Inequitable conduct on one claim will render the entire 
        patent unenforceable. \emph{Therasense}.
        \item \textbf{Exhaustion}.
        \begin{enumerate}
            \item The first sale exhausts a patentee's control over future 
            sales and repairs (but not reconstructions).
            \item Downstream \emph{selling} is permitted, but \emph{copying} 
            is not. \emph{Bowman}.
        \end{enumerate}
    \end{enumerate}
    \item \textbf{Remedies} (35 U.S.C. \S\S\ 283--87).
    \begin{enumerate}
        \item \textbf{Injunctions}: plaintiffs must satisfy a four-part test 
        to win a permanent injunction (irreparable injury, inadequacy of 
        monetary damages, balance of hardships between plaintiff and 
        defendant, and service of the public interest---see \emph{eBay}). 
        Kennedy, dissenting, would not allow injunctions for minor patents.
        \item \textbf{Damages}: (1) lost profits and (2) reasonable royalties.
        \begin{enumerate}
            \item Goal: find ``the difference between [patentee's] pecuniary 
            condition after the infringement, and what his condition would 
            have been if the infringement had not occurred.'' \emph{Yale 
            Lock}. Award the patentee's loss, not the infringer's gain.
            \item \emph{Panduit}: the patentee wants to show that would have 
            had sales that the infringer made. To do so, the patentee must 
            show the ``absence of acceptable noninfringing substitutes.''
            \item Reasonable royalty is a fallback when the patentee can't 
            show lost profits---the ``hypothetical bargain'' principle.
        \end{enumerate}
        \item \textbf{Willful infringement}: requires more than a showing of 
        negligence. \emph{Seagate}.
    \end{enumerate}
\end{enumerate}

\newpage

\subsection{Copyright}

\begin{enumerate}
    \item \textbf{Requirements}.
    \begin{enumerate}
        \item \textbf{Originality}: must have a modicum of creativity. Facts 
        are not copyrightable, but the ``selection, coordination, and 
        arrangement'' of facts can be. \emph{Feist}.
        \item \textbf{Fixation} in a tangible medium.
        \item \textbf{Formalities}:
        \begin{enumerate}
            \item \emph{Notice}: encouraged but not required.
            \item \emph{Publication}: not required, but still relevant---see 
            p. 457.
            \item (See ``Analyzing Publication and Notice Problems'' below.)
            \item \emph{Registration}: has always been voluntary. Required to 
            bring an infringement suit.
            \item \emph{Deposit}: mandatory as part of registration, but 
            failure results only in a fine, not invalidity.
        \end{enumerate}
    \end{enumerate}
    \item \textbf{Subject matter}.
    \begin{enumerate}
        \item \textbf{Idea/expression dichotomy}: no protection for ``any 
        idea, procedure, process, system~.~.~.~.'' A useful idea cannot be 
        copyrighted, but an explanation of how to use it can be. \emph{Baker 
        v. Selden}.
        \item Computer menus are ``methods of operation'' (under 17 U.S.C. \S\ 
        102(b)) and therefore not copyrightable. \emph{Lotus v. Borland}.
        \item \textbf{Merger doctrine}: no protection for an expression if 
        it's the only feasible way to express an idea---e.g., mathematical 
        formulae.
        \item \textbf{Useful article doctrine}: spectrum from purely 
        functional design (e.g., industrial design---\emph{unprotectable}) to 
        pure expression (e.g., applied art, like a lamp 
        sculpture---\emph{protectable}). The only protected parts of a 
        ``picture, graphic, or sculpture'' work are the ``features that can be 
        identified separately from, and are capable of existing independently 
        of, the utilitarian aspects of the article.'' \S\ 101. No protection 
        if form and function are inextricable. \emph{Brandir}.
        \item \textbf{Government works}: not protected.
        \item \textbf{Types of protected works} (``illustrative, not 
        limitative''---\S\ 101):
        \begin{enumerate}
            \item Literary. Includes software.
            \item Pictorial, graphic, sculptural. See \emph{Brandir} and the 
            useful article doctrine above.
            \item Architectural. Applies to the overall design.
            \item Musical works and sound recordings. Musical works (sheet 
            music, lyrics, arrangements): fully protected. Sound recordings: 
            no traditional performance rights (although they now have a 
            digital performance right). For radio broadcasts, the owner of the 
            copyright of the \emph{composition} gets a royalty, but the 
            \emph{performer} does not.\footnote{Casebook p. 502.}
            \item Dramatic, pantomime, choreographic.
            \item Motion pictures and audiovisual works. Includes soundtracks.
            \item Semiconductors and vessel hulls: sui generis.
            \item Derivative works: owner of the original controls. 
            Compilations: must involve some degree of creativity.
        \end{enumerate}
    \end{enumerate}
    \item \textbf{Ownership and duration}.
    \begin{enumerate}
        \item \textbf{Initial ownership rights}:
        \begin{enumerate}
            \item Vests at the moment of creation.
            \item \textbf{Works for hire}: employer owns. \S\ 201(b). A 
            commissioned work is a work for hire if it falls into one of nine 
            enumerated categories (``work made for hire,'' \S\ 101). Courts 
            apply the common law agency rule. See \emph{CCNV v. Reid} and p. 
            512 top.
            \item \textbf{Joint work}: required (1) a copyrightable work, (2) 
            two or more authors, and (3) intent to merge contributions into an 
            inseparable or interdependent unitary whole. \emph{Aalmuhammed v.  
            Lee}.
            \item \textbf{Collective works}: copyright in a contribution is 
            distinct from copyright in the entire collection. \S\ 201(c).
            \item (See ``Analyzing IP Ownership Problems.'')
        \end{enumerate}
        \item \textbf{Duration and renewal}:
        \begin{enumerate}
            \item \textbf{Duration}: life of the author plus 70 years.
            \item To determine duration for works published earlier, see the 
            chart on pp. 527--29.
            \item Renewal was required under the 1909 Act, but is \textbf{no 
            longer required}.
            \item Copyright law \textbf{restricts the alienability of 
            copyrights} in various ways.\footnote{Casebook p. 532.}
            \item \textbf{Division and transfer}:
            \begin{enumerate}
                \item 1909 Act: copyright holders could not divide their 
                right. They could assign the whole thing, but any lesser 
                transfer was considered a license.
                \item 1976 Act: Eliminated restrictions on indivisibility. 
                Allows for exclusive licenses. Any holder of an exclusive 
                license can bring an infringement suit.\footnote{Casebook p.  
                533.}
            \end{enumerate}
            \item \textbf{Renewal and termination of transfer}:
            \begin{enumerate}
                \item 1909 Act: upon renewal, copyright holders could reclaim 
                copyright interests that they had licensed. (But most 
                licensees insisted on advance assignment.)
                \item 1976 Act: eliminated the renewal requirement. Copyright 
                holders could terminate transfers of copyright between the 
                thirty-fifth and fortieth year from the execution of the 
                transfer.  Congress wanted to give stronger rights to authors 
                and their families.  This violates freedom of contract, but it 
                compensates for publishers' ``unequal bargaining 
                power.''\footnote{Casebook p. 533.}
            \end{enumerate}
        \end{enumerate}
    \end{enumerate}
    \item \textbf{Rights of owners}.
    \begin{enumerate}
        \item \textbf{Reproduction}: proving infringement requires (1) copying 
        (proof of access, striking similarity) and (2) improper appropriation.
        \begin{enumerate}
            \item Circuits are split on whether you must show access. Second: 
            no need to prove access if there is enough similarity. Seventh: 
            you \emph{must} show evidence of access.
            \item \textbf{Levels of abstraction}: infringement is not limited 
            to the literal text. But there is a level of abstraction where a 
            work is no longer protected [because at that level it's an idea, 
            not an expression]. \emph{Nichols v. Universal}.
            \item Software: ``those elements of a computer program that are 
            necessarily incidental to its function are~.~.~.~unprotectable.'' 
            \emph{Computer Associates v. Altai}. Apply the 
            \textbf{abstraction, filtration, and comparison} test:
            \begin{enumerate}
                \item \textbf{Abstract} the program into its structural parts.
                \item \textbf{Filter} out the merged ideas and public domain 
                parts.
                \item \textbf{Compare} the remaining protected pieces with the 
                structural parts of the allegedly infringing program.
            \end{enumerate}
        \end{enumerate}
        \item \textbf{Derivative works}: owner has the exclusive right. \S\ 
        106(2). Stock scenes and characters are not copyrightable, but 
        specific characters are. \emph{Anderson v. Stallone}.
        \item \textbf{Distribution}: owner has exclusive right---but the right 
        extends only to the \textbf{first sale}. \emph{Kirtsaeng v. Wiley}.
        \item \textbf{Public performance and display}:
        \begin{enumerate}
            \item 1972: no public performance rights for \emph{analog} sound 
            recordings. But there are performance rights for \emph{digital} 
            sound recordings (1995).
            \item \textbf{Public interest exemption} (\S\ 110): generally 
            applies to educational, free, or charitable performances and 
            displays.\footnote{Casebook p. 592.} Fairness in Music Licensing 
            Act (1998) broadened exemptions for homes, small business, 
            restaurants, and certain larger establishments.
            \item \textbf{Compulsory licenses}---five areas: cable, satellite, 
            jukeboxes, public broadcasting, webcasting.\footnote{Casebook p. 
            593--92.}
            \item \textbf{Moral rights}: applies only to visual works. 
            Resulted from Berne, 1990.
        \end{enumerate}
    \end{enumerate}
    \item \textbf{Indirect liability}.
    \begin{enumerate}
        \item \textbf{Indirect liability}: applies to those who 
        \textbf{contribute to, induce, or profit from} infringement, or those 
        who \textbf{sell products} that others can use to 
        infringe.\footnote{Casebook p. 598.}
        \item \textbf{Vicarious liability}: applies when someone exercises 
        direct control over another.
        \item \textbf{Substantial noninfringing uses} can insulate a 
        manufacturer from contributory liability. \emph{Sony v. Universal}.
    \end{enumerate}
    \item \textbf{Fair use}.
    \begin{enumerate}
        \item Four factors (\S\ 107):
        \begin{enumerate}
            \item \textbf{Purpose and character} (e.g., commercial/nonprofit).
            \item \textbf{Nature of the copyrighted work}.
            \item \textbf{Amount} and substantiality of the portion used.
            \item \textbf{Effect of the use upon the potential market} or 
            value of the original.
            \item (Detailed analysis of the four factors: see \emph{American 
            Geophysical Union v. Texaco}.)
        \end{enumerate}
        \item Overcoming fair use requires the copyright holder to show a 
        likelihood of harm. \emph{Sony v. Universal}.
        \item Critiques of \emph{Sony}'s ``substantial noninfringing use 
        standard:
        \begin{enumerate}
            \item If 85\% of uses are infringing and 15\% are infringing, 
            there is no liability. This is inefficient and costly.
            \item Menell and Nimmer: apply a ``reasonable alternative design'' 
            standard (from tort law). A different product design could 
            significantly reduce infringement at very little cost.
            \item Lanier: machines are shaping society instead of the other 
            way around.
        \end{enumerate}
        \item Commercial uses are presumptively \emph{not} fair uses. The 
        effect on the market is a key factor. \emph{Harper \& Row v. Nation}.
        \item Wendy Gordon, fair use as market failure: fair use makes sense 
        when there are no opportunities for private agreements---unfavorable 
        reviews of a book.
        \item \textbf{Parodies} are fair use because they are transformative, 
        although courts should consider the effects on derivative markets. 
        \emph{Campbell v. Acuff-Rose}.
        \item \textbf{Remixes} can copy entire works if the use is 
        transformative. \emph{Bill Graham Archives v. DK}.
        \item \textbf{Reverse engineering}: ``We conclude that where 
        disassembly is the \textbf{only way to gain access to the ideas and 
        functional elements} embodied in a copyrighted computer program and 
        where there is a legitimate reason for seeking such access, 
        disassembly is a fair use of the copyrighted work, as a matter of 
        law.'' \emph{Sega v. Accolade}.
    \end{enumerate}
    \item \textbf{Digital copyright law}.
    \begin{enumerate}
        \item \textbf{Anticircumvention} prohibits (1) circumventing or 
        enabling evasion of TPMs for \emph{access control} and (2) enabling 
        evasion of TPMs for \emph{copy control}. \S\S\ 1201(a), (b). 
        Circumventing authentication violates the DMCA. \emph{Realnetworks}.
        \item \textbf{Safe harbors}: OSPs must (1) have a termination policy 
        for repeat infringes, (2) must adopt ``standard technological 
        measures'' to screen copyrighted works, and (3) must identify a notice 
        and takedown agent. To be liable, the OSP must have \textbf{knowledge 
        or awareness of specific infringing activity}. \emph{Viacom v. 
        YouTube}.
        \item \textbf{Fair use}: Google Image search thumbnails are fair use. 
        A search engine ``transforms the image into a pointer, directing a 
        user to a source of information.'' \emph{Perfect 10 v. Amazon}.
        \item \textbf{Webcasting and compulsory licenses}:
        \begin{enumerate}
            \item Current law: ``interactive services'' must negotiate 
            directly with copyright owners; non-interactive services have a 
            compulsory license.
            \item Lessig, Fisher: there should be an Internet-wide compulsory 
            license for downloading music. We should separate compensation 
            from control. See \emph{Free Culture}.
        \end{enumerate}
    \end{enumerate}
    \item \textbf{International copyright law}.
    \begin{enumerate}
        \item Lack of formalities has led to nearly worldwide protection for 
        most works.\footnote{Casebook p. 742.}
        \item Berne, TRIPS.
    \end{enumerate}
    \item \textbf{Enforcement and remedies}.
    \begin{enumerate}
        \item \textbf{Injunctions}---four factors 
        (\emph{eBay}\footnote{Casebook p. 753.}):
        \begin{enumerate}
            \item Irreparable injury;
            \item The inadequacy of other remedies;
            \item Considering the balance of hardships between the plaintiff 
            and defendant, an injunction is warranted; and
            \item ``~.~.~.~that the public interest would not be 
            disserved~.~.~.''
        \end{enumerate}
        \item \textbf{Damages}: should be compensatory, not punitive.  
        \emph{Sheldon v. Metro-Goldwyn}.
        \item \textbf{Statutory damages}:
        \begin{enumerate}
            \item Available to copyright holders who registered their 
            work.\footnote{Casebook p. 759.}
            \item Unintentional infringement: \$750 to \$30,000 (or \$200 as a 
            lower bound if the infringer had no reason to believe the activity 
            constituted infringement) per infringed work.
            \item Willful infringement: up to \$150,000 per infringed work.
        \end{enumerate}
        \item \textbf{Attorney fees}: courts have 
        discretion.\footnote{Casebook p. 761.}
    \end{enumerate}
\end{enumerate}

\newpage

\subsection{Trademark}

\begin{enumerate}
    \item \textbf{Subject matter}.
    \begin{enumerate}
        \item \textbf{Trademark}: ``word, name, symbol, or device'' to 
        \textbf{identify and distinguish} goods and to \textbf{indicate the 
        source} of goods. The source can be unknown; what matters is that the 
        trademark is a unique identifier.\footnote{15 U.S.C. \S\ 1127; 
        casebook p.  771.}
        \item \textbf{Service mark}: ``word, name, symbol, or device'' to 
        identify \emph{services}. Generally subject to the same rules as 
        trademarks. E.g., ``Hyatt hotel services.''
        \item \textbf{Trade names}: can only be registered if they 
        \emph{identify the source of particular goods}, rather than a company 
        alone.\footnote{Casebook p. 772.}
        \item \textbf{Slogans}: ``the greatest show on earth.''
        \item \textbf{Certification mark}: ``word, name, symbol, or device'' 
        to certify characteristics of a product---i.e., a seal of approval. 
        Used by trade associations and commercial groups---e.g., ``Good 
        Housekeeping,'' the city of Roquefort.\footnote{Casebook pp. 772--73.}
        \item \textbf{Collective mark}: trademark or service mark adopted by a 
        collective. It can be (1) used by its members to distinguish its 
        products from non-member products, or (2) indicating membership in a 
        collective group, like a union.\footnote{Casebook p. 773.} Marks of 
        the first type are useful in franchising 
        arrangements.\footnote{Casebook p. 774.}
        \item \textbf{Trade dress and product configuration}: design and 
        packaging---and sometimes the design of the product 
        itself.\footnote{Casebook p. 774.}
        \item Colors, fragrances, sounds. \emph{Qualitex}.
    \end{enumerate}
    \item \textbf{Establishment of rights}.
    \begin{enumerate}
        \item \textbf{Distinctiveness}:
        \begin{enumerate}
            \item \textbf{Distinctive} (protectable): arbitrary, fanciful, 
            suggestive.
            \item \textbf{Non-distinctive} (require secondary meaning): 
            descriptive, geographic, personal names, colors, fragrances, 
            sounds.
            \item \textbf{Generic} (not protectable): born generic or suffer 
            genericide. \emph{Murphy Door Bed}. Firms try to police the use of 
            their marks (``You can't Xerox a Xerox on a Xerox''), but they 
            bump into the First Amendment.
            \item Descriptive terms can be trademarked, but the \textbf{fair 
            use} defense allows competitors to use them in their original, 
            descriptive sense. Fair use applies only to descriptive marks. 
            \emph{Zatarain's}.
            \item Anybody can petition for \textbf{cancellation} of a name 
            that has become generic. 15 U.S.C. \S\ 1064 (Lanham \S\ 14). 
            Courts can also declare trademarks to be generic.
            \item Proof of secondary meaning is \textbf{not required for an 
            inherently distinctive trade dress to be protectable}. Trade dress 
            should be treated the same as verbal or symbolic trademarks. 
            \emph{Two Pesos}.
            \item Product \emph{design}, like color, is not inherently 
            distinctive, so it's protectable only on a showing of secondary 
            meaning. project \emph{packaging} can be inherently distinctive, 
            so it doesn't require secondary meaning. \emph{Wal-Mart}.
            \item \textbf{Functional} trade dress is unprotected. Utility 
            patents can support a finding that trade dress is functional, 
            though courts are split.
        \end{enumerate}
        \item \textbf{Priority}:
        \begin{enumerate}
            \item \S\ 45(a): mark must be (1) used in commerce or (2) 
            registered with a bona fide intent to use in 
            commerce.\footnote{Casebook p. 828.}
            \item Knowledge that another person plans to use a mark does not 
            prevent you from using it. However, the \textbf{intent to use} 
            rule lets someone ``reserve'' a mark if it is actually used within 
            six months (extendable to up to three years for good cause). \S\ 
            1051(b), Lanham \S\ 1(b). \emph{Zazu}.
            \item Registration establishes national priority. For unregistered 
            marks, concurrent users can expand their geographic areas unless 
            it causes confusion.
            \item Two types of \textbf{concurrent use}: (1) different products 
            in the same market, (2) different geographic markets.
        \end{enumerate}
        \item \textbf{Trademark office procedures}:
        \begin{enumerate}
            \item \textbf{Principal register}: allows nationwide constructive 
            notice and use, and incontestable status after five years.
            \item \textbf{Secondary register}: to register in foreign 
            countries, it used to be required for a mark to be registered 
            domestically first.
            \item Marks can be rejected for being immoral, deceptive, or 
            scandalous.
            \item Can also be rejected for being \textbf{merely descriptive}, 
            \textbf{primarily geographically descriptive}, or 
            \textbf{primarily geographically deceptively 
            misdescriptive}---e.g., Wisconsin cheese, Washington apples. 
            Specific exceptions for appellations of origin for wine and 
            spirits (e.g., Bourbon) and certification marks (e.g., Roquefort).
            \item ``Nantucket'' for men's shirts is ok because buyers are 
            unlikely to be deceived about the shirts' origin. 
            \emph{Nantucket}.
        \end{enumerate}
        \item \textbf{Incontestability}:
        \begin{enumerate}
            \item Descriptiveness does not outweigh incontestability. For the 
            for requirements to gain incontestable status, see \S\ 1065. 
            \emph{Park 'N Fly}.
            \item Defenses that survive incontestability: \S\ 33(b). Also: 
            genericness, antitrust, equitable doctrines (laches, etc.).
        \end{enumerate}
    \end{enumerate}
    \item \textbf{Infringement}.
    \begin{enumerate}
        \item Infringement under the Lanham Act requires \textbf{``use in 
        commerce.''} \S\ 1114.  Liability arises when the use is ``likely to 
        cause confusion [or mistake or deception].'' \S\ 1125(a). 
        \emph{Rescuecom}.
        \item There must be a \textbf{likelihood of customer confusion}. See 
        the seven factors from \emph{Sleekcraft}.
        \item \textbf{Dilution}.
        \begin{enumerate}
            \item Five elements (\S\ 1125(c)):
            \begin{enumerate}
                \item Mark must be \textbf{famous}. Factors: 42(c)(2)(A).
                \item Mark must be \textbf{distinctive} (spectrum: generic to 
                arbitrary/fanciful).
                \item Junior user makes commercial use of the mark in 
                commerce...
                \item ...after the senior mark has become famous.
                \item Two kinds:
                \begin{itemize}
                    \item \textbf{Blurring}: ``impairs the distinctiveness.'' 
                    Six factors. \S\ 42(c)(2)(B).
                    \item \textbf{Tarnishment}: ``harms the reputation.'' \S\ 
                    43(c)(3)(A).
                \end{itemize}
            \end{enumerate}
            \item Limiting factors (43(c)):
            \begin{enumerate}
                \item Can only win an injunction.
                \item Registration of the defendant's mark is a complete 
                defense.
                \item \textbf{Exclusions}: fair use, news, noncommercial use.  
            \end{enumerate}
            \item Successful \textbf{parodies} do not dilute by blurring 
            because they do not ``impair the distinctiveness'' (43(c)(2)(B)) 
            of the original mark. But parody is a defense only if the 
            trademark is not being used qua trademark. \emph{Louis Vuitton}.
        \end{enumerate}
        \item \textbf{Cybersquatting}.
            \begin{enumerate}
                \item Domestic: Anticybersquatting Consumer Protection Act 
                (ACPA), 15 U.S.C. \S\ 1125(d). Creates a civil cause of action 
                when:\footnote{Casebook p. 911.}
                \begin{enumerate}
                    \item \textbf{Bad faith intent to profit}.
                    \item Name is \textbf{identical or confusingly similar} to 
                    a distinctive mark.
                    \item ...or \textbf{dilutive of a famous mark}.
                    \item \textbf{Legitimate use factors} (\S\ 43(d)(1)(B)): 
                    trademark or other IP rights of the person in the domain 
                    name; use of the person's legal name; prior bona fide 
                    uses; offers to transfer for financial gain; knowing 
                    registration of multiple confusingly similar names.
                \end{enumerate}
                \item ICANN/UDRP: WIPO-administered arbitration. Applies when:
                \begin{enumerate}
                    \item Name is \textbf{identical or confusingly similar}.
                    \item \textbf{No rights} or legitimate interests.
                    \item Name is registered and used in \textbf{bad faith}.
                \end{enumerate}
            \end{enumerate}
        \item Courts don't like bad faith registration. \emph{PETA}.
    \end{enumerate}
    \item \textbf{Defenses}.
    \begin{enumerate}
        \item \textbf{Abandonment}: requires (1) discontinuation of use (2) 
        with no intent to resume. \emph{MLB v. SNOD}.
        \item \textbf{Lack of supervision} in a licensing arrangement can 
        cause abandonment. \emph{Dawn Donut}.
        \item \textbf{Nominative use}: trademark owners do not have the right 
        to control uses beyond the source-identifying function. \emph{Mattel}.
        \item Noncommercial use exception: expressive uses are entitled to 
        First Amendment protections. \emph{Mattel}.
    \end{enumerate}
    \item \textbf{Remedies}.
    \begin{enumerate}
        \item \textbf{Injunctions}: the default.
        \item \textbf{Damages} (\S\ 1117(a)):
        \begin{enumerate}
            \item Plaintiff can recover:
            \begin{enumerate}
                \item Defendant's \textbf{profits}.
                \item Plaintiff's \textbf{damages}.
                \item \textbf{Costs} of the action.
            \end{enumerate}
            \item Courts can award up to \textbf{three times} actual damages.
        \end{enumerate}
        \item Courts can award \textbf{``corrective advertising''} damages 
        when a junior user co-opts a senior user's mark, creating confusion as 
        to the origin of the senior user's products.
    \end{enumerate}
\end{enumerate}

\newpage

\subsection{Trade Secret}

\begin{enumerate}
    \item \textbf{Structure of a claim}.
    \begin{enumerate}
        \item Eligible \textbf{subject matter}:
        \begin{enumerate}
            \item Derives value from remaining secret.
            \item Actually kept secret.
        \end{enumerate}
        \item Owner has taken \textbf{reasonable precautions} to keep it 
        secret.
        \item \textbf{Misappropriation}.
    \end{enumerate}
    \item \textbf{Subject matter}.
    \begin{enumerate}
        \item There must be value in keeping it a secret. Factors include: 
        level of confidentiality, the value to the secret holder, and the cost 
        of developing the secret. \emph{Metallurgical Industries v. Fourtek}.
        \item Evidence of \textbf{reasonable precautions} makes acquisition 
        through ``proper means'' very unlikely. \emph{Rockwell Graphic}.
        \item \textbf{Public disclosure} destroys the secret, but as long as 
        the secret remains secret, it's protectable. A non-disclosure 
        agreement \emph{may} be good enough. \emph{Data General}.
        \item Trade secrets can be disclosed in several ways: (1) publication, 
        (2) sale of a product that embodies the secret, (3) disclosure by 
        someone other than the owner, (4) inadvertent disclosure, or (5) 
        forced disclosure by government agencies.\footnote{Casebook p. 60-63.}
    \end{enumerate}
    \item \textbf{Misappropriation}.
    \begin{enumerate}
        \item Where a person has taken \textbf{reasonable precautions} to 
        preserve secrecy, taking the secret is improper. Illegal conduct or a 
        breach of a confidential relationship are not required. \emph{duPont}.
        \item Confidential relationships can be implied from the 
        circumstances. \emph{Smith v. Dravo}.
        \item Reverse engineering is a legitimate use of another's trade 
        secret. \emph{Kadant}.
        \item \textbf{IP assignment clauses}: some states, like CA, don't 
        allow employers to require assignment of completely independent 
        employee inventions. See \emph{Roberts v. Sears}.
        \item \textbf{Trailer clauses} (or follow-on clauses): IP assignment 
        agreement extends X months after the period of employment.
        \item California invalidates \textbf{noncompete clauses} in employment 
        contracts, unless they are necessary to protect the employer's trade 
        secrets. The policy is to promote ``free and full practice of one's 
        profession.'' But California is a distinct minority. Most states 
        enforce noncompetes. \emph{Edwards v. Arthur Andersen}.
        \item \textbf{Inevitable disclosure}: possible to win an injunction to 
        prevent employees from defecting to competitors because they will 
        inevitably disclose trade secrets. But some courts (e.g., Cal. App.) 
        have rejected it.
    \end{enumerate}
    \item \textbf{Agreements to keep secrets}.
    \begin{enumerate}
        \item If the information doesn't meet the requirements to be a trade 
        secret, can the parties still agree by contract to keep it secret?  
        Yes---\emph{Warner-Lambert}. But many courts disagree.
    \end{enumerate}
    \item \textbf{Remedies}.
    \begin{enumerate}
        \item The UTSA allows injunctions, damages, and attorney's 
        fees.\footnote{Casebook p. 111--12.}
        \item Criminal provisions: CA penal code \S\ 499(c); federal Economic 
        Espionage Act (1996), \S\ 1831.
        \item \textbf{Head-start injunction}: courts may grant it when (1) a 
        permanent injunction would harm employee mobility and the public 
        interest, and (2) no injunction would give the culpable employees an 
        unfair head start in the market. \emph{Winston v. 3M}.
    \end{enumerate}
\end{enumerate}

\newpage

\subsection{State Law and Federal Preemption}

\begin{enumerate}
    \item \textbf{Misappropriation}.
    \begin{enumerate}
        \item News stories can be protect as a form of ``quasi-property'' 
        which can be protected against misappropriation by direct competitors. 
        \emph{INS v. AP}.
    \end{enumerate}
    \item \textbf{Protection by contract}.
    \begin{enumerate}
        \item Software shrinkwrap license agreements are enforceable, even if 
        the consumer can't read them until he has bought the product. 
        \emph{ProCD v. Zeidenberg}.
        \item A statement of agreement is ``essential to the formation of a 
        contract.'' \emph{Specht v. Netscape}.
    \end{enumerate}
    \item \textbf{Idea submissions}.
    \begin{enumerate}
        \item At least under New York law, breach of contract actions for 
        stealing ideas are viable if the idea was novel \emph{to the buyer}, 
        even if it was generally known. \emph{Nadel v. Play-by-Play}.
        \item To be enforceable, there has to be a contract \emph{before} 
        disclosure of the idea. Don't blurt out your valuable ideas. 
        \emph{Desny v. Wilder}.
    \end{enumerate}
    \item \textbf{Right of publicity}.
    \begin{enumerate}
        \item Two theories: privacy and property. Should publicity rights be 
        assignable or descendable?
        \item Tort law protect singers against imitation of their voices. 
        \emph{Midler v. Ford}.
        \item \textbf{Transformative use test}: did the defendant use the 
        celebrity image as one input among many, adding more expressive 
        elements? If so, the use is transformative (and unlikely to harm the 
        market because it is not a good substitute for the original). 
        \emph{Comedy III v. Saderup}.
    \end{enumerate}
    \item \textbf{Copyright preemption}.
    \begin{enumerate}
        \item Two sources: Supremacy Clause and 17 U.S.C. \S\ 301.
        \item \textbf{Two-part preemption test}: a claim is preempted if it 
        (1) comes within the \textbf{subject matter} of copyright and (2) the 
        rights granted under the state law are \textbf{equivalent} to any of 
        the exclusive rights within the general scope of copyright as defined 
        in \S\ 106.
        \item Some contracts involving IP are enforceable, but not those that 
        (1) \textbf{expand affirmative exclusive rights} (e.g., duration) or 
        (2) \textbf{narrow or exclude statutory or common-law exceptions} 
        (e.g., fair use). \emph{ProCD v. Zeidenberg (II)}.
    \end{enumerate}
    \item \textbf{Patent preemption}.
    \begin{enumerate}
        \item Federal IP laws do not preempt state trade secret law. 
        \emph{Kewanee Oil v. Bicron}.
        \item 
    \end{enumerate}
\end{enumerate}
