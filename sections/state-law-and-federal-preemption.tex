\section{State Law and Federal Preemption}

\begin{enumerate}
    \item Six ways that state law protects IP (other than trade secret and 
    trademark):\footnote{Casebook p. 1005.}
    \begin{enumerate}
        \item Misappropriation.
        \item Contract.
        \item Idea submissions (implied contracts).
        \item Right of publicity.
        \item Trespass to chattel.
    \end{enumerate}
\end{enumerate}

\subsection{The Tort of Misappropriation}

\subsubsection{Quasi-Property: \emph{International News Service v. Associated 
Press}}

News stories can be protect as a form of ``quasi-property'' which can be 
protected against misappropriation by direct competitors.

\begin{enumerate}
    \item INS and AP competed to deliver ``hot news'' to newspapers across the 
    country. INS began copying AP's stories and sending them to newspapers 
    itself---for instance, by copying stores published on the East Coast and 
    sending them to West Coast newspapers.\footnote{Casebook p. 1006--07.}
    \item Can AP win an injunction?
    \item Justice Pitney:
    \begin{enumerate}
        \item INS argued that AP did not have a property right in its 
        published stories.
        \item News articles are copyrightable, but the underlying facts, apart 
        from their expression, are not.\footnote{Casebook p. 1008.}
        \item The case depends on whether INS's actions counted as unfair 
        competition.
        \item Since news has value in the context of the competition between 
        INS and AP, courts can consider to be a form of 
        ``quasi-property.''\footnote{Casebook p. 1009.} Misappropriation of 
        this form of property is actionable.
        \item Affirmed (held for AP).
    \end{enumerate}
    \item Justice Holmes, concurring:
    \begin{enumerate}
        \item This is the same harm as misrepresentation (i.e., passing off 
        your goods using another's name).\footnote{Casebook p. 1012--13.}
    \end{enumerate}
    \item Justice Brandeis, dissenting:
    \begin{enumerate}
        \item AP has no property right in its published stories. Failure to 
        give credit is not fraud.\footnote{Casebook p. 1014.}
        \item INS's conduct was unfair, but it's not the court's rule to 
        create a new, complex property right.
    \end{enumerate}
\end{enumerate}

\subsection{Protection by Contract}

\subsubsection{Shrinkwrap Licenses: \emph{ProCD, Inc. v. Zeidenberg}}
~\\\\
Software shrinkwrap license agreements are enforceable, even if the consumer 
can't read them until he has bought the product.

\begin{enumerate}
    \item ProCD published a CD compilation of 3,000 telephone directories. It 
    sold to commercial users at a higher price than individual consumers, and 
    used a shrinkwrap agreement (i.e., EULA) to enforce its price 
    discrimination scheme. Users did not have to agree to the license terms 
    before they bought the product or opened the box, but they did have to 
    agree before using the software.\footnote{Casebook p. 1021--1023.}
    \item Zeidenberg copied ProCD's products and put them online at a lower 
    price.
    \item Zeidenberg argued that only the terms on the outside of the box 
    counted as license terms.
    \item There are plenty of ``money now, terms later'' transactions---like 
    airline tickets, insurance, or concert tickets. Buyers can return the 
    product if they don't agree with the terms. The UCC supports this 
    approach.\footnote{Casebook p. 1024--26..}
    \item Held: the agreement was enforceable.
\end{enumerate}

\subsubsection{Browsewrap Agreements: \emph{Specht v. Netscape Communications 
Corp.}}

\begin{enumerate}
    \item When users downloaded Netscape's SmartDownload, they were not 
    required to agree to Netscape's terms. The terms were, however, printed on 
    the page---offscreen below the download button.
    \item In a class action, Netscape sought to enforce an arbitration clause 
    in the SmartDownload agreement. Netscape argued that the plaintiffs were 
    on ``inquiry notice'' of the terms.\footnote{Casebook p. 1033.}
    \item Held: a ``reasonably prudent offeree'' would not have been aware of 
    the SmartDownload terms.\footnote{Casebook p. 1034.} Downloading did not 
    count as assent. \emph{ProCD} is distinct because in that case, users of 
    had to agree to the terms each time they ran the application.
\end{enumerate}

\subsubsection{Copyright Preemption of Contract Law: \emph{ProCD II}}

\begin{enumerate}
    \item Does federal copyright law preempt the shrinkwrap agreement in 
    \emph{ProCD}?
    \item See casebook 1037 ff.
\end{enumerate}

\subsection{Idea Submissions}

\begin{enumerate}
    \item What happens when ideas are communicated without a formal contract? 
    Should state law protect ideas that are not trade secrets?
\end{enumerate}

\subsubsection{\emph{Nadel v. Play-by-Play Toys \& Novelties, Inc.}}

At least under New York law, breach of contract actions for stealing ideas are 
viable if the idea was novel \emph{to the buyer}, even if it was generally 
known.

\begin{enumerate}
    \item Nadel, a toy inventor, sent a prototype of a toy to Play-by-Play, 
    which released a very similar toy soon after. Nadel argued that 
    Play-by-Play used his idea without compensating him, claiming breach of 
    contract, a quasi-contract claim, and misappropriation. Play-by-Play 
    argued that it developed the idea independently.\footnote{Casebook p. 
    1045.}
    \item The district court granted Play-by-Play's motion for summary 
    judgment because the idea was not generally novel.
    \item On appeal, Nadel argued that for the breach of contract claim, he 
    only needed to prove that the idea was not novel \emph{to the buyer}, 
    rather than the industry generally.
    \item The New York standard (from \emph{Apfel}):\footnote{Casebook p. 1047 
    ff.}
    \begin{enumerate}
        \item Contract claim: idea need only have been novel \emph{to the 
        buyer}. Since it has value to the buyer, the idea counts as valuable 
        consideration.
        \item Misappropriation claim: originality (i.e., general novelty) is 
        needed.
    \end{enumerate}
    \item Misappropriation claim: remanded to determine whether the idea was 
    novel generally.
    \item Contract claim: there was a genuine issue of material fact as to 
    whether Nadel's idea was novel to Play-by-Play.
\end{enumerate}

\subsection{The Right of Publicity}

% TODO 1064-70, 1080-96

\subsection{Patent Preemption of State Laws}

% TODO 1109-18
