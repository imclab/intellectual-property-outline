\section{State Law and Federal Preemption}

\begin{enumerate}
    \item Six ways that state law protects IP (other than trade secret and 
    trademark):\footnote{Casebook p. 1005.}
    \begin{enumerate}
        \item Misappropriation.
        \item Contract.
        \item Idea submissions (implied contracts).
        \item Right of publicity.
        \item Trespass to chattel.
    \end{enumerate}
\end{enumerate}

\subsection{The Tort of Misappropriation}

\subsubsection{``Hot News'' and Quasi-Property: \emph{International News 
Service v. Associated Press}}

News stories can be protect as a form of ``quasi-property'' which can be 
protected against misappropriation by direct competitors.

\begin{enumerate}
    \item INS and AP competed to deliver ``hot news'' to newspapers across the 
    country. INS began copying AP's stories and sending them to newspapers 
    itself---for instance, by copying stores published on the East Coast and 
    sending them to West Coast newspapers.\footnote{Casebook p. 1006--07.}
    \item Can AP win an injunction?
    \item Justice Pitney:
    \begin{enumerate}
        \item INS argued that AP did not have a property right in its 
        published stories.
        \item News articles are copyrightable, but the underlying facts, apart 
        from their expression, are not.\footnote{Casebook p. 1008.}
        \item The case depends on whether INS's actions counted as unfair 
        competition.
        \item Since news has value in the context of the competition between 
        INS and AP, courts can consider to be a form of 
        ``quasi-property.''\footnote{Casebook p. 1009.} Misappropriation of 
        this form of property is actionable.
        \item Affirmed (held for AP).
    \end{enumerate}
    \item Justice Holmes, concurring:
    \begin{enumerate}
        \item This is the same harm as misrepresentation (i.e., passing off 
        your goods using another's name).\footnote{Casebook p. 1012--13.}
        \item INS ``impliedly denies to the plaintiff the credit of collecting 
        the facts and assumes that credit to the defendant.''
    \end{enumerate}
    \item Justice Brandeis, dissenting:
    \begin{enumerate}
        \item AP has no property right in its published stories. Failure to 
        give credit is not fraud.\footnote{Casebook p. 1014.}
        \item INS's conduct was unfair, but it's not the court's rule to 
        create a new, complex property right.
    \end{enumerate}
\end{enumerate}

\subsection{Protection by Contract}

\subsubsection{Shrinkwrap Licenses---``Deal Now, Terms Later'': \emph{ProCD, 
Inc. v. Zeidenberg}}
~\\\\
Software shrinkwrap license agreements are enforceable, even if the consumer 
can't read them until he has bought the product.

\begin{enumerate}
    \item ProCD published a CD compilation of 3,000 telephone directories. It 
    sold to commercial users at a higher price than individual consumers, and 
    used a shrinkwrap agreement (i.e., EULA) to enforce its price 
    discrimination scheme. Users did not have to agree to the license terms 
    before they bought the product or opened the box, but they did have to 
    agree before using the software.\footnote{Casebook p. 1021--1023.}
    \item Zeidenberg copied ProCD's products and put them online at a lower 
    price.
    \item Zeidenberg argued that only the terms on the outside of the box 
    counted as license terms.
    \item There are plenty of ``money now, terms later'' transactions---like 
    airline tickets, insurance, or concert tickets. Buyers can return the 
    product if they don't agree with the terms. The UCC supports this 
    approach.\footnote{Casebook p. 1024--26..}
    \item Held: the agreement was enforceable.
\end{enumerate}

\subsubsection{Browsewrap Agreements: \emph{Specht v. Netscape Communications 
Corp.}}

``We rule against Netscape and in favor of the users of its software because 
the users \textbf{would not have seen the terms Netscape exacted without 
scrolling down} their computer screens, and there was no reason for them to do 
so.'' A statement of agreement is ``essential to the formation of a 
contract.''

\begin{enumerate}
    \item When users downloaded Netscape's SmartDownload, they were not 
    required to agree to Netscape's terms. The terms were, however, printed on 
    the page---offscreen below the download button.
    \item In a class action, Netscape sought to enforce an arbitration clause 
    in the SmartDownload agreement. Netscape argued that the plaintiffs were 
    on ``inquiry notice'' of the terms.\footnote{Casebook p. 1033.}
    \item Held: a ``reasonably prudent offeree'' would not have been aware of 
    the SmartDownload terms.\footnote{Casebook p. 1034.} Downloading did not 
    count as assent. \emph{ProCD} is distinct because in that case, users of 
    had to agree to the terms each time they ran the application.
\end{enumerate}

\subsection{Copyright Preemption}

\begin{enumerate}
    \item Two sources:
    \begin{enumerate}
        \item Supremacy Clause.
        \item 17 U.S.C. \S\ 301: ``all legal or equitable rights that are 
        equivalent to any of the exclusive rights within the general scope of 
        copyright~.~.~.~and some within the subject matter of 
        copyright~.~.~.~are governed exclusively by this title.''
    \end{enumerate}
    \item Two-part preemption test---a claim is preempted if it (1) comes 
    within the \textbf{subject matter} of copyright and (2) the rights granted 
    under the state law are \textbf{equivalent} to any of the exclusive rights 
    within the general scope of copyright as defined in \S\ 106.
\end{enumerate}

\subsubsection{Copyright Preemption of Contract Law: \emph{ProCD II}}

\begin{enumerate}
    \item Does federal copyright law preempt the shrinkwrap agreement in 
    \emph{ProCD}?
    \item The contract here involved only two parties. It was not ``good 
    against the world.''
    \item ``Contracts do not create `exclusive rights.' Someone who found a 
    copy of SelectPhone (trademark) on the street would not be affected by the 
    shrinkwrap license---though the federal copyright laws of their own force 
    would limit the finder's ability to copy or transmit the application 
    program.''\footnote{Casebook p. 994.}
    \item But Easterbrook would not enforce \emph{all} contracts in the IP 
    field.\footnote{Casebook p. 995.} Two types of contracts can raise 
    preemption questions: (1) those that expand affirmative exclusive rights 
    (e.g., extending the duration of patent protection) or (2) narrow or 
    exclude statutory or common-law exceptions (e.g., fair use).
\end{enumerate}

\subsection{Idea Submissions}

\begin{enumerate}
    \item What happens when ideas are communicated without a formal contract?  
    Should state law protect ideas that are not trade secrets?
\end{enumerate}

\subsubsection{General Novelty vs. Novelty to the Buyer: \emph{Nadel v.  
Play-by-Play Toys \& Novelties, Inc.}}

At least under New York law, breach of contract actions for stealing ideas are 
viable if the idea was novel \emph{to the buyer}, even if it was generally 
known.

\begin{enumerate}
    \item Nadel, a toy inventor, sent a prototype of a toy to Play-by-Play, 
    which released a very similar toy soon after. Nadel argued that 
    Play-by-Play used his idea without compensating him, claiming breach of 
    contract, a quasi-contract claim, and misappropriation. Play-by-Play 
    argued that it developed the idea independently.\footnote{Casebook p.  
    1045.}
    \item The district court granted Play-by-Play's motion for summary 
    judgment because the idea was not generally novel.
    \item On appeal, Nadel argued that for the breach of contract claim, he 
    only needed to prove that the idea was not novel \emph{to the buyer}, 
    rather than the industry generally.
    \item The New York standard (from \emph{Apfel}):\footnote{Casebook p. 1047 
    ff.}
    \begin{enumerate}
        \item Contract claim: idea need only have been novel \emph{to the 
        buyer}. Since it has value to the buyer, the idea counts as valuable 
        consideration.
        \item Misappropriation claim: originality (i.e., general novelty) is 
        needed.
    \end{enumerate}
    \item Misappropriation claim: remanded to determine whether the idea was 
    novel generally.
    \item Contract claim: there was a genuine issue of material fact as to 
    whether Nadel's idea was novel to Play-by-Play.
\end{enumerate}

\subsubsection{Avoid Blurting: \emph{Desny v. Wilder}}

To be enforceable, there has to be a contract \emph{before} disclosure of the 
idea. Don't blurt out your valuable ideas.

\begin{enumerate}
    \item Desny gave Wilder a synopsis of his screenplay without a 
    compensation agreement. Wilder made a movie based on the idea without 
    paying Desny.
    \item The appellate court granted summary judgment for Wilder.
    \item Defendants argued that Desny disclosed the idea before they 
    expressed willingness to pay (or not).\footnote{Casebook p. 1056.}
    \item Brandeis in \emph{INS v. AP}: value alone does not turn ideas into 
    property.
    \item But ideas can be protected by contract---e.g., advice that doctors 
    and lawyers give.
    \item Ideas can constitute valuable consideration and be bargained for, 
    but after disclosure to a second party, that party is free to use them as 
    his own.\footnote{Casebook p. 1058.}
    \item To be enforceable, there has to be a contract \emph{before} 
    disclosure of the idea. Desny blurted out his idea, so he is not entitled 
    to contractual protections.
    \item Justice Carter, concurring;
    \begin{enumerate}
        \item Writers as sellers are in a mucb inferior bargaining 
        position.\footnote{Casebook p. 1059.}
        \item Buyers wouldn't buy ideas if the terms are ``I won't tell you 
        what my idea is until you promise to pay me for 
        it.''\footnote{Casebook p. 1059.}
        \item Department stores don't have to say explicitly that their goods 
        are for sale. Similarly, it should have been implied that Desny's idea 
        was for sale.
    \end{enumerate}
\end{enumerate}

\subsection{The Right of Publicity}

\begin{enumerate}
    \item One of the four privacy torts: appropriating the plaintiff's 
    identity for the defendant's benefit.
    \item Two theories of protection: privacy and property.
    \item Should the right to publicity be assignable or descendable?
    \begin{enumerate}
        \item If privacy is the theory, maybe not.
        \item If property, maybe so.
    \end{enumerate}
    \item 15 states: common law; 16 states: statute; California: both. 
    Publicity nights are fully assignable and descendable in California, 
    though the right is limited to life plus 70 years.
\end{enumerate}

\subsubsection{\emph{Midler v. Ford Motor Co.}}

Tort law protect singers against imitation of their voices.

\begin{enumerate}
    \item Ford couldn't get Bette Midler for one of its commercials, so it 
    hired one of her backup singers to impersonate her.\footnote{Casebook p. 
    1067.}
    \item No copyright issue here because Midler sued to protect her 
    \emph{likeness}, not her \emph{song}.
    \item No trademark issue here because there was no secondary meaning in 
    her voice.
    \item No unfair competition issue here because Midler was not in 
    competition with Ford.
    \item ``We hold that when a distinctive voice of a professional singer is 
    widely known and is deliberately imitated to sell a product, the sellers 
    have appropriated what is not theirs and have committed a tort in 
    California~.~.~.~.''
\end{enumerate}

\subsubsection{First Amendment and the Transformative Use Test: \emph{Comedy 
III Productions, Inc. v. Gary Saderup, Inc.}}

Did the defendant use the celebrity image as one input among many, adding more 
expressive elements? If so, the use is transformative (and unlikely to harm 
the market because it is not a good substitute for the original).

\begin{enumerate}
    \item Saderup sold lithographs and prints based on charcoal drawings he 
    made of the Three Stooges.
    \item Advertising is a form of commercial speech, entitled to a lower 
    level of protection than other forms of speech, including expressive 
    works.
    \item Held: there was no significant transformation here.
\end{enumerate}

\subsection{Patent Preemption}

\subsubsection{Employee Mobility: \emph{Kewanee} Oil Co. v. Bicron Corp.}

Federal IP laws do not preempt state trade secret law.

\begin{enumerate}
    \item Trade secret: process that created crystals useful in detecting 
    radiation.\footnote{Casebook p. 1109 ff.}
    \item Is state trade secret law compatible with the federal statutory IP 
    scheme?
    \item Easy cases: unpatentable trade secrets, e.g., those that are obvious 
    or that fall outside the \S\ 101 categories.
    \item Harder cases: those that the inventor could have patented but kept 
    as trade secrets. 
    \item Held: federal IP laws do not preempt state trade secret law.
    \item Justice Douglas, dissenting:
    \begin{enumerate}
        \item ``~.~.~.~every article not covered by a valid patent is in the 
        public domain.''
    \end{enumerate}
\end{enumerate}

\subsubsection{\emph{Bonito Boats, Inc. v. Thunder Craft Boats, Inc.}}

\begin{enumerate}
    \item Bonito developed a fiberglass boat hull. It did not patent the 
    design. The Florida legislature then enacted a statute preventing the use 
    of a direct molding process to duplicate vessel hulls.
    \item Bonito alleged that Thunder Craft duplicated its hull through direct 
    molding.
    \item Held: the Florida statute doesn't prevent unfair competition because 
    it doesn't address confusion about the source of goods. The statute was 
    invalid because it ``impedes the public use of the otherwise unprotected 
    design and utilitarian ideas'' in unpatented boat hulls.\footnote{Casebook 
    p. 1115.}
    \item In response, Congress added sui generis copyright protections for 
    boat hull designs.
\end{enumerate}
