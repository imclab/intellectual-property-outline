\section{Copyright}

\subsection{Introduction}

\subsubsection{17 U.S.C. \S\ 101: Definitions}

\begin{enumerate}
    \item See supplement p. 248.
    \item ``Created'': fixed for the first time. Work $\neq$ copy.
\end{enumerate}

\subsubsection{History}

\begin{enumerate}
    \item Venice in the fifteenth century recognized exclusive rights in the 
    printing of particular books (as opposed to the technology of creating the 
    books).\footnote{Casebook p. 430.}
    \item \textbf{Statute of Anne} (1710): authors gain exclusive rights for 
    14 years, renewable for another 14 years, subject to registration, notice, 
    and deposit. The government could also set maximum 
    prices.\footnote{Casebook p. 431.}
    \item The continental approach emphasized moral rights, as opposed to the 
    English system of property rights.
    \item The original Copyright Act (1790) mirrored the Statute of Anne. By 
    the end of the nineteenth century, it had been amended to include a range 
    of new technologies (e.g., photography).\footnote{Casebook p. 432.}
    \item \textbf{1909 Act}: extended protection to ``all writings''; extended 
    term to 28 years plus a 28 year renewal.
    \item \textbf{1976 Act}: extended protection to anything ``fixed in a 
    tangible medium of expression''; extended term to life of the author plus 
    50 years; loosened notice and registration requirements; established 
    compulsory licensing regimes; and codified exceptions, including fair use. 
    The 1998 Act extended the term to life of the author plus seventy 
    years.\footnote{Casebook p. 433.}
    \item 1989: the United States joined the Berne convention, scaling back 
    formalities and restoring copyright for foreign works still under 
    protection in their source countries.
    \item \textbf{DMCA} (1998): anticircumvention provisions and liability 
    protections for service providers for infringing acts of their 
    subscribers.\footnote{Casebook pp. 433--34.}
    \item The Patent Act is industry-neutral. The Copyright Act: not so.
\end{enumerate}

\subsubsection{Overview}

\paragraph{Elements of a Protectable Copyright}

\begin{enumerate}
    \item \textbf{Copyrightable subject matter}: any literary or artistic 
    expression (but not the idea itself).
    \item \textbf{Threshold for protection}: must (1) have a ``modicum of 
    \textbf{originality}'' and (2) be \textbf{fixed} in a ``tangible medium of 
    expression.''
    \item \textbf{Formalities}: notice is required for works published before 
    1989. Registration is required to sue for infringement (but not strictly 
    necessary to create copyright). Deposit is required for registration.
    \item \textbf{Authorship and ownership}: only the creator, transferee, or 
    employer can bring an infringement suit (whereas in patent law, the 
    inventor himself is the author, and then he assigns ownership to a 
    corporate entity).
    \item \textbf{Duration of copyright}: life plus 70 years; or 95 years from 
    the publication of anonymous, pseudonymous, or for-hire works; or 120 
    years from the year of creation, whichever comes first.
    \item The Copyright Office does not assess validity (other than requiring 
    a modicum of creativity) or prior art.
    \item A copyright is protectable at the moment of 
    creation.\footnote{Casebook p. 434.}
\end{enumerate}

\paragraph{Ownership Rights}

\begin{enumerate}
    \item Long but narrow protection.
    \item \textbf{Reproduction}: owner has the exclusive right to make copies. 
    She can sue for ``material'' and ``substantial'' copying.
    \item \textbf{Derivative works}: original owner has the exclusive right to 
    prepare derivative works (e.g., translations or a film based on a book).
    \item \textbf{Distribution}: the owner controls sale and distribution, but 
    only for the first sale.
    \item \textbf{Performance and display}: owner has the right to control.
    \item \textbf{Anticircumvention}: prohibitions on bypassing technological 
    protections (DMCA).
    \item \textbf{Moral rights}: visual artists have an attribution right and 
    a right to prevent ``intentional distortion, mutilation, or other 
    modification~.~.~.~''\footnote{Casebook p. 435.}
    \item Limitations: fair use, compulsory licensing.
\end{enumerate}

\paragraph{Copyright vs. Patent}
    
\begin{enumerate}
    \item Copyright holders can prevent copying and certain uses (e.g., public 
    performance), but they cannot prevent others from making, using, or 
    selling their creations.
    \item ``The independent development of a similar or even identical work is 
    perfectly legal.''\footnote{Merges, 
    \url{http://www.law.berkeley.edu/files/bclt_2-6-12_Intro_to_Copyright.ppt}} 
    (Not so in patent law.)
    \item Infringement requires (1) \textbf{proof of copying} and 
    \textbf{substantial similarity}.
    \item Limiting doctrines: idea/expression dichotomy; useful article 
    doctrine; government work (\S\ 105);  fair use.
\end{enumerate}

\subsubsection{Philosophy}

\begin{enumerate}
    \item Personhood, natural law.
    \item Predominant in the United States: utilitarian. The primary goal is 
    to enhance the public interest. The secondary goal is to reward 
    authors.\footnote{Casebook p. 436.}
    \item There is an increasing recognition of moral rights in the United 
    States.\footnote{Casebook p. 437.}
\end{enumerate}

\subsection{Requirements}

\subsubsection{Originality}

\paragraph{17 U.S.C. \S\ 102: Subject Matter of Copyright: In General}

\begin{enumerate}
    \item Original.
    \item ``~.~.~.~fixed in any tangible medium of expression~.~.~.~''
    \item A work gains copyright protection if it was independently created, 
    even if it's identical---see Learned Hand on Keats.\footnote{Casebook p. 
    439.} The originality requirement is very low; exceptions are slogans, 
    familiar symbols, etc.
    \item Must have a \textbf{modicum of creativity}.
\end{enumerate}

\paragraph{No Copyright for Facts: \emph{Feist Publications v. Rural Telephone 
Services}}
~\\\\
Copyright requires originality, because ``writings'' in the constitutional 
language implies it. Facts and unoriginal creations cannot be copyrighted. 
Authors discover facts; they don't originate them. 

Things that \emph{can} be copyrighted in a compilation: the selection, 
coordination, and arrangement.

\begin{enumerate}
    \item Feist copied records from Rural's phonebook for its own phonebook. 
    Rural sued for infringement.
    \item Justice O'Connor:
    \begin{enumerate}
        \item Facts are not copyrightable. But compilations of facts generally 
        are.\footnote{Casebook p. 441.}
        \item Originality is ``the touchstone of copyright 
        protection'' because it is constitutionally and statutorily 
        mandated.\footnote{Casebook p. 443.} Originality requires (1) 
        independent creation and (2) a minimal degree of creativity.  
        Facts are not original. Therefore, facts are not copyrightable.
        \item Copyright of an entire work does not imply copyright of each 
        element.\footnote{Casebook p. 443.}
        \item ``Not all copying, however, is copyright 
        infringement.''\footnote{Casebook p. 443.} Infringement requires 
        copying of elements that are original. 
        \item Rural's alphabetical arrangement ``is not only unoriginal, it is 
        practically inevitable.''\footnote{Casebook p. 445.} Since Rural's 
        arrangement was unoriginal, it was not copyrightable. Therefore, 
        Feist's copying was non-infringing.
    \end{enumerate}
\end{enumerate}

\subsubsection{Fixation}

\paragraph{H.R. Rep. No. 94-1476: On the Copyright Act of 1976}

\begin{enumerate}
    \item Any medium satisfies the fixation requirement.\footnote{Casebook p. 
    449.} Unfixed works may still be eligible for protection under State 
    common or statutory law, but federal protection requires 
    fixation.\footnote{Casebook p. 450.}
    \item Broadcasts are covered by a provision that allows protections for 
    works that are simultaneously recorded and transmitted.
    \item ``The two essential elements---\textbf{original work} and 
    \textbf{tangible object}---must merge through fixation in order to produce 
    subject matter copyrightable under the statute.''\footnote{Casebook p. 
    451.}
    \item Fixation is (1) a requirement for protection and (2) plays a role in 
    determining whether a defendant has infringed a copyright (because copies 
    are material objects in which a work is ``fixed'').\footnote{Casebook p. 
    551.}
    \item Do bootlegs infringe? They're based on non-fixed performances. The 
    Second Circuit upheld the anti-bootlegging criminal provisions under the 
    Commerce Clause because they do not create additional rights; however, the 
    implication is that the civil provisions are 
    unconstitutional.\footnote{Casebook pp. 451--52.}
    \item Why require fixation?\footnote{Casebook p. 452.}
    \begin{enumerate}
        \item If copyright is meant to protect communication, then it should 
        not apply to expressions that do not actually communicate.
        \item It's a practical requirement for litigation (cf. the statute of 
        frauds in contract law).
    \end{enumerate}
    \item Data residing in a buffer for less than 1.2 seconds is ``merely of 
    \textbf{transitory duration}'' and thus not a copy.
\end{enumerate}

\subsubsection{Formalities}

\paragraph{Three Eras of Formalities}

\begin{enumerate}
    \item 1909 Act.
    \item 1976 Act (effective Jan. 1, 1978).
    \item 1989 Berne Implementation Act (effective March 1, 1989).
\end{enumerate}

\paragraph{Notice}

\begin{enumerate}
    \item Historically: required; today: encouraged.
    \item 1909 Act: failure to follow precise notice requirements resulted in 
    forfeiture. Notice \emph{required}.
    \item 1976 Act/Pre-Berne: copyright begins upon creation, not 
    publication. Notice was still required, but the requirements were 
    significantly looser.
    \item Post-Berne: completely eliminated the notice requirement, but 
    encouraged voluntary notice (e.g., by allowing the innocent infringement 
    defense if the work lacked proper notice).\footnote{Casebook p. 455.} 
    Notice \emph{encouraged}.
\end{enumerate}

\paragraph{Publication}

\begin{enumerate}
    \item 1909 Act: publication triggered copyright. Unpublished works could 
    be protected under state common law, or ``constructively'' published by 
    registration with the Copyright Office.
    \begin{enumerate}
        \item \emph{Divestive} publication: publication without notice; 
        results in forfeiture of common law copyright protection. 
        \item \emph{Investive} publication: publication \emph{with} 
        notice; results in forfeiture of federal statutory protection if 
        notice is inadequate.
    \end{enumerate}
    \item 1976 Act/Pre-Berne: creation triggered copyright, but still 
    determined when notice was required.\footnote{Casebook pp. 456--57.}
    \item Post-Berne: after March 1, 1989, publication no longer determines 
    validity, though it still has relevance in several areas---see casebook p. 
    457.
\end{enumerate}

\paragraph{Analyzing Publication and Notice Problems}

\begin{enumerate}
    \item For works fixed or published \textbf{\emph{before} 1/1/78}, ask: was 
    there strict compliance with the 1909 notice rules?
    \begin{enumerate}
        \item Yes: copyright applies.
        \item No: work is in the public domain.
    \end{enumerate}
    \item For works fixed or published \textbf{\emph{after}} 1/1/78, ask: was 
    the work disclosed to the public before 3/1/89?
    \begin{enumerate}
        \item No (i.e., published after 3/1/89): copyright applies.
        \item Yes: ask, was notice included?
        \begin{enumerate}
            \item Yes: copyright applies.
            \item No: ask, was publication limited?
            \begin{enumerate}
                \item Limited publication: copyright applies
                \item General publication: in public domain if there was no 
                notice. 
            \end{enumerate}
        \end{enumerate}
    \end{enumerate}
\end{enumerate}

\paragraph{Registration}

\begin{enumerate}
    \item Has always been voluntary.
    \item 1909 Act: term was 28 years, plus a 28 year renewal, but only if the 
    owner registered.
    \item 1976 Act/Pre-Berne: abolished renewal, but created new incentives 
    for registration.\footnote{Casebook p. 458.}
    \item Post-Berne: no registration requirement for foreign works, but 
    Congress preserved the registration requirement for domestic works for 
    bringing infringement suits.
    \begin{enumerate}
        \item You can register after infringement, but you only recover 
        damages for the period after registration, so there's an incentive to 
        register early.
    \end{enumerate}
\end{enumerate}

\paragraph{Deposit}

\begin{enumerate}
    \item After 1976, still mandatory (as part of registration), but failure 
    results only in a fine, not invalidity or forfeiture of the right to bring 
    suit.\footnote{Casebook p.  459.}
\end{enumerate}

\paragraph{Restoration of Foreign Copyrighted Works}

\begin{enumerate}
    \item Post-Berne, many foreign works that had entered the public domain in 
    the United States (e.g., J.R.R. Tolkien's works) were restored to 
    copyright. The Supreme Court affirmed in \emph{Golan v. Holder}, holding 
    that the IP clause allows Congress to ``induc[e] the dissemination of 
    works~.~.~.~to promote science.''\footnote{Casebook p. 460.}
\end{enumerate}

\subsection{Copyrightable Subject Matter}

\subsubsection{Limitations on Copyrightability: Distinguishing Function and 
Expression}

\paragraph{The Idea-Expression Dichotomy}

\paragraph{17 U.S.C. \S\ 102(b)}

\begin{enumerate}
    \item Copyright protection does not extend to ``any idea, procedure, 
    process, system~.~.~.~.''\footnote{Casebook p. 461.}
\end{enumerate}

\paragraph{Use vs. Explanation, Idea vs. Expression: \emph{Baker v. Seden}}
~\\\\
A useful idea cannot be copyrighted, but an explanation of how to use it can 
be. ``There is a clear distinction between a book, as such, and the art which 
it is intended to illustrate.''\footnote{Casebook p. 462.}

\begin{enumerate}
    \item Seden published a book about a ledger system, which included his 
    specific ledger design. He sued Baker for infringement. Baker argued that 
    the book was ``not a lawful subject of copyright.''\footnote{Casebook p. 
    461.}
    \item Justice Bradley:
    \begin{enumerate}
        \item Seden did not have a valid copyright in the ledger system 
        itself.\footnote{Casebook p. 462.}
        \item The book itself might be copyrightable, but the underlying 
        system is not.
        \item ``To give the author of the book an exclusive property in the 
        art described therein, when no examination of its novelty has ever 
        been officially made, would be a surprise and a fraud upon the public. 
        That is the province of letters-patent, not of copyright. The claim to 
        an invention or discovery of art or manufacture must be subjected to 
        the examination of the Patent Office before an exclusive right therein 
        can be obtained; it can only be secured from a patent from the 
        government.''\footnote{Casebook p. 462.}
        \item The use cannot be copyrighted, but the explanation can 
        be.\footnote{Casebook p. 463.} Any other holding would allow copyright 
        to swallow patent law.
    \end{enumerate}
    \item At a certain level of abstraction, descriptions of how to use an 
    idea are no longer copyrightable (Learned Hand).\footnote{Casebook pp. 
    465--66.}
    \item Another example of the idea-vs.-expression principle: Edward Tufte 
    diagrams of public domain data.
\end{enumerate}

\paragraph{Computer Menus and Subject Matter: \emph{Lotus Development Corp. v. 
Borland International}}
~\\\\
Computer menus are ``methods of operation'' (under 17 U.S.C. \S\ 102(b)) and 
therefore not copyrightable.

\begin{enumerate}
    \item Borland copied the menu configuration of Lotus 1-2-3.
    \item The district court found infringement. The question on appeal was 
    whether computer menus are copyrightable subject matter.
    \item This case is distinct from \emph{Baker v. Selden} because that case 
    involved the design of the spreadsheet grid, while this case involves the 
    commands used for interaction.\footnote{Casebook p. 471.}
    \item Held: the menu commands are a \textbf{``method of operation''} 
    because they are necessary for the program to work. Thus, they are not 
    copyrightable subject matter under \S\ 102(b). Lotus's underlying code is 
    copyrightable, but the menu structure is not.\footnote{Casebook p. 472.}
    \item The menu commands are analogous to the buttons on a VCR.
    \item Judge Boudin, concurring:
    \begin{enumerate}
        \item Users are likely to be locked in to Lotus's menu structure (like 
        a QWERTY keyboard). If Borland comes along with a better product, 
        there are good reasons for freeing users to make the switch.
        \item There are two ways Borland might prevail:\footnote{Casebook p. 
        477.}
        \begin{enumerate}
            \item The menu is a ``method of operation.''
            \item Borland's use is privileged under something like fair use, 
            because it enables users to exploit their previous efforts in 
            writing macros.
        \end{enumerate}
        \item The second approach would introduce administrative problems, so 
        the majority's approach is as good as it gets for 
        now.\footnote{Casebook p. 478.}
    \end{enumerate}
\end{enumerate}

\paragraph{Merger Doctrine}

\begin{enumerate}
    \item When ``there is only one feasible way of expressing an idea, so that 
    if the expression were copyrightable it would mean that the idea was 
    copyrightable,'' the \textbf{expression is not protected}. \emph{Bucklew 
    v. Hawkins} (7th Cir. 2003).
    \item E.g., mathematical diagrams are not protected if they are the only 
    way of expressing an idea. The specific visual arrangement might be 
    protected, but the formula itself would not be.
\end{enumerate}

\paragraph{The Useful Article Doctrine}

\begin{enumerate}
    \item Under \S\ 101, a ``useful article'' is a picture, graphic, or 
    sculpture only if the PGS features can be identified separately from the 
    ``utilitarian aspects.''\footnote{Casebook p. 482--83.} A useful article 
    has a utilitarian function that is separate from its appearance.
    \item The protected aspects of a PGS work are the parts that include 
    ``artistic craftsmanship'' but not the ``mechanical or utilitarian 
    aspects.'' \S\ 101.
    \item The only protected parts of a PGS work are the \textbf{``features 
    that can be identified separately from, and are capable of existing 
    independently of, the utilitarian aspects of the article.''} \S\ 101.
    \item At a certain level of abstraction, aspects of a work are no longer 
    protected. E.g., \emph{Romeo and Juliet} might be protected, but the 
    general ``boy-meets-girl'' plot is not.
    \item Problem areas: applied art vs. industrial design.
    \item Spectrum:
    \begin{enumerate}
        \item \textbf{Not protectable}: industrial design.
        \item \textbf{Protectable}: art, e.g., a lamp sculpture.
        \item The tough cases are in the middle.
    \end{enumerate}
\end{enumerate}

\paragraph{H.R. Rep. No. 94-1476: On the 1976 Act}

\begin{enumerate}
    \item Congress attempted to distinguish between ``copyrightable works of 
    applied art and uncopyrighted works of industrial 
    design.''\footnote{Casebook p. 483.}
\end{enumerate}

\paragraph{Industrial Design: \emph{Brandir International, Inc. v. Cascade 
Pacific Lumber Co.}}
~\\\\
Here, the bike rack's design was determined entirely by its utilitarian 
purpose, so there was no separate protectable expression. ``Form and function 
are inextricably intertwined in the rack~.~.~.~.''\footnote{Casebook p. 482.}

\begin{enumerate}
    \item Brandir developed bicycle racks based on wire sculptures. The 
    Copyright Office and the district court denied copyright.
    \item When are a work's design and utility ``conceptually separate''? 
    Courts have struggled to come up with a good test.\footnote{Casebook p. 
    484.}
    \item The court here rejected Judge Newman's test from \emph{Carol 
    Barnhart}, the ``temporal displacement'' test, which finds the two to be 
    separate if they ``stimulate in the mind'' a conceptual 
    separation.\footnote{Casebook pp. 484--85.}
    \item Instead, it adopted the Denicola test, which finds conceptual 
    separation if the artistic judgment was separate from the functional 
    influence, but not if the design elements reflect both design and 
    functional considerations.\footnote{Casebook p. 487.}
    \item The bike rack here was based on a wire sculpture of a bike. It was 
    enlarged to hold actual bikes, and slightly modified so that it could hold 
    bikes more efficiently.\footnote{Casebook pp. 486--87.}
    \item The court held that the rack was not copyrightable under the 
    Denicola test because its design was ``in its final form essentially a 
    product of industrial design.'' There was no separate, independent 
    artistic aspect.
    \item Judge Winter, dissenting: a better test would be whether the article 
    ``causes an ordinary reasonable observer to perceive an aesthetic concept 
    not related to the article's use.''\footnote{Casebook p. 489.} The rack 
    passes that test---for instance, it could be successfully displayed as a 
    sculpture with no utilitarian purpose---and so it should be copyrightable.
\end{enumerate}

\paragraph{Government Works}

\begin{enumerate}
    \item Any ``law, which, binding every citizen, is free for publication to 
    all, whether it is a declaration of unwritten law, or an interpretation of 
    a constitution or statute.''\footnote{Casebook p. 493.}
    \item \emph{Veeck} centered on a dispute over whether model laws adopted 
    by municipalities must be released into the public 
    domain.\footnote{Casebook pp. 494--96.}
\end{enumerate}

\subsubsection{The Domain and Scope of Copyright Protection: Types of Works of 
Authorship}

\begin{enumerate}
    \item \S\ 101 describes eight categories of protected works, but the list 
    is ``illustrative and not limitative.''\footnote{Casebook p. 498.}
\end{enumerate}

\paragraph{Literary Works}

\begin{enumerate}
    \item Courts will not judge artistic merits.
    \item ``Short phrases'' are not protected.
    \item Protection extends to non-literal elements (e.g., structure).
    \item Courts have struggled with whether to protect fictional 
    characters.\footnote{Casebook p. 499.}
    \item Computer software is considered a ``literary work.''
\end{enumerate}

\paragraph{Pictorial, Graphic, and Sculptural Works}

\begin{enumerate}
    \item Most significant limitation: the utilitarian function 
    exception---see \emph{Brandir} above.
\end{enumerate}

\paragraph{Architectural Works}

\begin{enumerate}
    \item Protection applies to overall form and elements in the design, but 
    does not include ``individual standard features.''\footnote{Casebook p. 
    500.} Original, non-functional design elements are protected.
    \item Protection does not extend to pictorial representations of the 
    building. It also does not prevent others from modifying or destroying 
    it.\footnote{Casebook p. 501.}
\end{enumerate}

\paragraph{Musical Works and Sound Recordings}

\begin{enumerate}
    \item Different rules apply depending on whether the work is a 
    \textbf{musical work} or a \textbf{sound recording}. (This category only 
    includes musical works; see below for sound recordings.)
    \item Musical works (sheet music, lyrics, arrangements): fully protected, 
    including performance rights.
    \item Sound recordings: no traditional performance rights (although they 
    now have a digital performance right). For radio broadcasts, the owner of 
    the copyright of the composition gets a royalty, but the performer does 
    not.\footnote{Casebook p. 502.}
\end{enumerate}

\paragraph{Dramatic, Pantomime, and Choreographic Works}

\begin{enumerate}
    \item Protection extends to written or fixed instructions. 
    \item Performance and display rights can vary depending on whether the 
    work is classified as dramatic or nondramatic.\footnote{Casebook p. 503.}
    \item Short dance steps are not protected in the same way that ``short 
    phrases'' are not protected as literary works.
\end{enumerate}

\paragraph{Motion Pictures and Other Audiovisual Works}

\begin{enumerate}
    \item Soundtracks are integral parts.\footnote{Casebook p. 504.}
\end{enumerate}

\paragraph{Semiconductors and Vessel Hulls}

\begin{enumerate}
    \item Both have received sui generis protections through separate acts of 
    Congress.\footnote{Casebook p. 504.}
\end{enumerate}

\paragraph{Derivative Works and Compliations}

\begin{enumerate}
    \item Derivative works: translations, film adaptations, etc. The original 
    copyright owner controls the rights to derivative works.
    \item Compilations: anthologies, encyclopedias, etc. Must involve ``some 
    minimal degree of creativity'' (\emph{Feist}), but just how much is a 
    contentious issue.\footnote{Casebook pp. 505--06.}
\end{enumerate}

\subsection{Ownership and Duration}

\subsubsection{Initial Ownership of Copyrights}

\begin{enumerate}
    \item 17 U.S.C. \S\ 201(a): copyright vests initially in the owner or 
    owners.
    \item 201(b): for works for hire, the employer is the author.
    \item Cf. three levels of ownership in patents:
    \begin{enumerate}
        \item Hired to invent: employer owns.
        \item Related to employer's business, made with employer's resources: 
        employee owns, but the employer may have ``shop rights.''
        \item Unrelated to the employer's business, made with the employee's 
        resources: employee owns.
    \end{enumerate}
\end{enumerate}

\paragraph{Scope of Employment---Employee vs. Independent Contractor: 
\emph{CCNV v. Reid}}

\begin{enumerate}
    \item Work for hire: (1) made by an employee or (2) specially commissioned 
    in one of nine enumerated categories---see \S\ 101 (``work made for 
    hire''), supp. p. 253.
    \item Four tests:
    \begin{enumerate}
        \item Hiring party retains right to control.
        \item Hiring party wields actual control.
        \item Common law agency meaning.
        \item Only formal, salaried employees.
    \end{enumerate}
    \item Supreme Court: ``Congress intended to describe the conventional 
    master-servant relationship as understood by the common-law agency 
    doctrine~.~.~.~''\footnote{Casebook p. 511--12.} (i.e., it rejected the 
    first two approaches in the list above). For applying the law of agency, 
    see casebook p. 512 top.
    \item Held: Reid was an independent contractor, not an employee.
\end{enumerate}

\paragraph{Joint Works: \emph{Aalmuhammed v. Lee}}

\begin{enumerate}
    \item Why was this not a work for hire? \S\ 101: work for hire agreements 
    require a written agreement.\footnote{Supp. p. 254.}
    \item ``Joint work'': two or more authors intend to merge their 
    contributions into inseparable or interdependent parts of a unitary whole.
    \item Basic elements:
    \begin{enumerate}
        \item A copyrightable work.
        \item Two or more authors.
        \item Intent to merge.
    \end{enumerate}
    \item Ninth Circuit, here: making a valuable and copyrightable 
    contribution is not the same thing as authorship. Authorship requires a 
    degree of control. Aalmuhammed lacked that degree of control.
    \item Scholars (Nimmer and Goldstein) disagree about whether the 
    individual contributions must be independently copyrightable.
    \item Result of joint authorship:
    \begin{enumerate}
        \item Undivided interest in the whole.
        \item Duty of accounting to co-authors.
        \item Each can license non-exclusively.
        \item All must agree to license exclusively or assign.
        \item Duration measured by last life.
    \end{enumerate}
\end{enumerate}

\paragraph{Contribution to Collective Works: 17 U.S.C. \S\ 201(c)}

\begin{enumerate}
    \item Copyright in a contribution is distinct from copyright in the entire 
    collection.
    \item Owner of the collective work only acquires rights to include the 
    contribution in that work and later editions.
    \item Open source software is a challenge. Are programs jointly authored? 
    Is each new modification a derivative work?
\end{enumerate}

\paragraph{Analyzing IP Ownership Problems}

\begin{enumerate}
    \item What does the statute say? Who is the initial or default owner?
    \item Did the parties enter into an employment or other 
    \emph{pre-creation} agreement?
    \item Has there been an assignment or license \emph{after the work was 
    created}?
\end{enumerate}

\subsubsection{Duration and Renewal}

\begin{enumerate}
    \item Current duration: life of the author plus 70 years.
    \item To determine duration for works published earlier, see the chart on 
    pp. 527--29.
    \item Renewal was required under the 1909 Act, but is \textbf{no longer 
    required}.
    \item Should we bring back renewal to address the orphan works problem?
    \item Renewal lives on in the ``termination of transfer'' provisions. See 
    below.
\end{enumerate}

\subsubsection{Division, Transfer, and Reclaiming of Copyrights}

\paragraph{Division and Transfer Generally}

\begin{enumerate}
    \item Copyright is distinct from the property rights in a material 
    object. You can own a book without owning the rights to its copyrighted 
    contents.
    \item Similarly, the author of a letter retains her copyright interests 
    even though she mails the letter to the addressee.
    \item Unlike real property, copyright law \textbf{restricts the 
    alienability of copyrights} in various ways.\footnote{Casebook p. 532.}
\end{enumerate}

\paragraph{Division and Transfer under the 1909 Act}

\begin{enumerate}
    \item \textbf{Indivisibility}: copyright holders could not divide their 
    right. They could assign the whole thing, but any lesser transfer was 
    considered a license.
\end{enumerate}

\paragraph{Division and Transfer under the 1976 Act}

\begin{enumerate}
    \item Eliminated restrictions on indivisibility. Allows for exclusive 
    licenses. Any holder of an exclusive license can bring an infringement 
    suit.\footnote{Casebook p. 533.}
    \item \textbf{Transfer of copyright}: any sale or assignment of all or 
    part of the copyright; any exclusive or non-exclusive license; and/or any 
    mortgage or hypothecation.
\end{enumerate}

\paragraph{Reclaiming Copyrights (Renewal and Termination of Transfer)}

\begin{enumerate}
    \item 1909 Act--\textbf{termination of transfer}: upon renewal, copyright 
    holders could reclaim copyright interests that they had licensed. (But 
    most licensees insisted on advance assignment.)\footnote{Casebook p. 533.}
    \item 1976 Act:
    \begin{enumerate}
        \item Eliminated the renewal requirement.
        \item Copyright holders could terminate transfers of copyright between 
        the thirty-fifth and fortieth year from the execution of the transfer. 
        Congress wanted to give stronger rights to authors and their families. 
        This violates freedom of contract, but it compensates for publishers' 
        ``unequal bargaining power.''\footnote{Casebook p. 533.}
    \end{enumerate}
\end{enumerate}

\subsection{Traditional Rights of Copyright Owners}

\subsubsection{17 U.S.C. \S\ 106: Exclusive Rights}

\begin{enumerate}
    \item Reproduction.
    \item Derivative works.
    \item Distribution.
    \item Performance.
    \item Display.
\end{enumerate}

\subsubsection{The Right to Make Copies}

\paragraph{Copying: \emph{Arnstein v. Porter}}

\begin{enumerate}
    \item Arnstein claimed Porter copied one of his songs.
    \item District court granted Porter's summary judgment motion.
    \item What's the standard for proving infringement? Two elements:
    \begin{enumerate}
        \item \textbf{Copying}.\footnote{Casebook p. 539.}
        \begin{enumerate}
            \item Proof of access.
            \item Striking similarity.
        \end{enumerate}
        \item That the copying went to far as to constitute \textbf{improper 
        appropriation}.
    \end{enumerate}
    \item Held:
    \begin{enumerate}
        \item Similarities alone do not permit the inference that Porter 
        copied the songs. However, if there was enough evidence of access for 
        the case to go to a jury, the jury could infer that the similarities 
        did not result from coincidence.\footnote{Casebook p. 540.}
        \item The standard is the judgment of the lay listener, not the 
        expert.
    \end{enumerate}
\end{enumerate}

\paragraph{Second vs. Seventh Circuit: Do You Need to Show Access to Prove 
Infringement?}

\begin{enumerate}
    \item Second: no evidence of access needed if there is enough similarity.
    \item Seventh: infringement suit must show \emph{some} evidence of access.
\end{enumerate}

\paragraph{Improper Appropriation: \emph{Nichols v. Universal Pictures 
Corporation}}

\begin{enumerate}
    \item Did the film \emph{The Cohens and the Kellys} infringe the play 
    \emph{Abie's Irish Rose}?
    \item Judge Hand: ``~.~.~.~the right \textbf{cannot be limited literally 
    to the text}, or else a plagiarist would escape by immaterial 
    variations.''\footnote{Casebook p. 546.}
    \item \textbf{Abstractions test}: ``Upon any work, and especially upon a 
    play, \textbf{a great number of patterns of increasing generality will fit 
    equally well, as more and more of the incident is left out}~.~.~.~there is 
    a point in this series of abstractions where they are no longer protected 
    [because they are the idea, not the expression].''\footnote{Casebook p. 
    546--47.}
\end{enumerate}

\paragraph{Computer Software and the ``Abstraction, Filtration, Comparison'' 
Test: \emph{Computer Associates International v. Altai}}

\begin{enumerate}
    \item The software in question was (1) CA scheduler (a task manager) and 
    (2) CA adapter (an OS interface module).
    \item Allegations of copying involved a departing employee.
    \item CA argued that Altai infringed its copyright in its software. Altai 
    argued that it took steps to ensure that the literal elements of its 
    software were no longer substantially similar to the literal elements of 
    CA's program. But what about the structure?
    \item Software is both literary expression (protected) and process or 
    method (unprotected under \S\ 102(b)).
    \item \textbf{Merger}: ``we conclude that those elements of a computer 
    program that are necessarily incidental to its function are similarly 
    unprotectable.''\footnote{Casebook p. 559.} I.e., there is only one 
    expression for a certain idea, so the idea and the expression merge.
    \item CA approach---the \textbf{abstraction, filtration, comparison test}:
    \begin{enumerate}
        \item \textbf{Abstract} the program into its structural parts.
        \item \textbf{Filter} out the merged ideas and public domain parts.
        \item \textbf{Compare} the remaining protected pieces with the 
        structural parts of the allegedly infringing program.
    \end{enumerate}
\end{enumerate}

\paragraph{Limitations on the Exclusive Right to Copy}

\begin{enumerate}
    \item See pp. 572--73.
\end{enumerate}

\subsubsection{The Right to Prepare Derivative Works}

\begin{enumerate}
    \item \textbf{Derivative work}: based on a preexisting work. \S\ 101.
    \item \S\ 106(2): copyright owner has the exclusive right to prepare 
    derivative works.
    \item Generic ``scenes a faire'' and stock characters are not 
    copyrightable, but specific characters are. See \emph{Stallone} below.
    \item Aren't all derivative works ``substantially similar,'' rendering 
    them infringing, rendering \S\ 106(2) superfluous? Often, yes---but if 
    there is a chain of derivatives, the last may not be substantially similar 
    to the original.
\end{enumerate}

\paragraph{Derivative Works and Fictional Characters: \emph{Anderson v. 
Stallone}}

\begin{enumerate}
    \item Stallone publicly sketched some ideas for \emph{Rocky IV}. Anderson 
    wrote a 31-page script based on the ideas and presented it to MGM.
    \item The court granted summary judgment to the defendants. It held:
    \begin{enumerate}
        \item The \emph{Rocky} characters are copyrightable.
        \item Anderson lacked copyright in his script because it was an 
        unauthorized derivative work (under \S\ 102(6)) based on these 
        characters.
    \end{enumerate}
\end{enumerate}

\subsubsection{The Distribution Right}

\paragraph{First Sale Doctrine: \emph{Kirstaeng v. Wiley}}

\begin{enumerate}
    \item Kirtsaeng imported Wiley textbooks from Thailand, where they were 
    cheaper, and sold them at a profit in the U.S.
    \item Wiley argued that Kirtsaeng violated its ``exclusive right to 
    distribute copies'' under 17 U.S.C. \S\ 602(a)(1).
    \item Kirtsaeng argued he was protected under the first sale doctrine, 
    i.e., the rightsholder cannot control sales past the first sale. The work 
    (copyrightable) is distinct from the copy (unprotectable). First sale: 17 
    U.S.C. \S\ 109.
    \item Supreme Court (grudgingly) found for Kirtsaeng.
    \item First sale doctrine applies \textbf{only to sales---not to 
    licenses}.
\end{enumerate}

\subsubsection{Public Performance and Display Rights}

\begin{enumerate}
    \item If it moves, it's a performance, and if it's still, it's a display.
    \item No public display rights in architectural works.\footnote{Casebook 
    p. 588.}
    \item Limited public performance right in \emph{sound recordings}---see 
    below. But broadcasting a recording still infringed the \emph{musical 
    composition}; licenses for broadcasting are typically handled through 
    blanket licenses from ASCAP, BMI, or SESAC.\footnote{Casebook p. 588 n. 
    31.}
    \item The distinction between performance and display matters. Displaying 
    a painting in an art gallery is not infringement, but showing a movie in a 
    movie theater is.
    \item See \S\ 101 for definitions of ``performance'' and ``display.''
\end{enumerate}

\paragraph{Public Performance of Sound Recordings}

\begin{enumerate}
    \item When sound recordings entered federal copyright protection in 1972, 
    broadcasters convinced Congress to \textbf{deny public performance rights 
    for sound recordings}.
    \item However, there are performance rights for \textbf{digital sound 
    recordings}. Digital Performance Right in Sound Recordings Act 
    (1995).\footnote{Casebook p. 591.}
\end{enumerate}

\paragraph{Statutory Limits on Performance and Display Rights}

\begin{enumerate}
    \item Statutory exemptions limiting performance and display rights (in 
    addition to fair use):
    \begin{enumerate}
        \item \textbf{Public interest exemption} (\S\ 110):
        \begin{enumerate}
            \item Generally applies to educational, free, or charitable 
            performances and displays.\footnote{Casebook p. 592.}
            \item Fairness in Music Licensing Act (1998): broadened exemptions 
            for homes, small business, restaurants, and certain larger 
            establishments.
        \end{enumerate}
        \item \textbf{Compulsory licenses} (\S\ 111):
        \begin{enumerate}
            \item Cable, satellite, jukeboxes, public broadcasting, 
            webcasting.\footnote{Casebook p. 593--92.}
        \end{enumerate}
    \end{enumerate}
\end{enumerate}

\subsubsection{Moral Rights}

\begin{enumerate}
    \item Apply only to works of visual art. \S\ 106(a). Results from the 
    Visual Artists Rights Act (1990), a consequence of Berne. For specific 
    protections, see casebook pp. 594--95.
\end{enumerate}

\subsection{Indirect Liability}

\begin{enumerate}
    \item Indirect liability: applies to those who \textbf{contribute to, 
    induce, or profit from} infringement, or those who \textbf{sell products} 
    that others can use to infringe.\footnote{Casebook p. 598.}
    \item Common law roots (vicarious liability, etc.) with major implications 
    for the digital age.
    \item Line between vicarious liability and indirect liability: whether a 
    person exercises \textbf{direct legal control} over the other.
\end{enumerate}

\subsubsection{Contributory Infringement: \emph{Sony v. Universal}}
~\\\\
Sony found not liable for contributory infringement. (See below for details on 
the fair use analysis.)

\begin{enumerate}
    \item Contributory infringement: first, there needs to be an act of 
    \textbf{actual infringement}.
    \item Universal and Disney argued that Sony was liable for contributory 
    infringement because its customers were using Betamax machines for time 
    shifting.
    \item District court: no direct or indirect infringement. There was an 
    implied home taping privilege.
    \item Ninth Circuit: Sony was liable for contributory infringement; remand 
    to determine remedies.
    \item Issue before the Supreme Court: if Sony knew that the primary use of 
    its machines was to make illegal copies, was it liable for contributory 
    infringement?
    \item Justice Stevens:
    \begin{enumerate}
        \item Time-shifting was fair use, so the Betamax had 
        \textbf{substantial noninfringing uses}, so it was ok.
    \end{enumerate}
    \item Justice Blackmun, dissenting: 
    \begin{enumerate}
        \item Private taping was not fair use.
    \end{enumerate}
    \item Dispute between the majority and the dissent focused on the nature 
    of the copying. Stevens treated it as \textbf{time-shifting}; Blackmun 
    treated it as \textbf{archiving/librarying}. The potential effect on the 
    market was key.
\end{enumerate}

\subsection{Defenses}

\subsubsection{Fair Use}

\paragraph{17 U.S.C. \S\ 107: Four Fair Use Factors}

\begin{enumerate}
    \item \textbf{Purpose and character} (e.g., commercial/nonprofit).
    \item \textbf{Nature of the copyrighted work}.
    \item \textbf{Amount} and substantiality of the portion used.
    \item \textbf{Effect of the use upon the potential market} or value of the 
    original.
    \item (Codified in 1976; evolved from common law.)
\end{enumerate}

\paragraph{Videotaping and ``Substantial Noninfringing Uses'': \emph{Sony 
Corporation of America v.  Universal City Studios, Inc.}}
~\\\\
Overcoming fair use requires the copyright holder to show a likelihood of 
harm. (See below for more details on Sony.)

\begin{enumerate}
    \item Universal sued Sony over VCRs for indirect infringement.
    \item Buyers used VCRs for time shifting (recording shows and 
    watching them later). Sony could only be found liable indirectly if time 
    shifting was directly infringing. So, the issue was whether time shifting 
    is fair use.
    \item Plaintiffs had less than 10\% market share. There may have been 
    other broadcasters who would have welcomed time shifting. In an action for 
    infringement against a seller of equipment, the holder can only prevail if 
    the infringement affects only his programs or if he speaks for virtually 
    all copyright holders with a stake in the outcome.\footnote{Casebook p. 
    624.}
    \item Time shifting was noninfringing. It counted as a substantial 
    noninfringing use. 
    \item Courts should ask if the use is commercial or nonprofit. 
    Noncommercial uses fall outside fair use if they would adversely affect 
    the market if they became widespread. Here, the plaintiffs failed to show 
    current or future harm.\footnote{Casebook pp. 625--27.}
    \item Justice Blackmun, dissenting: this does harm the market. There's no 
    private use exception.
\end{enumerate}

\paragraph{Critique of the ``Substantial Noninfringing Use'' Standard}

\begin{enumerate}
    \item If 85\% of uses are infringing and 15\% are infringing, there is no 
    liability. This is inefficient and costly.
    \item Menell and Nimmer: apply a ``reasonable alternative design'' 
    standard (from tort law). A different product design could significantly 
    reduce infringement at very little cost.
    \item Lanier: machines are shaping society instead of the other way 
    around.
\end{enumerate}

\paragraph{\emph{Harper \& Row Publishers, Inc. v. Nation Enterprises}}

\begin{enumerate}
    \item An anonymous source disclosed part of a forthcoming Gerald Ford 
    memoir to the Nation, which published excerpts from it.\footnote{Casebook 
    p. 610--11.} It was times to scoop the same piece to come out in Time. As 
    a result, Time canceled its agreement with Harper \& Row.
    \item Second Circuit: this was fair use.
    \item Justice O'Connor:
    \begin{enumerate}
        \item The author has a right to control prepublication of works.
        \item Commercial uses are presumptively \emph{not} fair uses. The 
        effect on the market is a key factor.
        \item 1992: unpublished nature of work does not bar fair use.
        \item \S\ 107: news reporting is one example of fair use.
        \item Held: this was not fair use.
    \end{enumerate}
    \item Wendy Gordon, fair use as market failure: fair use makes sense when 
    there are no opportunities for private agreements---unfavorable reviews of 
    a book.
\end{enumerate}

\paragraph{Photocopying and the Four Factors: \emph{American Geophysical Union 
v. Texaco Inc.}}

\begin{enumerate}
    \item Instead of buying multiple journal subscriptions, Texaco would buy 
    one and let its scientists make photocopies for their personal archives.
    \item Here, fair use aims to strike a balance between the ``integrity of 
    copyright'' and ``the benefits that photocopying technology 
    offers.''\footnote{Casebook p. 629.}
    \item Four \S\ 107 factors:
    \begin{enumerate}
        \item \textbf{Purpose and character of use}: commercial or nonprofit? 
        Transformative? The main purpose here was to archival---``personal 
        archival copies''---to make copies to avoid payment. The use was not 
        transformative because it did not add anything new of 
        value. It's a duplication, not a transformation. 
        Weighs against Texaco.\footnote{Casebook pp. 631--32.}
        \item \textbf{Nature of copyrighted work}: ``manifestly factual,'' and 
        therefore entitled to lower protection. Favors Texaco.
        \item \textbf{Amount and substantiality of portion used}: here, all of 
        it. Weighs against Texaco.
        \item \textbf{Effect upon potential market or value}: there is no 
        traditional market for individual journal articles. If there were no 
        market, unauthorized use should be considered ``more fair'' (the 
        \emph{market failure} theory of fair use). However, the publishers 
        could have earned licensing fees through a clearinghouse. There 
        \emph{was} a market. An unauthorized use is ``less fair'' when there 
        is a market to pay for the use.
        \item Judge Newman also looked at the impact of a fair use holding on 
        \emph{future} markets (e.g., copyright clearinghouses that are in the 
        process of developing).
    \end{enumerate}
    \item Held for the publishers since they were able to show ``substantial 
    harm.''\footnote{Casebook p. 636.}
\end{enumerate}

\paragraph{Parodies: ``Separating the Fair Use Sheep from the Infringing Goats:'' 
\emph{Campbell v. Acuff-Rose Music, Inc.}}
~\\\\
Transformative parodies are fair use, although courts should consider the 
effects on derivative markets.

\begin{enumerate}
    \item Parodies criticize specific works; satires criticize society 
    broadly.
    \item 2 Live Crew parodied ``Pretty Woman,'' using only as much as was 
    necessary to ``conjure up'' the original.
    \item 2LC's use would be infringing but for a finding of fair 
    use.\footnote{Casebook p. 641.}
    \item \textbf{Purpose and character}:
    \begin{enumerate}
        \item Does it ``supersede the objects'' of the original or add 
        something new?
        \item The more transformative, the less important are the other 
        factors, including commercialism. (Commercial use is not presumptively 
        harmful.)
        \item Parody needs to mimic the original.\footnote{Casebook p. 643.}
    \end{enumerate}
    \item \textbf{Nature of the copyrighted work}:
    \begin{enumerate}
        \item ``~.~.~.~some works are closer to the core of intended copyright 
        protection than others''--e.g., expressive works, as opposed to 
        factual works.
        \item This factor is not ``likely to help much in separating the fair 
        use sheep from the infringing goats'' because parodies almost always 
        copy publicly known, expressive works.\footnote{Casebook p. 645.}
    \end{enumerate}
    \item \textbf{Amount and substantiality}:
    \begin{enumerate}
        \item ``~.~.~.~extent of permissible copying varies with the purpose 
        and character of the use.''\footnote{Casebook p. 645.}
        \item Parodies require the original to be recognizable. This parody 
        added something new in addition, and it took no more than what was 
        necessary.\footnote{Casebook p. 646.}
    \end{enumerate}
    \item \textbf{Effect on the market}:
    \begin{enumerate}
        \item Parodies are not substitutes for the original.
        \item The district court failed to address the possible effect of the 
        parody on derivative rap markets (e.g., if the copyright holder wanted 
        to create its own rap parody of ``Pretty Woman''). Remanded for 
        factfinding.
    \end{enumerate}
\end{enumerate}

\paragraph{Remixes: \emph{Bill Graham Archives v. Dorling Kindersley Ltd.}}
~\\\\
Remixes can copy entire works if the use is transformative.

\begin{enumerate}
    \item DK published a coffee table book on the Grateful Dead that included 
    seven images of posters copyrighted by Bill Graham Archives.
    \item \textbf{Purpose and character}:
    \begin{enumerate}
        \item DK's use was transformative because its purpose (biography) was 
        different than the images' original purpose (promotion). DK published 
        the images at a smaller size and in collages. Finally, BGA's images 
        were a tiny fraction of DK's overall work. The commercial purpose of 
        DK's use was not dispositive---``DK does not seek to exploit the 
        images' expressive value for commercial gain.''\footnote{Casebook pp. 
        654--56.}
    \end{enumerate}
    \item \textbf{Nature of the copyrighted work}: although the original 
    images were expressive and thus entitled to strong copyright protections, 
    DK did not exploit the images' expressive value.\footnote{Casebook pp. 
    656--57.}
    \item \textbf{Amount and substantiality}: DK copied the images in their 
    entirety, but the reduced size was consistent with the transformative 
    use.\footnote{Casebook pp. 657--58.}
    \item \textbf{Effect upon the market}: DK's use did not harm BGA's license 
    market, even if DK had been willing to pay licensing fees to BGA and other 
    copyright holders.\footnote{Casebook pp. 658--59.}
\end{enumerate}

\paragraph{Reverse Engineering: \emph{Sega Enterprises, Ltd. v. Accolade, Inc.}}

\begin{enumerate}
    \item Accolade took apart Sega cartridges to discover an initialization 
    code to make games run on the Genesis console.\footnote{Casebook p. 671.}
    \item Ninth Circuit here: disassembly is fair use because it's 
    \textbf{necessary to understand the functional requirements} for 
    compatibility with the Genesis.
    \item ``We conclude that where disassembly is the \textbf{only way to gain access 
    to the ideas and functional elements} embodied in a copyrighted computer 
    program and where there is a legitimate reason for seeking such access, 
    disassembly is a fair use of the copyrighted work, as a matter of 
    law.''
    \item (DMCA prevents circumvention of technological measures that control 
    access to a work---but it provides an exemption for reverse engineering 
    for the ``sole purpose of identifying and analyzing those elements of the 
    program that are necessary to achieve \textbf{interoperability} of an 
    independently created computer program~.~.~.~'')
\end{enumerate}

\subsection{Digital Copyright Law}

\subsubsection{Digital Copyright Legislation}

\paragraph{Anticircumvention Prohibitions}

\begin{enumerate}
    \item Prohibits two broad sets of activities:
    \begin{enumerate}
        \item Circumventing and making items designed primarily to evade 
        ``technological protection measures'' (TPMs) for \textbf{access 
        control}. 17 U.S.C. \S\ 1201(a).
        \item Making items designed primarily to evade TPMs for \textbf{copy 
        control}. 17 U.S.C. \S\ 1201(b).
    \end{enumerate}
    \item \emph{Realnetworks}: Streambox VCR product violated the DMCA by 
    circumventing authentication measures to allow users to download streaming 
    content.\footnote{Casebook p. 692.}
\end{enumerate}

\paragraph{Online Service Provider Safe Harbors}

\begin{enumerate}
    \item Online service providers: ISPs (e.g., Comcast), platforms (e.g., 
    Yahoo!).\footnote{Casebook p. 694 ff.}
    \item OSPs must have and use termination policies for repeat infringers.
    \item They must adopt ``standard technological measures'' to screen 
    copyrighted works.
    \item They must identify a ``notice and takedown'' agent.
\end{enumerate}

\paragraph{Safe Harbors: \emph{Viacom Int'l, Inc. v. YouTube}}

For the OSP to be liable, \S\ 512(c)(1)(A) (safe harbors) requires it to have 
knowledge or awareness of \emph{specific} infringing activity. Remanded for 
findings on other issues: (1) right and ability to control infringing acts, 
(2) financial benefit from infringement. 512(c)(1)(B).

\paragraph{\emph{MGM v. Grokster}}

\begin{enumerate}
    \item See Merges PPT 10/14/13.
\end{enumerate}

\subsubsection{Fair Use in Cyberspace: \emph{Perfect 10, Inc. v. Amazon}}

Google Image search thumbnails are fair use. A search engine ``transforms the 
image into a pointer, directing a user to a source of information.''

\begin{enumerate}
    \item Perfect 10 sued Google for showing its images in Google Images 
    search results.\footnote{Casebook p. 732.}
    \item Specifically, Perfect 10 argued that Google's thumbnails infringed 
    its \textbf{display rights and distribution rights}.
    \item Held: thumbnails \emph{transform} the original, rather than copy or 
    frame it, so they do not violate the display right.
    \item Also held: links to the original do not violate the distribution 
    right.
\end{enumerate}

\subsubsection{Webcasting}

\begin{enumerate}
    \item Compulsory licenses or private negotiations?\footnote{See Merges PPT 
    10/14/13.}
    \item Current law: ``interactive services'' must negotiate directly with 
    copyright owners; non-interactive services have a compulsory license.
    \item Lessig, Fisher: there should be an Internet-wide compulsory license 
    for downloading music. We should separate compensation from control. See 
    \emph{Free Culture}.
\end{enumerate}

\subsection{International Copyright Law}

\begin{enumerate}
    \item Lack of formalities has led to nearly worldwide protection for most 
    works.\footnote{Casebook p. 742.}
\end{enumerate}

\subsubsection{Evolution of the International Copyright System and U.S. 
Participation}

\begin{enumerate}
    \item Berne, 1886: foreign authors get ``domestic treatment''---i.e., the 
    same rights as domestic authors.\footnote{Casebook p. 742.}
    \item The United States joined in 1989.\footnote{Casebook p. 743.}
\end{enumerate}

\subsubsection{International Copyright Treaties}

\paragraph{Berne Convention for the Protection of Literary and Artistic Works}

\begin{enumerate}
    \item \textbf{National treatment}: foreign authors from Berne member 
    nations receive the same protections as domestic 
    authors.\footnote{Casebook p. 744.}
    \item Minimum requirements:\footnote{Casebook pp. 744--45.}
    \begin{enumerate}
        \item Works covered.
        \item Limitations on formalities.
        \item Duration.
        \item Exclusive rights.
        \item Moral rights.
        \item Restoration of rights.
    \end{enumerate}
\end{enumerate}

\paragraph{Agreement on Trade-Related Aspects of Intellectual Property Rights 
(TRIPs)}

\begin{enumerate}
    \item Expands Berne, including more extensive enforcement obligations and 
    a new WTO dispute resolution process.\footnote{Casebook p. 746.}
    \item Expansions on Berne:\footnote{Casebook p. 746.}
    \begin{enumerate}
        \item Works covered.
        \item Exclusive rights.
        \item Exceptions.
    \end{enumerate}
\end{enumerate}

\subsubsection{Protection of U.S. Works Against Infringement Abroad}

\begin{enumerate}
    \item U.S. copyright law does not apply beyond the United States. 
    Copyright holders can only enforce their rights in nations that are 
    parties to multi-lateral treaties (Berne, TRIPs) or where the United 
    States has specific bilateral agreements.\footnote{Casebook p. 747.}
    \item Standing is the threshold question.\footnote{Casebook p. 747.}
    \item The extend of protections depend on applicable law. Local law also 
    determines applicability and remedies.\footnote{Casebook p. 748.}
    \item \textbf{Rule of the shorter term}: under Berne, if the term of 
    protection is different between the protecting country and the country of 
    origin, the shorter duration applies, unless the protection country 
    chooses otherwise.\footnote{Casebook p. 748.}
\end{enumerate}

\subsubsection{Protection of Foreign Works Against Infringement in the United 
States}

\begin{enumerate}
    \item The United States protects all copyrights originating in Berne and 
    TRIPs countries.
    \item The United States does not apply the rule of the shorter term. So, 
    if the term is shorter in the originating country, the work still enjoys 
    the full term of U.S. copyright protection.\footnote{Casebook p. 751.}
\end{enumerate}

\subsection{Enforcement and Remedies}

\subsubsection{Injunctions}

\begin{enumerate}
    \item Before 2006, courts routinely granted injunctive relief. This 
    approach has since been replaced by a ``searching, equitable balancing 
    framework.''\footnote{Casebook p. 752.}
    \item The \emph{eBay} court established the four-factor test for 
    determining whether injunctive relief is called for. The plaintiff must 
    show:\footnote{Casebook p. 753.}
    \begin{enumerate}
        \item Irreparable injury;
        \item The inadequacy of other remedies;
        \item Considering the balance of hardships between the plaintiff and 
        defendant, an injunction is warranted; and
        \item ``~.~.~.~that the public interest would not be 
        disserved~.~.~.''
    \end{enumerate}
\end{enumerate}

\subsubsection{Damages}

\paragraph{Actual Damages and Profits: \emph{Sheldon v. Metro-Goldwyn Pictures 
Corp.}}
~\\\\
Damages should be compensatory, not punitive.

\begin{enumerate}
    \item What portion of the infringer's profits should be awarded to the 
    copyright holder?
    \item The district court (grudgingly, but with obedience to precedent) 
    awarded \emph{all} of the defendants' profits.\footnote{Casebook p. 755.}
    \item The appellate court awarded one-fifth of profits.
    \item Damages should give just compensation. They should not be 
    punitive---even if the infringement was deliberate.\footnote{Casebook p. 
    757.}
\end{enumerate}

\paragraph{Statutory Damages}

\begin{enumerate}
    \item Available to copyright holders who registered their 
    work.\footnote{Casebook p. 759.}
    \item Unintentional infringement: \$750 to \$30,000 (or \$200 as a lower 
    bound if the infringer had no reason to believe the activity constituted 
    infringement) per infringed work.
    \item Willful infringement: up to \$150,000 per infringed work.
\end{enumerate}

\subsubsection{Attorney Fees}

\begin{enumerate}
    \item Can be awarded to prevailing plaintiffs or defendants on an equal 
    basis.\footnote{Casebook p. 761.}
\end{enumerate}
