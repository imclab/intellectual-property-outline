\section{Copyright}

\subsection{Introduction}

\subsubsection{17 U.S.C. \S\ 101: Definitions}

See supplement p. 248.

\subsubsection{History}

\begin{enumerate}
    \item Venice in the fifteenth century recognized exclusive rights in the 
    printing of particular books (as opposed to the technology of creating the 
    books).\footnote{Casebook p. 430.}
    \item \textbf{Statute of Anne} (1710): authors gain exclusive rights for 
    14 years, renewable for another 14 years, subject to registration, notice, 
    and deposit. The government could also set maximum 
    prices.\footnote{Casebook p. 431.}
    \item The continental approach emphasized moral rights, as opposed to the 
    English system of property rights.
    \item The original Copyright Act (1790) mirrored the Statute of Anne. By 
    the end of the nineteenth century, it had been amended to include a range 
    of new technologies (e.g., photography).\footnote{Casebook p. 432.}
    \item \textbf{1909 Act}: extended protection to ``all writings''; extended 
    term to 28 years plus a 28 year renewal.
    \item \textbf{1976 Act}: extended protection to anything ``fixed in a 
    tangible medium of expression''; extended term to life of the author plus 
    50 years; loosened notice and registration requirements; established 
    compulsory licensing regimes; and codified exceptions, including fair use. 
    The 1998 Act extended the term to life of the author plus seventy 
    years.\footnote{Casebook p. 433.}
    \item 1989: the United States joined the Berne convention, scaling back 
    formalities and restoring copyright for foreign works still under 
    protection in the source country.
    \item \textbf{DMCA} (1998): anticircumvention provisions and liability 
    protections for service providers for infringing acts of their 
    subscribers.\footnote{Casebook pp. 433--34.}
\end{enumerate}

\subsubsection{Overview}

\paragraph{Elements of a Protectable Copyright}

\begin{enumerate}
    \item \textbf{Copyrightable subject matter}: any literary or artistic 
    expression (but not the idea itself).
    \item \textbf{Threshold for protection}: must (1) have a ``modicum of 
    originality'' and (2) be fixed in a ``tangible medium of expression.''
    \item \textbf{Formalities}: notice is required for works published before 
    1989. Registration is required to sue for infringement. Deposit is 
    required for registration.
    \item \textbf{Authorship and ownership}: only the creator, transferee, or 
    employer can bring an infringement suit.
    \item \textbf{Duration of copyright}: life plus 70 years; or 95 years from 
    the publication of anonymous, pseudonymous, or for-hire works; or 120 
    years from the year of creation, whichever comes first.
    \item The Copyright Office does not assess validity (other than requiring 
    a modicum of creativity) or prior art.
    \begin{enumerate}
        \item TODO: does the copyright office ever reject applications?
    \end{enumerate}
    \item A copyright is protectable at the moment of 
    creation.\footnote{Casebook p. 434.}
\end{enumerate}

\paragraph{Ownership Rights}

\begin{enumerate}
    \item \textbf{Reproduction}: owner has the exclusive right to make copies. 
    She can sue for ``material'' and ``substantial'' copying.
    \item \textbf{Derivative works}: original owner has the exclusive right to 
    prepare derivative works (e.g., translations).
    \item \textbf{Distribution}: the owner controls sale and distribution, but 
    only for the first sale.
    \item \textbf{Performance and display}: owner has the right to control.
    \item \textbf{Anticircumvention}: prohibitions on bypassing technological 
    protections.
    \item \textbf{Moral rights}: visual artists have an attribution right and 
    a right to prevent ``intentional distortion, mutilation, or other 
    modification~.~.~.~''\footnote{Casebook p. 435.}
    \item Limitations: fair use, compulsory licensing.
    \item Copyright vs. patent: copyright holders can prevent copying and 
    certain uses (e.g., public performance), but they cannot prevent others 
    from making, using, or selling their creations.
\end{enumerate}

\subsubsection{Philosophy}

\begin{enumerate}
    \item Personhood, natural law.
    \item Predominant in the United States: utilitarian. The primary goal is 
    to enhance the public interest. The secondary goal is to reward 
    authors.\footnote{Casebook p. 436.}
    \item There is an increasing recognition of moral rights in the United 
    States.\footnote{Casebook p. 437.}
\end{enumerate}

\subsection{Requirements}

\subsubsection{Original Works of Authorship}

\paragraph{17 U.S.C. \S\ 102: Subject Matter of Copyright: In General}

\begin{enumerate}
    \item Original.
    \item ``~.~.~.~fixed in any tangible medium of expression~.~.~.~''
    \item A work gains copyright protection if it was independently created, 
    even if it's identical---see Learned Hand on Keats.\footnote{Casebook p. 
    439.} The originality requirement is very low; exceptions are slogans, 
    familiar symbols, etc.
\end{enumerate}

\paragraph{No Copyright for Facts: \emph{Feist Publications v. Rural Telephone 
Services}}
~\\\\Copyright requires originality. Facts and unoriginal creations cannot be 
copyrighted.

\begin{enumerate}
    \item Feist copied records from Rural's phonebook for its own phonebook. 
    Rural sued for infringement.
    \item Justice O'Connor:
    \begin{enumerate}
        \item Facts are not copyrightable. But compilations of facts generally 
        are.\footnote{Casebook p. 441.}
        \item Originality is ``the touchstone of copyright 
        protection'' because it is constitutionally and statutorily 
        mandated.\footnote{Casebook p. 443.}Originality requires (1) 
        independent creation and (2) a minimal degree of creativity.  
        Facts are not original. Therefore, facts are not copyrightable.
        \item Copyright of an entire work does not imply copyright of each 
        element.\footnote{Casebook p. 443.}
        \item ``Not all copying, however, is copyright 
        infringement.''\footnote{Casebook p. 443.} Infringement requires 
        copying of elements that are original. 
        \item Rural's alphabetical arrangement ``is not only unoriginal, it is 
        practically inevitable.''\footnote{Casebook p. 445.} Since Rural's 
        arrangement was unoriginal, it was not copyrightable. Therefore, 
        Feist's copying was non-infringing.
    \end{enumerate}
\end{enumerate}

\subsubsection{Fixation in a Tangible Medium of Expression}

\paragraph{H.R. Rep. No. 94-1476: On the Copyright Act of 1976}

\begin{enumerate}
    \item Any medium satisfies the fixation requirement.\footnote{Casebook p. 
    449.} Unfixed works may still be eligible for protection under State 
    common or statutory law, but federal protection requires 
    fixation.\footnote{Casebook p. 450.}
    \item Broadcasts are covered by a provision that allows protections for 
    works that are simultaneously recorded and transmitted.
    \item ``The two essential elements---\textbf{original work} and 
    \textbf{tangible object}---must merge through fixation in order to produce 
    subject matter copyrightable under the statute.''\footnote{Casebook p. 
    451.}
    \item Fixation is (1) a requirement for protection and (2) plays a role in 
    determining whether a defendant has infringed a copyright (because copies 
    are material objects in which a work is ``fixed'').\footnote{Casebook p. 
    551.}
    \item Do bootlegs infringe? They're based on non-fixed performances. The 
    Second Circuit upheld the anti-bootlegging criminal provisions under the 
    Commerce Clause because they do not create additional rights; however, the 
    implication is that the civil provisions are 
    unconstitutional.\footnote{Casebook pp. 451--52.}
    \item Why require fixation?\footnote{Casebook p. 452.}
    \begin{enumerate}
        \item If copyright is meant to protect communication, then it should 
        not apply to expressions that do not actually communicate.
        \item It's a practical requirement for litigation (cf. the statute of 
        frauds in contract law).
    \end{enumerate}
\end{enumerate}

\subsubsection{Formalities}

\paragraph{Notice}

\begin{enumerate}
    \item 1909 Act: failure to follow precise notice requirements resulted in 
    forfeiture.
    \item 1976 Act/Pre-Berne: copyright begins upon creation, not 
    publication. Notice was still required, but the requirements were 
    significantly looser.
    \item Post-Berne: completely eliminated the notice requirement, but 
    encouraged voluntary notice (e.g., by allowing the innocent infringement 
    defense if the work lacked proper notice).\footnote{Casebook p. 455.}
\end{enumerate}

\paragraph{Publication}

\begin{enumerate}
    \item 1909 Act: publication triggered copyright. Unpublished works could 
    be protected under state common law, or ``constructively'' published by 
    registration with the Copyright Office.
    \begin{enumerate}
        \item \emph{Divestive} publication: results in forfeiture of common 
        law copyright protection. 
        \item \emph{Investive} publication: resulting in forfeiture of federal 
        statutory protection if notice is inadequate.
    \end{enumerate}
    \item 1976 Act/Pre-Berne: creation triggered copyright, but still 
    determined when notice was required.\footnote{Casebook pp. 456--57.}
    \item Post-Berne: after March 1, 1989, publication no longer determines 
    validity, though it still has relevance in several areas---see casebook p. 
    457.
\end{enumerate}

\paragraph{Registration}

\begin{enumerate}
    \item Has always been voluntary.
    \item 1909 Act: term was 28 years, plus a 28 year renewal, but only if the 
    owner registered.
    \item 1976 Act/Pre-Berne: abolished renewal, but created new incentives 
    for registration.\footnote{Casebook p. 458.}
    \item Post-Berne: no registration requirement for foreign works, but 
    Congress preserved the registration requirement for domestic works for 
    bringing infringement suits.
\end{enumerate}

\paragraph{Deposit}

\begin{enumerate}
    \item After 1976, still mandatory, but failure results only in a fine, not 
    invalidity or forfeiture of the right to bring suit.\footnote{Casebook p. 
    459.}
\end{enumerate}

\paragraph{Restoration of Foreign Copyrighted Works}

\begin{enumerate}
    \item Post-Berne, many foreign works that had entered the public domain in 
    the United States (e.g., J.R.R. Tolkien's works) were restored to 
    copyright. The Supreme Court affirmed in \emph{Golan v. Holder}, holding 
    that the IP clause allows Congress to ``induc[e] the dissemination of 
    works~.~.~.~to promote science.''\footnote{Casebook p. 460.}
\end{enumerate}

\subsection{Copyrightable Subject Matter}

\subsubsection{Limitations on Copyrightability: Distinguishing Function and 
Expression}

\paragraph{The Idea-Expression Dichotomy}

\paragraph{17 U.S.C. \S\ 102(b)}

\begin{enumerate}
    \item Copyright protection does not extend to ``any idea, procedure, 
    process, system~.~.~.~.''\footnote{Casebook p. 461.}
\end{enumerate}

\paragraph{Use vs. Explanation: \emph{Baker v. Seden}}
~\\\\
A useful idea cannot be copyrighted, but an explanation of how to use it can 
be.

\begin{enumerate}
    \item Seden published a book about a ledger system, which included his 
    specific ledger design. He sued Baker for infringement. Baker argued that 
    the book was ``not a lawful subject of copyright.''\footnote{Casebook p. 
    461.}
    \item Justice Bradley:
    \begin{enumerate}
        \item Selden did not have a valid copyright in the ledger system 
        itself.\footnote{Casebook p. 462.}
        \item The book itself might be copyrightable, but the underlying 
        system is not.
        \item ``To give the author of the book an exclusive property in the 
        art described therein, when no examination of its novelty has ever 
        been officially made, would be a surprise and a fraud upon the public. 
        That is the province of letters-patent, not of copyright. The claim to 
        an invention or discovery of art or manufacture must be subjected to 
        the examination of the Patent Office before an exclusive right therein 
        can be obtained; it can only be secured from a patent from the 
        government.''\footnote{Casebook p. 462.}
        \item The use cannot be copyrighted, but the explanation can 
        be.\footnote{Casebook p. 463.}
    \end{enumerate}
    \item At a certain level of abstraction, descriptions of how to use an 
    idea are no longer copyrightable (Learned Hand).\footnote{Casebook pp. 
    465--66.}
\end{enumerate}

\newpage % TODO remove

\paragraph{Computer Menus and Subject Matter: \emph{Lotus Development Corp. v. 
Borland International}}
~\\\\
Computer menus are ``methods of operation'' (under 17 U.S.C. \S\ 102(b)) and 
therefore not copyrightable.

\begin{enumerate}
    \item Borland copied the menu configuration of Lotus 1-2-3.
    \item The district court found infringement. The question on appeal was 
    whether computer menus are copyrightable subject matter.
    \item This case is distinct from \emph{Baker v. Selden} because that case 
    involved the design of the grid, while this case involves the commands 
    used for interaction.\footnote{Casebook p. 471.}
    \item Held: the menu commands are a \textbf{``method of operation''} 
    because they are necessary for the program to work. Thus, they are not 
    copyrightable subject matter under \S\ 102(b). Lotus's underlying code is 
    copyrightable, but the menu structure is not.\footnote{Casebook p. 472.}
    \item The menu commands are analogous to the buttons on a VCR.
    \item Judge Boudin, concurring:
    \begin{enumerate}
        \item Users are likely to be locked in to Lotus's menu structure (like 
        a QWERTY keyboard). If Borland comes along with a better product, 
        there are good reasons for freeing users to make the switch.
        \item There are two ways Borland might prevail:\footnote{Casebook p. 
        477.}
        \begin{enumerate}
            \item The menu is a ``method of operation.''
            \item Borland's use is privileged under something like fair use, 
            because it enables users to exploit their previous efforts in 
            writing macros.
        \end{enumerate}
        \item The second approach would introduce administrative problems, so 
        the majority's approach is as good as it gets for 
        now.\footnote{Casebook p. 478.}
    \end{enumerate}
\end{enumerate}

\paragraph{The Useful Article Doctrine}

\begin{enumerate}
    \item Under \S\ 101, a ``useful article'' is a picture, sculpture, or 
    graphic only if the PSG features can be identified separately from the 
    ``utilitarian aspects.''\footnote{Casebook p. 482--83.}
\end{enumerate}

\paragraph{H.R. Rep. No. 94-1476: On the 1976 Act}

\begin{enumerate}
    \item Congress attempted to distinguish between ``copyrightable works of 
    applied art and uncopyrighted works of industrial 
    design.''\footnote{Casebook p. 483.}
\end{enumerate}

\paragraph{\emph{Brandir International, Inc. v. Cascade Pacific Lumber Co.}}

\begin{enumerate}
    \item Brandir developed bicycle racks based on wire sculptures. The 
    Copyright Office and the district court denied copyright.
    \item When are a work's design and utility ``conceptually separate''? 
    Courts have struggled to come up with a good test.\footnote{Casebook p. 
    484.}
    \item The court here rejected Judge Newman's test from \emph{Carol 
    Barnhart}, the ``temporal displacement'' test, which finds the two to be 
    separate if they ``stimulate in the mind'' a conceptual 
    separation.\footnote{Casebook pp. 484--85.}
    \item Instead, it adopted the Denicola test, which finds conceptual 
    separation if the artistic judgment was separate from the functional 
    influence, but not if the design elements reflect both design and 
    functional considerations.\footnote{Casebook p. 487.}
    \item The bike rack here was based on a wire sculpture of a bike. It was 
    enlarged to hold actual bikes, and slightly modified so that it could hold 
    bikes more efficiently.\footnote{Casebook pp. 486--87.}
    \item The court held that the rack was not copyrightable under the 
    Denicola test because its design was ``in its final form essentially a 
    product of industrial design.'' There was no separate, independent 
    artistic aspect.
    \item Judge Winter, dissenting: a better test would be whether the article 
    ``causes an ordinary reasonable observer to perceive an aesthetic concept 
    not related to the article's use.''\footnote{Casebook p. 489.} The rack 
    passes that test---for instance, it could be successfully displayed as a 
    sculpture with no utilitarian purpose---and so it should be copyrightable.
    % FIXME Additional tests: 490
\end{enumerate}

\paragraph{Government Works}

\begin{enumerate}
    \item Any ``law, which, binding every citizen, is free for publication to 
    all, whether it is a declaration of unwritten law, or an interpretation of 
    a constitution or statute.''\footnote{Casebook p. 493.}
    \item \emph{Veeck} centered on a dispute over whether model laws adopted 
    by municipalities must be released into the public 
    domain.\footnote{Casebook pp. 494--96.}
\end{enumerate}

\subsubsection{The Domain and Scope of Copyright Protection}

\begin{enumerate}
    \item \S\ 101 describes eight categories of protected works, but the list 
    is ``illustrative and not limitative.''\footnote{Casebook p. 498.}
\end{enumerate}

\paragraph{Literary Works}

\begin{enumerate}
    \item Courts will not judge artistic merits.
    \item ``Short phrases'' are not protected.
    \item Protection extends to non-literal elements (e.g., structure).
    \item Courts have struggled with whether to protect fictional 
    characters.\footnote{Casebook p. 499.}
\end{enumerate}

\paragraph{Pictorial, Graphic, and Sculptural Works}

\begin{enumerate}
    \item Most significant limitation: the utilitarian function 
    exception---see \emph{Brandir} above.
\end{enumerate}

\paragraph{Architectural Works}

\begin{enumerate}
    \item Protection applies to overall form and elements in the design, but 
    does not include ``individual standard features.''\footnote{Casebook p. 
    500.} Original, non-functional design elements are protected.
    \item Protection does not extend to pictorial representations of the 
    building. It also does not prevent others from modifying or destroying 
    it.\footnote{Casebook p. 501.}
\end{enumerate}

\paragraph{Musical Works and Sound Recordings}

\begin{enumerate}
    \item Different rules apply depending on whether the work is a 
    \textbf{musical work} or a \textbf{sound recording}.
    \item Musical works (sheet music, lyrics, arrangements): fully protected, 
    including performance rights.
    \item Sound recordings: no traditional performance rights (although they 
    now have a digital performance right). For radio broadcasts, the owner of 
    the copyright of the composition gets a royalty, but the performer does 
    not.\footnote{Casebook p. 502.}
\end{enumerate}

\paragraph{Dramatic, Pantomime, and Choreographic Works}

\begin{enumerate}
    \item Protection extends to written or fixed instructions. 
    \item Performance and display rights can vary depending on whether the 
    work is classified as dramatic or nondramatic.\footnote{Casebook p. 503.}
    \item Short dance steps are not protected in the same way that ``short 
    phrases'' are not protected as literary works.
\end{enumerate}

\paragraph{Motion Pictures and Other Audiovisual Works}

\begin{enumerate}
    \item Soundtracks are integral parts.\footnote{Casebook p. 504.}
\end{enumerate}

\paragraph{Semiconductors and Vessel Hulls}

\begin{enumerate}
    \item Both have received sui generis protections through separate acts of 
    Congress.\footnote{Casebook p. 504.}
\end{enumerate}

\paragraph{Derivative Works and Compliations}

\begin{enumerate}
    \item Derivative works: translations, film adaptations, etc. The original 
    copyright owner controls the rights to derivative works.
    \item Compilations: anthologies, encyclopedias, etc. Must involve ``some 
    minimal degree of creativity'' (\emph{Feist}), but just how much is a 
    contentious issue.\footnote{Casebook pp. 505--06.}
\end{enumerate}

\newpage % TODO remove

\subsection{Ownership and Duration}

% TODO 17 usc 201c, 301-05

\subsubsection{Initial Ownership of Copyrights}

% TODO 506-526

\subsubsection{Duration and Renewal}

% TODO 526-31

\subsubsection{Division, Transfer, and Reclaiming of Copyrights}

% TODO 531-36

\subsection{Traditional Rights of Copyright Owners}

% TODO 17 usc 106

\subsubsection{The Right to Make Copies}

% TODO 537-73

\subsubsection{The Right to Prepare Derivative Works}

% TODO 573-83

\subsubsection{The Distribution Right}

% TODO 583-88

\subsubsection{Public Performance and Display Rights}

% TODO 588-94

\subsubsection{Moral Rights}

% TODO 594-98

\subsection{Indirect Liability}

% TODO 598-609

\subsection{Defenses}

% TODO 35 usc 107

\newpage % TODO remove

\subsubsection{Fair Use}

\paragraph{Videotaping: \emph{Sony Corporation of America v. Universal City 
Studios, Inc.}}
~\\\\
Overcoming fair use requires the copyright holder to show a likelihood of 
harm.

\begin{enumerate}
    \item Universal sued Sony over VCRs for indirect infringement.
    \item Buyers used VCRs for time shifting (i.e., recording shows and 
    watching them later). Sony could only be found liable indirectly if time 
    shifting was directly infringing. So, the issue was whether time shifting 
    is fair use.
    \item Plaintiffs had less than 10\% market share. There may have been 
    other broadcasters who would have welcomed time shifting. In an action for 
    infringement against a seller of equipment, the holder can only prevail if 
    the infringement affects only his programs or if he speaks for virtually 
    all copyright holders with a stake in the outcome.\footnote{Casebook p. 
    624.}
    \item Courts should ask if the use is commercial or nonprofit. 
    Noncommercial uses fall outside fair use if they would adversely affect 
    the market if they became widespread. Here, the plaintiffs failed to show 
    current or future harm.\footnote{Casebook pp. 625--27.}
\end{enumerate}

\paragraph{Photocopying: \emph{American Geophysical Union v. Texaco Inc.}}

\begin{enumerate}
    \item Instead of buying multiple journal subscriptions, Texaco would buy 
    one and let its scientists make photocopies for their personal archives.
    \item Fair use aims to strike a balance between the ``integrity of 
    copyright'' and ``the benefits that photocopying technology 
    offers.''\footnote{Casebook p. 629.}
    \item Four \S\ 107 factors:
    \begin{enumerate}
        \item \textbf{Purpose and character of use}: commercial or nonprofit? 
        Transformative? The main purpose here was to archival---``personal 
        archival copies''---to make copies to avoid payment. The use was not 
        transformative because it did not add anything new of 
        value. It's a duplication, not a transformation. 
        Weighs against Texaco.\footnote{Casebook pp. 631--32.}
        \item \textbf{Nature of copyrighted work}: ``manifestly factual,'' and 
        therefore entitled to lower protection. Favors Texaco.
        \item \textbf{Amount and substantiality of portion used}: here, all of 
        it. Weighs against Texaco.
        \item \textbf{Effect upon potential market or value}: there is no 
        traditional market for individual journal articles. If there were no 
        market, unauthorized use should be considered ``more fair'' (the 
        \emph{market failure} theory of fair use). However, the publishers 
        could have earned licensing fees through a clearinghouse. There 
        \emph{was} a market. An unauthorized use is ``less fair'' when there 
        is a market to pay for the use.
        \item Judge Newman also looked at the impact of a fair use holding on 
        \emph{future} markets (e.g., copyright clearinghouses that are in the 
        process of developing).
    \end{enumerate}
    \item Held for the publishers since they were able to show ``substantial 
    harm.''\footnote{Casebook p. 636.}
\end{enumerate}

\newpage % TODO remove

\paragraph{Parodies---``Separating the Fair Use Sheep from the Infringing 
Goats:'' \emph{Campbell v. Acuff-Rose Music, Inc.}}
~\\\\
Parodies are fair use, although courts should consider the effects on 
derivative markets.

\begin{enumerate}
    \item Parodies criticize specific works; satires criticize society 
    broadly.
    \item 2 Live Crew parodied ``Pretty Woman.''
    \item 2LC's use would be infringing but for a finding of fair 
    use.\footnote{Casebook p. 641.}
    \item \textbf{Purpose and character}:
    \begin{enumerate}
        \item Does it ``supersede the objects'' of the original or add 
        something new?
        \item The more transformative, the less important are the other 
        factors, including commercialism.
        \item Parody needs to mimic the original.\footnote{Casebook p. 643.}
    \end{enumerate}
    \item \textbf{Nature of the copyrighted work}:
    \begin{enumerate}
        \item ``~.~.~.~some works are closer to the core of intended copyright 
        protection than others''--e.g., expressive works, as opposed to 
        factual works.
        \item This factor is not ``likely to help much in separating the fair 
        use sheep from the infringing goats'' because parodies almost always 
        copy publicly known, expressive works.\footnote{Casebook p. 645.}
    \end{enumerate}
    \item \textbf{Amount and substantiality}:
    \begin{enumerate}
        \item ``~.~.~.~extent of permissible copying varies with the purpose 
        and character of the use.''\footnote{Casebook p. 645.}
        \item Parodies require the original to be recognizable. This parody 
        added something new in addition, and it took no more than what was 
        necessary.\footnote{Casebook p. 646.}
    \end{enumerate}
    \item \textbf{Effect on the market}:
    \begin{enumerate}
        \item Parodies are not substitutes for the original.
        \item The district court failed to address the possible effect of the 
        parody on derivative rap markets (e.g., if the copyright holder wanted 
        to create its own rap parody of ``Pretty Woman''). Remanded for 
        factfinding.
    \end{enumerate}
\end{enumerate}

\paragraph{Remixes: \emph{Bill Graham Archives v. Dorling Kindersley Ltd.}}
~\\\\
Remixes can copy entire works if the use is transformative.

\begin{enumerate}
    \item DK published a coffee table book on the Grateful Dead that included 
    seven images of posters copyrighted by Bill Graham Archives.
    \item \textbf{Purpose and character}:
    \begin{enumerate}
        \item DK's use was transformative because its purpose (biography) was 
        different than the images' original purpose (promotion). DK published 
        the images at a smaller size and in collages. Finally, BGA's images 
        were a tiny fraction of DK's overall work. The commercial purpose of 
        DK's use was not dispositive---``DK does not seek to exploit the 
        images' expressive value for commercial gain.''\footnote{Casebook pp. 
        654--56.}
    \end{enumerate}
    \item \textbf{Nature of the copyrighted work}: although the original 
    images were expressive and thus entitled to strong copyright protections, 
    DK did not exploit the images' expressive value.\footnote{Casebook pp. 
    656--57.}
    \item \textbf{Amount and substantiality}: DK copied the images in their 
    entirety, but the reduced size was consistent with the transformative 
    use.\footnote{Casebook pp. 657--58.}
    \item \textbf{Effect upon the market}: DK's use did not harm BGA's license 
    market, even if DK had been willing to pay licensing fees to BGA and other 
    copyright holders.\footnote{Casebook pp. 658--59.}
\end{enumerate}
\newpage % TODO remove

\subsubsection{Other Defenses}

% TODO 670-682

\subsection{Digital Copyright Law}

\subsubsection{Digital Copyright Legislation}

% TODO 690-710

\subsubsection{Fair Use in Cyberspace}

% TODO 732-741

\subsection{International Copyright Law}

\begin{enumerate}
    \item Lack of formalities has led to nearly worldwide protection for most 
    works.\footnote{Casebook p. 742.}
\end{enumerate}

\subsubsection{Evolution of the International Copyright System and U.S. 
Participation}

\begin{enumerate}
    \item Berne, 1886: foreign authors get ``domestic treatment''---i.e., the 
    same rights as domestic authors.\footnote{Casebook p. 742.}
    \item The United States joined in 1989.\footnote{Casebook p. 743.}
\end{enumerate}

\subsubsection{International Copyright Treaties}

\paragraph{Berne Convention for the Protection of Literary and Artistic Works}

\begin{enumerate}
    \item \textbf{National treatment}: foreign authors from Berne member 
    nations receive the same protections as domestic 
    authors.\footnote{Casebook p. 744.}
    \item Minimum requirements:\footnote{Casebook pp. 744--45.}
    \begin{enumerate}
        \item Works covered.
        \item Limitations on formalities.
        \item Duration.
        \item Exclusive rights.
        \item Moral rights.
        \item Restoration of rights.
    \end{enumerate}
\end{enumerate}

\paragraph{Agreement on Trade-Related Aspects of Intellectual Property Rights 
(TRIPs)}

\begin{enumerate}
    \item Expands Berne, including more extensive enforcement obligations and 
    a new WTO dispute resolution process.\footnote{Casebook p. 746.}
    \item Expansions on Berne:\footnote{Casebook p. 746.}
    \begin{enumerate}
        \item Works covered.
        \item Exclusive rights.
        \item Exceptions.
    \end{enumerate}
\end{enumerate}

\subsubsection{Protection of U.S. Works Against Infringement Abroad}

\begin{enumerate}
    \item U.S. copyright law does not apply beyond the United States. 
    Copyright holders can only enforce their rights in nations that are 
    parties to multi-lateral treaties (Berne, TRIPs) or where the United 
    States has specific bilateral agreements.\footnote{Casebook p. 747.}
    \item Standing is the threshold question.\footnote{Casebook p. 747.}
    \item The extend of protections depend on applicable law. Local law also 
    determines applicability and remedies.\footnote{Casebook p. 748.}
    \item \textbf{Rule of the shorter term}: under Berne, if the term of 
    protection is different between the protecting country and the country of 
    origin, the shorter duration applies, unless the protection country 
    chooses otherwise.\footnote{Casebook p. 748.}
\end{enumerate}

\subsubsection{Protection of Foreign Works Against Infringement in the United 
States}

\begin{enumerate}
    \item The United States protects all copyrights originating in Berne and 
    TRIPs countries.
    \item The United States does not apply the rule of the shorter term. So, 
    if the term is shorter in the originating country, the work still enjoys 
    the full term of U.S. copyright protection.\footnote{Casebook p. 751.}
\end{enumerate}

\subsection{Enforcement and Remedies}

\subsubsection{Injunctions}

\begin{enumerate}
    \item Before 2006, courts routinely granted injunctive relief. This 
    approach has since been replaced by a ``searching, equitable balancing 
    framework.''\footnote{Casebook p. 752.}
    \item The \emph{eBay} court established the four-factor test for 
    determining whether injunctive relief is called for. The plaintiff must 
    show:\footnote{Casebook p. 753.}
    \begin{enumerate}
        \item Irreparable injury;
        \item The inadequacy of other remedies;
        \item Considering the balance of hardships between the plaintiff and 
        defendant, an injunction is warranted; and
        \item ``~.~.~.~that the public interest would not be 
        disserved~.~.~.''
    \end{enumerate}
\end{enumerate}

\subsubsection{Damages}

% TODO 755-761

\subsubsection{Attorney Fees}

% TODO 761
