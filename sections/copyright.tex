\section{Copyright}

\subsection{Introduction}

\subsubsection{17 U.S.C. \S\ 101: Definitions}

See supplement p. 248.

\subsubsection{History}

\begin{enumerate}
    \item Venice in the fifteenth century recognized exclusive rights in the 
    printing of particular books (as opposed to the technology of creating the 
    books).\footnote{Casebook p. 430.}
    \item \textbf{Statute of Anne} (1710): authors gain exclusive rights for 
    14 years, renewable for another 14 years, subject to registration, notice, 
    and deposit. The government could also set maximum 
    prices.\footnote{Casebook p. 431.}
    \item The continental approach emphasized moral rights, as opposed to the 
    English system of property rights.
    \item The original Copyright Act (1790) mirrored the Statute of Anne. By 
    the end of the nineteenth century, it had been amended to include a range 
    of new technologies (e.g., photography).\footnote{Casebook p. 432.}
    \item \textbf{1909 Act}: extended protection to ``all writings''; extended 
    term to 28 years plus a 28 year renewal.
    \item \textbf{1976 Act}: extended protection to anything ``fixed in a 
    tangible medium of expression''; extended term to life of the author plus 
    50 years; loosened notice and registration requirements; established 
    compulsory licensing regimes; and codified exceptions, including fair use. 
    The 1998 Act extended the term to life of the author plus seventy 
    years.\footnote{Casebook p. 433.}
    \item 1989: the United States joined the Berne convention, scaling back 
    formalities and restoring copyright for foreign works still under 
    protection in the source country.
    \item \textbf{DMCA} (1998): anticircumvention provisions and liability 
    protections for service providers for infringing acts of their 
    subscribers.\footnote{Casebook pp. 433--34.}
\end{enumerate}

\subsubsection{Overview}

\paragraph{Elements of a Protectable Copyright}

\begin{enumerate}
    \item \textbf{Copyrightable subject matter}: any literary or artistic 
    expression (but not the idea itself).
    \item \textbf{Threshold for protection}: must (1) have a ``modicum of 
    originality'' and (2) be fixed in a ``tangible medium of expression.''
    \item \textbf{Formalities}: notice is required for works published before 
    1989. Registration is required to sue for infringement. Deposit is 
    required for registration.
    \item \textbf{Authorship and ownership}: only the creator, transferee, or 
    employer can bring an infringement suit.
    \item \textbf{Duration of copyright}: life plus 70 years; or 95 years from 
    the publication of anonymous, pseudonymous, or for-hire works; or 120 
    years from the year of creation, whichever comes first.
    \item The Copyright Office does not assess validity (other than requiring 
    a modicum of creativity) or prior art.
    \begin{enumerate}
        \item TODO: does the copyright office ever reject applications?
    \end{enumerate}
    \item A copyright is protectable at the moment of 
    creation.\footnote{Casebook p. 434.}
\end{enumerate}

\paragraph{Ownership Rights}

\begin{enumerate}
    \item \textbf{Reproduction}: owner has the exclusive right to make copies. 
    She can sue for ``material'' and ``substantial'' copying.
    \item \textbf{Derivative works}: original owner has the exclusive right to 
    prepare derivative works (e.g., translations).
    \item \textbf{Distribution}: the owner controls sale and distribution, but 
    only for the first sale.
    \item \textbf{Performance and display}: owner has the right to control.
    \item \textbf{Anticircumvention}: prohibitions on bypassing technological 
    protections.
    \item \textbf{Moral rights}: visual artists have an attribution right and 
    a right to prevent ``intentional distortion, mutilation, or other 
    modification~.~.~.~''\footnote{Casebook p. 435.}
    \item Limitations: fair use, compulsory licensing.
    \item Copyright vs. patent: copyright holders can prevent copying and 
    certain uses (e.g., public performance), but they cannot prevent others 
    from making, using, or selling their creations.
\end{enumerate}

\subsubsection{Philosophy}

\begin{enumerate}
    \item Personhood, natural law.
    \item Predominant in the United States: utilitarian. The primary goal is 
    to enhance the public interest. The secondary goal is to reward 
    authors.\footnote{Casebook p. 436.}
    \item There is an increasing recognition of moral rights in the United 
    States.\footnote{Casebook p. 437.}
\end{enumerate}

\subsection{Requirements}

\subsubsection{Original Works of Authorship}

\paragraph{17 U.S.C. \S\ 102: Subject Matter of Copyright: In General}

\begin{enumerate}
    \item Original.
    \item ``~.~.~.~fixed in any tangible medium of expression~.~.~.~''
    \item A work gains copyright protection if it was independently created, 
    even if it's identical---see Learned Hand on Keats.\footnote{Casebook p. 
    439.} The originality requirement is very low; exceptions are slogans, 
    familiar symbols, etc.
\end{enumerate}

\paragraph{No Copyright for Facts: \emph{Feist Publications v. Rural Telephone 
Services}}
~\\\\Copyright requires originality. Facts and unoriginal creations cannot be 
copyrighted.

\begin{enumerate}
    \item Feist copied records from Rural's phonebook for its own phonebook. 
    Rural sued for infringement.
    \item Justice O'Connor:
    \begin{enumerate}
        \item Facts are not copyrightable. But compilations of facts generally 
        are.\footnote{Casebook p. 441.}
        \item Originality is ``the touchstone of copyright 
        protection'' because it is constitutionally and statutorily 
        mandated.\footnote{Casebook p. 443.}Originality requires (1) 
        independent creation and (2) a minimal degree of creativity.  
        Facts are not original. Therefore, facts are not copyrightable.
        \item Copyright of an entire work does not imply copyright of each 
        element.\footnote{Casebook p. 443.}
        \item ``Not all copying, however, is copyright 
        infringement.''\footnote{Casebook p. 443.} Infringement requires 
        copying of elements that are original. 
        \item Rural's alphabetical arrangement ``is not only unoriginal, it is 
        practically inevitable.''\footnote{Casebook p. 445.} Since Rural's 
        arrangement was unoriginal, it was not copyrightable. Therefore, 
        Feist's copying was non-infringing.
    \end{enumerate}
\end{enumerate}

\subsubsection{Fixation in a Tangible Medium of Expression}

\paragraph{H.R. Rep. No. 94-1476: On the Copyright Act of 1976}

\begin{enumerate}
    \item Any medium satisfies the fixation requirement.\footnote{Casebook p. 
    449.} Unfixed works may still be eligible for protection under State 
    common or statutory law, but federal protection requires 
    fixation.\footnote{Casebook p. 450.}
    \item Broadcasts are covered by a provision that allows protections for 
    works that are simultaneously recorded and transmitted.
    \item ``The two essential elements---\textbf{original work} and 
    \textbf{tangible object}---must merge through fixation in order to produce 
    subject matter copyrightable under the statute.''\footnote{Casebook p. 
    451.}
    \item Fixation is (1) a requirement for protection and (2) plays a role in 
    determining whether a defendant has infringed a copyright (because copies 
    are material objects in which a work is ``fixed'').\footnote{Casebook p. 
    551.}
    \item Do bootlegs infringe? They're based on non-fixed performances. The 
    Second Circuit upheld the anti-bootlegging criminal provisions under the 
    Commerce Clause because they do not create additional rights; however, the 
    implication is that the civil provisions are 
    unconstitutional.\footnote{Casebook pp. 451--52.}
    \item Why require fixation?\footnote{Casebook p. 452.}
    \begin{enumerate}
        \item If copyright is meant to protect communication, then it should 
        not apply to expressions that do not actually communicate.
        \item It's a practical requirement for litigation (cf. the statute of 
        frauds in contract law).
    \end{enumerate}
\end{enumerate}

\subsubsection{Formalities}

\paragraph{Notice}

\begin{enumerate}
    \item 1909 Act: failure to follow precise notice requirements resulted in 
    forfeiture.
    \item 1976 Act/Pre-Berne: copyright begins upon creation, not 
    publication. Notice was still required, but the requirements were 
    significantly looser.
    \item Post-Berne: completely eliminated the notice requirement, but 
    encouraged voluntary notice (e.g., by allowing the innocent infringement 
    defense if the work lacked proper notice).\footnote{Casebook p. 455.}
\end{enumerate}

\paragraph{Publication}

\begin{enumerate}
    \item 1909 Act: publication triggered copyright. Unpublished works could 
    be protected under state common law, or ``constructively'' published by 
    registration with the Copyright Office.
    \begin{enumerate}
        \item \emph{Divestive} publication: results in forfeiture of common 
        law copyright protection. 
        \item \emph{Investive} publication: resulting in forfeiture of federal 
        statutory protection if notice is inadequate.
    \end{enumerate}
    \item 1976 Act/Pre-Berne: creation triggered copyright, but still 
    determined when notice was required.\footnote{Casebook pp. 456--57.}
    \item Post-Berne: after March 1, 1989, publication no longer determines 
    validity, though it still has relevance in several areas---see casebook p. 
    457.
\end{enumerate}

\paragraph{Registration}

\begin{enumerate}
    \item Has always been voluntary.
    \item 1909 Act: term was 28 years, plus a 28 year renewal, but only if the 
    owner registered.
    \item 1976 Act/Pre-Berne: abolished renewal, but created new incentives 
    for registration.\footnote{Casebook p. 458.}
    \item Post-Berne: no registration requirement for foreign works, but 
    Congress preserved the registration requirement for domestic works for 
    bringing infringement suits.
\end{enumerate}

\paragraph{Deposit}

\begin{enumerate}
    \item After 1976, still mandatory, but failure results only in a fine, not 
    invalidity or forfeiture of the right to bring suit.\footnote{Casebook p. 
    459.}
\end{enumerate}

\paragraph{Restoration of Foreign Copyrighted Works}

\begin{enumerate}
    \item Post-Berne, many foreign works that had entered the public domain in 
    the United States (e.g., J.R.R. Tolkien's works) were restored to 
    copyright. The Supreme Court affirmed in \emph{Golan v. Holder}, holding 
    that the IP clause allows Congress to ``induc[e] the dissemination of 
    works~.~.~.~to promote science.''\footnote{Casebook p. 460.}
\end{enumerate}

\subsection{Copyrightable Subject Matter}

% TODO 17 usc 102(b) [?]

\subsubsection{Limitations on Copyrightability: Distinguishing Function and 
Expression}

% TODO 461-479
% TODO 482-97

\subsubsection{The Domain and Scope of Copyright Protection}

% TODO 497-506

\subsection{Ownership and Duration}

% TODO 17 usc 201c, 301-05

\subsubsection{Initial Ownership of Copyrights}

% TODO 506-526

\subsubsection{Duration and Renewal}

% TODO 526-31

\subsubsection{Division, Transfer, and Reclaiming of Copyrights}

% TODO 531-36

\subsection{Traditional Rights of Copyright Owners}

% TODO 17 usc 106

\subsubsection{The Right to Make Copies}

% TODO 537-73

\subsubsection{The Right to Prepare Derivative Works}

% TODO 573-83

\subsubsection{The Distribution Right}

% TODO 583-88

\subsubsection{Public Performance and Display Rights}

% TODO 588-94

\subsubsection{Moral Rights}

% TODO 594-98

\subsection{Indirect Liability}

% TODO 598-609

\subsection{Defenses}

% TODO 35 usc 107

\subsubsection{Fair Use}

% TODO 623-659

% TODO 670-682

\subsection{Digital Copyright Law}

\subsubsection{Digital Copyright Legislation}

% TODO 690-710

\subsubsection{Fair Use in Cyberspace}

% TODO 732-741

\subsection{International Copyright Law}

\subsubsection{Evolution of the International Copyright System and U.S. 
Participation}

% TODO 742-44

\subsubsection{International Copyright Treaties}

% TODO 744-47

\subsubsection{Protection of U.S. Works Against Infringement Abroad}

% TODO 747-51

\subsubsection{Protection of Foreign Works Against Infringement in the United 
States}

% TODO 751

\subsection{Enforcement and Remedies}

\subsubsection{Injunctions}

% TODO 751-55

\subsubsection{Damages}

% TODO 755-761

\subsubsection{Attorney Fees}

% TODO 761
