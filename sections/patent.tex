\section{Patent}

\subsection{Introduction}

\subsubsection{Basic Patent Procedure}

\begin{enumerate}
    \item File for a patent with the PTO.
    \item An examiner is assigned to the application. The review process takes 
    two to four years on average.
    \item Applicants can appeal adverse rulings of the PTO Board of Appeals to a 
    federal appellate court.
    \begin{enumerate}
        \item Pre-1982: Court of Customs and Patent Appeals.
        \item 1982 and on: U.S. Court of Appeals for the Federal Circuit.
    \end{enumerate}
\end{enumerate}

\subsubsection{History}

\begin{enumerate}
    \item First patent system: Venice, fifteenth century.\footnote{Casebook p. 
    123.}
    \item 
    \item First U.S. patent statute: 1790.
    \item 1952: Patent Act.
    \item 1982: Federal Courts Improvement Act. Created the Federal Circuit. 
    Aimed to unify patent doctrine, and had the effect of strengthening patent 
    protections.\footnote{Casebook p. 127.}
\end{enumerate}

\subsubsection{Overview of the Patent Laws}

\paragraph{Requirements for Patentability}

\begin{enumerate}
    \item \textbf{Subject matter}.\footnote{Casebook pp. 128--29.}
    \item \textbf{Novelty}: not preceded in identical form by prior public art. 
    35 U.S.C. \S\ 102.
    \item \textbf{Utility}: it must be useful (a low barrier). 35 U.S.C. \S\ 
    101.
    \item \textbf{Nonobviousness}: a nontrivial extension of what was known (the 
    most important requirement). 35 U.S.C. \S\ 103.
    \item \textbf{Disclosure and enablement}: disclosed in a way that enables a 
    person having ordinary skill in the art (PHOSITA) to make and use the 
    invention. 35 U.S.C. \S\ 112.
\end{enumerate}

\paragraph{Rights Conferred by a Patent}

\begin{enumerate}
    \item \textbf{Claims}: boundaries of the property right the patent 
    confers.\footnote{Casebook p. 129.}
    \item \textbf{Specification}: describes the invention.
    \item Patents confer exclusive rights for a term of years.\footnote{Casebook 
    p. 130.}
    \begin{enumerate}
        \item Post--GATT-TRIPS (1995): 20 years from the filing date.
        \item Previously: 17 years from the date the USPTO issued the patent.
    \end{enumerate}
    \item Patent rights are negative, i.e., they let you prevent others from 
    acting.
\end{enumerate}

\subsubsection{Theories of Patent Law}

\begin{enumerate}
    \item Patents give a market-driven incentive to invest in 
    innovation.\footnote{Casebook p. 131.}
\end{enumerate}

\subsection{Elements of Patentability}

\subsubsection{Patentable Subject Matter}

\paragraph{35 U.S.C. \S\ 101: Inventions Patentable}

\begin{enumerate}
    \item ``Whoever invents or discovers any new and useful process, machine, 
    manufacture, or composition of matter, or any new and useful improvement 
    thereof, may obtain a patent therefor, subject to the conditions and 
    requirements of this title.''
\end{enumerate}

\paragraph{Compositions of Matter: \emph{Diamond v. Chakrabarty}}

\begin{enumerate}
    \item Chakrabarty invented a genetically engineered version of pseudomonas 
    to clean up toxic spills.
    \item Three claims:\footnote{Casebook p. 129.}
    \begin{enumerate}
        \item The \textbf{process} of creating the bacteria.
        \item An ``inoculum,'' including a carrier (a \textbf{combination} 
        claim).
        \item ``~.~.~.~the bacteria themselves.''
    \end{enumerate}
    \item Patentable subject matter comprises ``anything under the sun that is 
    made by [humans].''
    \item Things that are \emph{not} patentable:
    \begin{enumerate}
        \item Laws of nature---e.g., Newton's law of gravitation.
        \item Physical phenomena---e.g., naturally occurring plants or minerals.
        \item Abstract ideas---e.g., E=mc\textsuperscript{2}.
    \end{enumerate}
    \item Chakrabarty's bacterium was not naturally occurring. Rather, it was 
    \textbf{``a product of human ingenuity~.~.~.~.''}
\end{enumerate}

\paragraph{Abstract Ideas: \emph{Bilski v. Kappos}} 
~\\\\
Rather than adopt categorical rules, the Supreme Court held narrowly that the 
hedging technique in question was not patentable because it was an abstract 
idea.

The Federal Circuit's machine-or-transformation test is \emph{not} the sole 
test for what counts as a ``process.'' Business methods \emph{are} protectable 
today.\footnote{Merges, \emph{Six Impossible Patents before Breakfast}, 
\url{http://scholarship.law.berkeley.edu/cgi/viewcontent.cgi?article=1162&context=facpubs}.}

Unpatentable abstractions include things like risk hedging or algorithms. But 
the specific applications of these abstract ideas may be patentable.

\begin{enumerate}
    \item Bilski patented a commodity price hedging technique known as the fixed 
    price contract. He called it a ``method.'' He also made a dependent claim, 
    which applied the general method to energy markets.
    \item Held: this process did not physically transform anything (compared to, 
    e.g., a chemical process).
    \item Federal circuit (which the Supreme Court rejected):
    \begin{enumerate}
        \item 
        \item \textbf{Machine or transformation test}: to be patentable, a process 
        must either:
        \begin{enumerate}
            \item Transform an article to a different state or thing, and/or
            \item Be ``tied to a particular machine.''
        \end{enumerate}
    \end{enumerate}
    \item Supreme Court:
    \begin{enumerate}
        \item Four categories of patentable subject matter:
        \begin{enumerate}
            \item Process.
            \item Machine. 
            \item Manufacture.
            \item Composition of matter.
        \end{enumerate}
        \item Three exceptions (derived from the requirement that patentable 
        inventions be new and useful):
        \begin{enumerate}
            \item Laws of nature.
            \item Physical phenomena.
            \item Abstract ideas.
        \end{enumerate}
    \end{enumerate}
    \item Justice Kennedy, concurring: the machine-or-transformation test might 
    be useful for Industrial Age inventions, but it may not be well suited to 
    the Information Age.
    \item Justice Stevens, dissenting: business methods are categorically 
    unpatentable as processes under \S\ 101.
\end{enumerate}

\paragraph{Preemption: \emph{Mayo v. Prometheus}}

\begin{enumerate}
    \item ``The Court has repeatedly emphasized~.~.~.~a concern that patent 
    law not inhibit further discovery by \textbf{improperly tying up} the 
    future use of laws of nature.
    \item Preemption = overbreadth.
    \item Purpose of \S\ 101: preserve the ``basic tools'' of scientific 
    discovery for common use.
\end{enumerate}

\paragraph{Genes: \emph{AMP v. Myriad Genetics, Inc.}}

\begin{enumerate}
    \item Are human DNA sequences patentable?
    \item Is isolating DNA a inventive act that gives patent rights to the one 
    who isolated it?
    \item Supreme Court:
    \begin{enumerate}
        \item Naturally occurring DNA segments are not patentable, even if the 
        isolation involved significant effort. Cf. \emph{Chakrabarty}.
        \item But non--naturally occurring cDNA is patentable because it's the 
        product of human invention.
    \end{enumerate}
\end{enumerate}

\subsubsection{Utility}

\begin{enumerate}
    \item \S\ 101: ``Whoever invents or discovers any new \textbf{and useful} 
    process~.~.~.~''
    \item No patents for:
    \begin{enumerate}
        \item \textbf{No known utility}, e.g., perpetual motion machines.
        \item \textbf{Malicious utility}.
    \end{enumerate}
    \item What about new chemical compounds with no known uses?
\end{enumerate}

\paragraph{Promising \emph{Experimental} Results---Not Patentable: 
\emph{Brenner v.  Manson}}

\begin{enumerate}
    \item Patents cannot be granted for chemical compounds with no known uses. 
    ``~.~.~.~a patent is not a hunting license. It is not a reward for the 
    search, but compensation for its successful 
    conclusion.''\footnote{Casebook p. 179.}
\end{enumerate}

\paragraph{Promising \emph{Clinical} Results---Patentable: \emph{In re Brana}}

\begin{enumerate}
    \item Results from mice testing suggest efficacy in humans, so they pass 
    the utility threshold.\footnote{Casebook p. 181.}
\end{enumerate}

\paragraph{Timing: \emph{In re Fisher}}

\begin{enumerate}
    \item Applicant must show utility as of the filing date.\footnote{Casebook 
    p. 182--85.}
\end{enumerate}

\paragraph{Patent Office Utility Guidelines}

\begin{enumerate}
    \item It has to actually work.\footnote{Casebook p. 185--87.}
\end{enumerate}

\paragraph{Moral Utility}

\begin{enumerate}
    \item Historically, often used to deny gambling patents.\footnote{Casebook 
    p. 187--88.}
\end{enumerate}

\paragraph{Designed to Deceive: \emph{Juicy Whip, Inc. v. Orange Bang, Inc.}} 

\begin{enumerate}
    \item Deceptive inventions are patentable.
    \item ``All that the law requires is, that the invention should not be 
    frivolous or injurious to the well-being, good policy, or sound morals of 
    society. The word "useful," therefore, is incorporated into the act in 
    contradistinction to mischievous or immoral.'' \emph{Lowell v. Lewis}, 15 
    F. Cas. 1018, 1019 (CC Mass. 1817).
\end{enumerate}

\subsubsection{Written Description and Enablement}

\paragraph{35 U.S.C. \S\ 112}

\begin{enumerate}
    \item Must contain a \textbf{written description} that can 
    \textbf{``enable any person skilled in the art''} to make it.
\end{enumerate}

\paragraph{Overly Broad Claims and Enablement: \emph{The Incandescent Lamp 
Patent}}

\begin{enumerate}
    \item Sawyer and Man had a patent on a lightbulb involving ``fibrous or 
    textile material.'' They tried to assert it against Edison, who was using 
    bamboo in his own bulbs.\footnote{Casebook p. 196 ff.}
    \item S\&M tried to monopolize the use of ``all fibrous or textile 
    material'' in lightbulbs. But they had only tried a few types. They hadn't 
    discovered ``a quality common to them all,'' so their claim was too broad.  
    Their claim would have included over 6,000 types of plants and products, 
    but really they had only tested a few.
    \item The S\&M patent only enabled the use of the materials they had 
    tested (carbonized paper, etc.).
\end{enumerate}

\paragraph{Written Description: \emph{The Gentry Gallery, Inc. v. 
The Berkline Corp.}}

\begin{enumerate}
    \item Claims are limited to the written description.
    \item ``~.~.~.~the scope of the right to exclude may be limited by a 
    narrow disclosure.''\footnote{Casebook p. 208.} Gentry's patent only 
    described the placement of recliner controls inside the console, so it 
    could not prevent its competitors from placing controls elsewhere.
\end{enumerate}

\paragraph{\emph{Ariad Pharmaceuticals, Inc. v. Eli Lilly \& Co.}} % TODO 211-22

\begin{enumerate}
    \item The written description requirement requires the disclosure of 
    actual structures and actual working examples in order to justify the 
    broad scope of a claim.
    \item \S\ 112 excludes inventions from patent protection due to inadequate 
    disclosure. Compare to \S\ 101, which excludes entire \emph{categories} 
    for overbreadth.
\end{enumerate}

\subsubsection{Novelty and Statutory Bars}

\paragraph{Novelty}

\begin{enumerate}
    \item Old law (1952 Act): \textbf{first to invent}. Defines novelty from 
    the \emph{date of first invention}.
    \item New law (AIA, 2011): \textbf{first to file}. Novelty is measured as 
    of the \emph{date of the patent application}. Takes effect for patents 
    filed on or after March 16, 2013.\footnote{Casebook p. 226.}
    \item \textbf{Novelty}: \emph{newness}. Was it new compared to the prior 
    art? 35 U.S.C. \S\ 102(a).
    \item \textbf{Statutory bar}: \emph{timeliness}. A bar based on too long a 
    delay in seeking patent protection. Did the inventor file soon enough? \S\ 
    102(b).
\end{enumerate}

\paragraph{Main Issues under Novelty}

\begin{enumerate}
    \item \textbf{Reference}: a single piece of prior art.
    \item \textbf{Timing}: what does the prior art include?
    \item \textbf{Similarity}: how similar must prior art be to anticipate 
    (i.e., destroy) a patent?
    \item \textbf{Categories of prior art}---35 U.S.C. \S\ 102(a)(1):
    \begin{enumerate}
        \item Patented.
        \item Described in a printed publication.
        \item In public use, on sale, or otherwise available to the public.
    \end{enumerate}
\end{enumerate}

\paragraph{Statutory Bars}

\begin{enumerate}
    \item Under \S\ 102(b), an inventor loses the right to patent if, more 
    than one year prior to filing, the invention was:
    \begin{enumerate}
        \item Patented by another elsewhere.
        \item Patented by the applicant in a foreign country.
        \item Described in a printed publication anywhere.
        \item In public use in the United States.
        \item On sale in the U.S.
    \end{enumerate}
\end{enumerate}

\paragraph{35 U.S.C. \S\ 102 (1952): Conditions for patentability; novelty and 
loss of right to patent}

\begin{enumerate}
    \item (a) Novelty.
    \item (b) Statutory Bars.
    \item (c), (d): (omitted).
    \item (e) Secret Prior Art: Previously filed applications.
    \item (f) Derivation.
    \item (g) (inteference proceedings).
\end{enumerate}

\paragraph{Novelty: \emph{Rosaire v. National Lead Co.}}
~\\\\
Earlier use, even if not publicly known, establishes prior art.

\begin{enumerate}
    \item 1936: Rosaire and Horovitz patented a method for analyzing soil to 
    detect nearby oil.\footnote{Casebook p. 228.}
    \item National Lead argued that the patent was invalid and that there had 
    been no infringement.
    \item Rosaire admitted that Teplitz first invented the method, but argued 
    that Teplitz did not publish his ideas and that they were experimental.
    \item Held: Teplitz use was actual, not experimental, and that publication 
    was not required as long as Teplitz did the work openly.
\end{enumerate}

\paragraph{\emph{Inherency Doctrine}} 

\begin{enumerate}
    \item If somebody accidentally comes up with an invention, is that 
    invention prior art if somebody later deliberately invents the same thing? 
    Courts are divided. Newer cases have focused on whether the earlier 
    invention provided a public benefit.\footnote{Casebook p. 233.}
\end{enumerate}

\paragraph{Statutory Bars---Defining ``Publication'': \emph{In Re Hall}}
~\\\\
Publication of a single copy in a foreign university library can satisfy 
the publication limitation.

\begin{itemize}
    \item Hall applied for a chemical patent. The PTO rejected it because the 
    same process had been published in a German doctoral dissertation more than 
    a year earlier.\footnote{Casebook p. 234.}
    \item On appeal, the court held that publication in a university library was 
    ``sufficiently accessible, at least to the public interested in the art, so 
    that such a one by examining the reference could make the claimed invention 
    without further research or experimentation.''\footnote{Casebook p. 235.} 
    Affirmed.
\end{itemize}

\paragraph{Statutory Bars---Public Use: \emph{Egbert v. Lippmann}}
~\\\\
Hidden use (i.e., ``non-informing public use'') can still be public use. The 
inventor here slept on his rights for 11 years.

\begin{enumerate}
    \item Barnes invented a new type of corset spring in 1855 but did not apply 
    for a patent until 1866. In the meantime, he had given at least two pairs to 
    Frances, who later became his wife and the assignee of the 
    patent.\footnote{Casebook pp. 237--39.}
    \item Frances sued for patent infringement. The question was whether the 
    invention had been in public use for more than two years\footnote{Now the 
    period is one year.} before the patent application.
    \item The Court (Justice Woods) held that it was a public use because (1) a 
    single use is enough, (2) giving the invention to one person is enough, and 
    (3) selling a hidden component of a public machine is still public use.
    \item Justice Miller, dissenting: if a hidden corset spring is a public use, 
    how can anything be a private use?\footnote{Casebook p. 240.}
\end{enumerate}

\paragraph{Experimental Use Exception: \emph{City of Elizabeth v. Pavement 
Company}}
~\\\\
Experimental use is not public use.

\begin{enumerate}
    \item Nicholson was issued a patent for wooden pavement in 1854. He had been 
    testing the design for the previous six years on a public street in 
    Boston.\footnote{Casebook pp. 243--44.}
    \item Nicholson sued the City of Elizabeth for infringement. Elizabeth 
    argued that the patent was invalid because the invention was in public use.
    \item The court held that experimental use is an exception to the public use 
    restriction. Here, Nicholson intended the use to be experimental, and the 
    public's use was only ``incidental.''\footnote{Casebook pp. 245--46.}
\end{enumerate}

\paragraph{Priority under the 1952 Act: 35 U.S.C. \S\ 102: Novelty and Loss of 
Right}

\begin{enumerate}
    \item Priority is about being \textbf{first}. Novelty is about being 
    \textbf{new}.
    \item A person gets a patent unless:
    \begin{enumerate}
        \item (g)(1) during the course of an interference, another inventor (from 
        any WTO country) establishes that he got there first, or
        \item (g)(2) another U.S. inventor made the invention first.
    \end{enumerate}
    \item ``~.~.~.~priority generally goes to the first inventor to (1) reduce 
    an invention to practice, without (2) abandoning the 
    invention.''\footnote{Casebook p. 248.} An exception arises if the inventor 
    was not the first to reduce to practice but was reasonably diligent in doing 
    so.
    \item AIA sweeps away this complexity in favor of a first-to-file 
    system---see below. 
\end{enumerate}

\paragraph{Priority and the Reasonable Diligence Exception: \emph{Griffith v. 
Kanamaru}}
~\\\\
Delays to make an invention more marketable do not fall within the reasonable 
diligence exception. In other words, you don't get priority if you 
unreasonably delayed reduction to practice.

\begin{enumerate}
    \item Griffith and Kanamaru both invented the same compound.
    \item Timeline:
    \begin{enumerate}
        \item June 30, 1981: Griffith established conception.
        \item November 17, 1982: Kanamaru filed for a patent.
        \item June 15, 1983 to September 13. 1983: Griffith was inactive.
        \item January 11, 1984: Griffith reduced to practice.
    \end{enumerate}
    \item Griffith argued that he was reasonably diligent in reducing the 
    invention to practice while (1) he sought funding, as a form of peer review, 
    and (2) awaited the matriculation of a particular grad student, even though 
    there were other students who could have done the work.\footnote{Casebook p. 
    250.}
    \item The court held that the reasonable diligence exception applies to 
    ``everyday problems and limitations'' (e.g., illness), but held that delays 
    in order to ``refine an invention to the most marketable and profitable form 
    have not been accepted as sufficient excuses for 
    inactivity.''\footnote{Casebook p. 250--51.}
    \item Held (affirming the PTO): Griffith failed to establish a prime facie 
    case of reasonable diligence or a sufficient excuse for 
    inactivity.\footnote{Casebook p. 251.}
\end{enumerate}

\paragraph{Prior User Rights}

\begin{enumerate}
    \item Under the 1952 act, a patent holder can exclude another who 
    independently developed the same invention from using it.
    \item The AIA addressed this problem by granting ``prior user rights'' to 
    non-patentees who were using the invention before someone else patented it, 
    with some limitations (e.g., non-transferability).\footnote{Casebook p. 252.}
\end{enumerate}

\paragraph{Novelty, Priority, and Statutory Bars under the AIA}

\begin{enumerate}
    \item The AIA introduced \textbf{three main changes regarding novelty and 
    priority}:\footnote{Casebook p. 253.}
    \begin{enumerate}
        \item The critical date is when a patent is first filed.
        \item Relevant prior art consists of all references available prior to 
        the filing date, with a one-year grace period.
        \item Priority contests will be determined by filing dates.
    \end{enumerate}
    \item \textbf{Structure}:
    \begin{enumerate}
        \item 102(a) Novelty; Prior Art: patent granted unless---
        \begin{enumerate}
            \item (1) Published before filing date.
            \item (2) Described in another previously filed patent.
            \item 
        \end{enumerate}
        \item (b) Exceptions---
        \begin{enumerate}
            \item (1) Disclosure made less than one year before filing is not 
            prior art if---
            \begin{enumerate}
                \item Made by the inventor.
                \item Based on information from the inventor.
            \end{enumerate}
            \item (2) Disclosure is not prior art under (a)(2) if---
            \begin{enumerate}
                \item Subject matter was obtained from the inventor.
                \item Subject matter had been disclosed by the inventor.
                \item Subject matter disclosed and invention were owned by the 
                same person.
            \end{enumerate}
        \end{enumerate}
    \end{enumerate}
    \item No \textbf{geographic restrictions} on prior 
    art.\footnote{Casebook p. 254.}
    \item \textbf{Novelty vs. priority}:
    \begin{enumerate}
        \item Novelty: inventor vs. inventor. Who invented it first?
        \item Priority: inventor vs. prior art. Is it new?
        \item The AIA only asks \textbf{who filed first} (including the one 
        year grace period).\footnote{Casebook pp. 244--45.}
    \end{enumerate}
\end{enumerate}

\paragraph{Grace period under the AIA}

\begin{enumerate}
    \item Generally, under the AIA, an application has to be filed before a 
    prior art event.
    \item But there's a grace period exception. A disclosure (i.e., any prior 
    art reference) is eligible for the grace period if made by the inventor or 
    if the inventor himself previously made a public 
    disclosure.\footnote{Casebook p. 255.}
    \begin{enumerate}
        \item Under the 1952 act, independent third-party disclosure 
        \emph{\textbf{did}} create a grace period. Not so under the 
        AIA.\footnote{Casebook p. 256.}
        \item ``In public use'' or ``on sale'' (AIA \S\ 102) can refer to 
        confidential activity. For instance, an inventor's \textbf{non-informing 
        public use} is a disclosure (but not third-party non-informing 
        uses).\footnote{Casebook p. 257.}
    \end{enumerate}
\end{enumerate}

\subsubsection{Nonobviousness}

\paragraph{35 U.S.C. \S\ 103}
~\\\\
\begin{enumerate}
    \item ``A patent for a claimed invention may not be obtained, 
    notwithstanding that the claimed invention is not identically disclosed as 
    set forth in section 102, if the differences between the claimed invention 
    and the prior art are such that the claimed invention as a whole 
    \textbf{would have been obvious} before the effective filing date of the 
    claimed invention \textbf{to a person having ordinary skill in the art} to 
    which the claimed invention pertains. Patentability shall not be negated 
    by the manner in which the invention was made.''
    \item Policy: avoid a profusion of paltry patents.
    \item Another policy: obvious patents may compromise the incentives to 
    make nonobvious inventions.
\end{enumerate}

\paragraph{Testing for Obviousness: \emph{Graham v. John Deere Co.}}
~\\\\
\S\ 103 requires an ``inventive leap.''\footnote{Casebook p. 269.} Minor obvious 
improvements don't qualify.

\begin{enumerate}
    \item First case interpreting \S\ 103.
    \item History of the patent system: Jefferson; \enquote{limited monopoly 
    might serve to incite \enquote{ingenuity}~.~.~.~.}\footnote{Casebook p. 
    259--61.}
    \item First appearance of nonobviousness: \emph{Hotchkiss v. Greenwood}, 
    1851. Codified in \S\ 103.\footnote{Casebook p. 261--62.}
    \item \S\ 103 test:\footnote{Casebook pp. 262--63.}
    \begin{enumerate}
        \item Determine scope and content of the prior art.
        \item Determine differences between prior art and the claims.
        \item Determine the level of ordinary skill in the art.
        \item Evaluate obviousness.
        \item Secondary considerations: commercial success, long felt need, 
        failure of others.
    \end{enumerate}
    \item 1950: Graham got a tractor patent ('811).
    \item 1953: Graham modified the design and got a second patent ('798), at 
    issue here. Graham argued that John Deere infringed the '798 patent.
    \item Another patent, held by Glencoe, had all the elements of the '798 
    patent.
    \item The Court found that the improvement in the '798 patent was obvious 
    and therefore invalid. It also found that the Glencoe patent was 
    mechanically identical.\footnote{Casebook p. 268.}
\end{enumerate}

\paragraph{\emph{KSR International Co. v. Teleflex Inc.}}
~\\\\
The Federal Circuit's ``teaching, suggestion, or motivation'' (TSM) test is 
too restrictive. The standard for nonobviousness is flexible.

\begin{enumerate}
    \item Engelgau patented a pedal assembly. Teleflex held the exclusive 
    license.\footnote{Casebook p. 269.}
    \item Teleflex sued KSR for infringement. KSR countered that the relevant 
    claim was invalid because its subject matter was obvious.
    \item The Federal Circuit had developed the ``teaching, suggestion, or 
    motivation'' (TSM) test, which held that a patent was nonobvious only if 
    ``some motivation or suggestion to combine the prior art teachings'' could 
    be found in prior art, the nature of the problem, or the knowledge of a 
    person having ordinary skill in the art.\footnote{Casebook p. 270.}
    \item The PTO rejected one of Engelbau's other claims on the basis of 
    nonobviousness. It cited two other patents: an adjustable pedal (Redding) 
    and a method for mounting an electronic sensor on a pedal's support 
    structure (Smith). The PTO rejected the claim that simply put the two 
    together, but it allowed a variation on Redding using a fixed pivot 
    point.\footnote{Casebook p. 272--73.}
    \item The District Court found that the claim was obvious. It granted 
    summary judgment in favor of KSR. The Federal Circuit reversed because the 
    District Court had improperly applied its TSM test.\footnote{Casebook p. 
    273.}
    \item The Supreme Court here found that the TSM test was too formalistic and 
    restrictive. It argued for a broader, more flexible test. It found that 
    Engelbau patent was obvious because any person with ordinary skill in the 
    art would have thought to combine the other patents. 
    Reversed.\footnote{Casebook pp. 274--77.}
\end{enumerate}

\paragraph{Nonobviousness and the AIA}

\begin{enumerate}
    \item Shifts the time for determining obviousness from the invention date 
    to the filing date.\footnote{Casebook p. 289.}
\end{enumerate}

\subsection{Administrative Procedures at the PTO}

\subsubsection{The AIA's New Administrative Procedures}

\begin{enumerate}
    \item The AIA created five new procedures:\footnote{Casebook p. 290.}
    \begin{enumerate}
        \item PGR.
        \item IPR.
        \item Supplemental examination.
        \item Transitional post-grant review for business method patents.
        \item Derivation proceedings.
    \end{enumerate}
\end{enumerate}

\subsubsection{Post-Grant Review (PGR)}

\begin{enumerate}
    \item Anyone can challenge the validity of a patent within nine months of 
    issuance.\footnote{Casebook p. 290.}
    \item In effect, a mini-trial, as an alternative to district court 
    proceedings.
    \item In responding to a PGR complaint, the patent holder cannot introduce 
    new evidence of patentability.
    \item Appeals go to the Federal Circuit.\footnote{Casebook p. 292.}
    \item Can't file both a district court claim and a PGR 
    claim.\footnote{Casebook p. 293.}
    \item Issue preclusion from PGR applies to later district court 
    claims.
\end{enumerate}

\subsubsection{Inter Partes Review (IPR)}

\begin{enumerate}
    \item Anyone can request a review of an issued patent. The one requesting 
    review has to show that he'll likely prevail on at least one 
    claim.\footnote{Casebook p. 293.}
    \item Only patents and printed publications can be cited.\footnote{Casebook 
    p. 293--94.}
    \item Estoppel applies.
    \item Can only follow a PGR decision or the nine-month PGR window.
    \item Why would someone choose IPR over PGR?
\end{enumerate}

\subsubsection{Derivation Proceeding}

\begin{enumerate}
    \item Responds to the fear that some might try to steal an invention and 
    file it first.\footnote{Casebook p. 295.}
    \item This is a ``special administrative proceeding to sort out claims that 
    one applicant stole or `derived' an invention of another.''
    \item Must be brought within one year of patent publication (or unpublished 
    patent issuance).
\end{enumerate}

\subsection{Infringement}

\subsubsection{Claim Interpretation}

\begin{enumerate}
    \item Claims define the boundaries of the owner's property 
    right.\footnote{Casebook p. 295.}
    \item To a businessman, the claim is the shelf space, the bottom line.
    \item \textbf{Broader claims are more valuable but easier to invalidate on 
    the basis of prior art.} ``[T]he stronger a patent the weaker it is and the 
    weaker a patent the stronger it is.''\footnote{Casebook p. 296.}
\end{enumerate}

\paragraph{Role of Judge and Jury and the Standard of Appellate Review}

\begin{enumerate}
    \item Is claim interpretation a matter of law (for judges) or fact (for 
    juries)?\footnote{Casebook p. 296.}
    \item \emph{Markman v. Westview Instruments} (Fed. Cir. 1995): although 
    infringement claims must be tried by a jury, construction of patent claims 
    is best left to judges, since they have ``training in exegesis'' and are 
    likely to construct claims more uniformly.\footnote{Casebook pp. 296--97.}
    \item The Federal Circuit has held that it will review district court 
    findings of patent construction de novo (although it may be moving away from 
    the position that claim construction is purely a matter of 
    law).\footnote{Casebook p. 298.}
\end{enumerate}

\paragraph{Standards for Construing Claims; Sources of Construction: 
\emph{Phillips v. AWH Corporation}}

\begin{enumerate}
    \item The dispute centered on the meaning of ``baffle'' in Phillips's 
    patent for vandalism-resistant walls. The Federal Circuit initially found 
    that the use of the term in the patent meant that baffles could not be 
    baffles if they were attached at 90 degree angles (because one of their 
    purposes was to deflect projectiles, like bullets, and they can't deflect if 
    they're mounted at 90 degrees).\footnote{Casebook pp. 298--99.}
    \item In an en banc rehearing, the Federal Circuit here reversed.
    \item The key issue was whether the court rely on the patent specification 
    to determine the meaning of the terms in the claim, or whether it could look 
    to extrinsic sources, like dictionaries or treatises.\footnote{Casebook p. 
    300.}
    \item \emph{Texas Digital}: courts can seek definitions in extrinsic 
    sources, as long as the definitions are consistent with the usage in the 
    patent itself.\footnote{Casebook p. 301.}
    \item \emph{Markman} (holding that patent construction is a matter of law, 
    i.e., for judges, not juries, to decide) is reaffirmed.\footnote{Casebook p. 
    305.}
    \item ``Baffles'' doesn't just mean projectile-deflecting baffles; it can 
    also mean baffles mounted at 90 degrees.\footnote{Casebook p. 306.}
    \item Judge Mayer, dissenting: claim construction absolutely has a factual 
    component. The Federal Circuit has failed to develop a consistent standard. 
    By shifting its standard and reviewing each district court determination 
    novo, it wastes time and resources.\footnote{Casebook pp. 308--09.}
\end{enumerate}

\subsubsection{Literal Infringement: \emph{Larami Corp. v. Amron}}

\begin{enumerate}
    \item Amron sued Larami, claiming that Larami's SUPER SOAKERS infringed its 
    patents. According to Larami's motion for summary judgment, six of 35 claims 
    were in dispute, all regarding the hand pump mechanism.\footnote{Casebook 
    pp. 323--24.}
    \item There are two ways to establish infringement:
    \begin{enumerate}
        \item \textbf{Literal infringement},
        \item \textbf{The doctrine of equivalents}.
        \item The patent owner must ``show the presence of every element or its 
        substantial equivalent in the accused device.''\footnote{Casebook p. 
        324.} If a single element is missing, there is no infringement.
    \end{enumerate}
    \item Claim 1 describes a housing with a ``chamber therein for a 
    liquid~.~.~.'' Amron argued that Larami's design literally infringed Claim 
    1. The court held that Laramie's design did not infringe because it used an 
    external chamber.\footnote{Casebook pp. 324--25.}
\end{enumerate}

\subsubsection{The Doctrine of Equivalents}

\paragraph{Basic Issues: \emph{Graver Tank} and \emph{Warner-Jenkinson}}

\begin{enumerate}
    \item \emph{Graver Tank} (U.S. 1950): to avoid infringing, the copyist has 
    to make important and substantial changes. The rationale is to prevent new 
    inventors from defrauding patent holders by making trivial changes to 
    circumvent the patent's protections.\footnote{Casebook p. 328.}
    \begin{enumerate}
        \item \textbf{``Triple identity'' test}: the doctrine of equivalents 
        applies if a device ``performs substantially the same function in 
        substantially the same way to obtain the same 
        result.''\footnote{Casebook p. 329.}
    \end{enumerate}
    \item In \emph{Warner-Jenkinson} (U.S. 1997), the Court articulated a new 
    test: ``Does the accused product or process contain elements identical or 
    equivalent to each claimed element of the patented invention?'' The new test 
    focuses on individual elements, so it's known as the \textbf{``all elements 
    rule.''}\footnote{Casebook pp. 329--30.}
\end{enumerate}

\paragraph{Prosecution History Estoppel: \emph{Festo Corp. v. Shoketsu}} % TODO 333-45

\begin{enumerate}
    \item 
\end{enumerate}

\paragraph{After-Arising Technologies} % TODO 350-55

\begin{enumerate}
    \item 
\end{enumerate}

\subsubsection{The ``Reverse'' Doctrine of Equivalents} % TODO 355-58

\begin{enumerate}
    \item 
\end{enumerate}

\subsubsection{Equivalents for Means-Plus-Function Claims} % TODO 358-62

\begin{enumerate}
    \item 
\end{enumerate}

\subsection{Defenses}

\subsubsection{Inequitable Conduct: \emph{Therasense, Inc. v. Becton-Dickinson, 
Inc.}} % TODO 375-82

\begin{enumerate}
    \item 
\end{enumerate}

\subsubsection{Exhaustion of Patent Rights} % TODO 382-84

\begin{enumerate}
    \item 
\end{enumerate}

% TODO [bowman supp 12-16]

\subsection{Remedies}

\subsubsection{35 U.S.C. \S\ 285} % TODO 

\begin{enumerate}
    \item 
\end{enumerate}

\subsubsection{35 U.S.C. \S\ 286} % TODO 

\begin{enumerate}
    \item 
\end{enumerate}

\subsubsection{35 U.S.C. \S\ 287} % TODO 

\begin{enumerate}
    \item 
\end{enumerate}

\subsubsection{Injunctions: \emph{eBay, Inc. v. MercExchange, LLC}} % TODO 402-07

\begin{enumerate}
    \item 
\end{enumerate}

\subsubsection{Damages}

\paragraph{Lost Profits} % TODO 407-12

\begin{enumerate}
    \item 
\end{enumerate}

\paragraph{Reasonable Royalty} % TODO 412-15

\begin{enumerate}
    \item 
\end{enumerate}

\paragraph{Willful Infringement: \emph{In re Seagate Technology, LLC}} % TODO 415-421

\begin{enumerate}
    \item 
\end{enumerate}
