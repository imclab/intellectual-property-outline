\section{Patent}

\subsection{Introduction}

% TODO 123-31

\subsection{Elements of Patentability}

\subsubsection{Patentable Subject Matter}

% TODO 35 usc 101

% TODO 132-139 [compositions of matter: chakrabarty]

% TODO 158-177 [abstract ideas: bilski]

% TODO myriad, supp 3-11

\subsubsection{Utility}

% TODO 177-191

\subsubsection{Describing and Enabling the Invention}

% TODO 35 usc 112

% TODO 195-201 [incandescent lamp]

% TODO 205-226

\subsubsection{Novelty and Statutory Bars}

\paragraph{Novelty}

\begin{enumerate}
    \item Old law (1952 Act): defines novelty from the \emph{date of first 
    invention}.
    \item New law (AIA, 2011): novelty is measured as of the \emph{date of the 
    patent application}.\footnote{Casebook p. 26.}
    \item \textbf{Novelty}: new compared to the prior art.
    \item \textbf{Statutory bar}: a bar based on too long a delay in seeking 
    patent protection.
\end{enumerate}

\paragraph{35 U.S.C. \S\ 102 (1952): Conditions for patentability; novelty and 
loss of right to patent}

\begin{enumerate}
    \item (a) Novelty.
    \item (b) Statutory Bars.
    \item (c), (d): (omitted).
    \item (e) Secret Prior Art: Previously filed applications.
    \item (f) Derivation.
    \item (g) (inteference proceedings).
\end{enumerate}

\paragraph{Novelty: \emph{Rosaire v. National Lead Co.}}
~\\\\
Earlier use, even if not publicly known, establishes prior art.

\begin{enumerate}
    \item 1936: Rosaire and Horovitz patented a method for analyzing soil to 
    detect nearby oil.\footnote{Casebook p. 228.}
    \item National Lead argued that the patent was invalid and that there had 
    been no infringement.
    \item Rosaire admitted that Teplitz first invented the method, but argued 
    that Teplitz did not publish his ideas and that they were experimental.
    \item Held: Teplitz use was actual, not experimental, and that publication 
    was not required as long as Teplitz did the work openly.
\end{enumerate}

\paragraph{\emph{Inherency Doctrine}}

% FIXME 232-234

\paragraph{Statutory Bars---Defining ``Publication'': \emph{In Re Hall}}
~\\\\
Publication of a single copy in a foreign university library can satisfy 
the publication limitation.

\begin{itemize}
    \item Hall applied for a chemical patent. The PTO rejected it because the 
    same process had been published in a German doctoral dissertation more than 
    a year earlier.\footnote{Casebook p. 234.}
    \item On appeal, the court held that publication in a university library was 
    ``sufficiently accessible, at least to the public interested in the art, so 
    that such a one by examining the reference could make the claimed invention 
    without further research or experimentation.''\footnote{Casebook p. 235.} 
    Affirmed.
\end{itemize}

\paragraph{Statutory Bars---Public Use: \emph{Egbert v. Lippmann}}
~\\\\
Hidden use can be public use.

\begin{enumerate}
    \item Barnes invented a new type of corset spring in 1855 but did not apply 
    for a patent until 1866. In the meantime, he had given at least two pairs to 
    Frances, who later became his wife and the assignee of the 
    patent.\footnote{Casebook pp. 237--39.}
    \item Frances sued for patent infringement. The question was whether the 
    invention had been in public use for more than two years\footnote{Now the 
    period is one year.} before the patent application.
    \item The Court (Justice Woods) held that it was a public use because (1) a 
    single use is enough, (2) giving the invention to one person is enough, and 
    (3) selling a hidden component of a public machine is still public use.
    \item Justice Miller, dissenting: if a hidden corset spring is a public use, 
    how can anything be a private use?\footnote{Casebook p. 240.}
\end{enumerate}

\paragraph{Experimental Use Exception: \emph{City of Elizabeth v. Pavement 
Company}}
~\\\\
Experimental use is not public use.

\begin{enumerate}
    \item Nicholson was issued a patent for wooden pavement in 1854. He had been 
    testing the design for the previous six years on a public street in 
    Boston.\footnote{Casebook pp. 243--44.}
    \item Nicholson sued the City of Elizabeth for infringement. Elizabeth 
    argued that the patent was invalid because the invention was in public use.
    \item The court held that experimental use is an exception to the public use 
    restriction. Here, Nicholson intended the use to be experimental, and the 
    public's use was only ``incidental.''\footnote{Casebook pp. 245--46.}
\end{enumerate}

\newpage % TODO remove
\paragraph{Priority under the 1952 Act: 35 U.S.C. \S\ 102: Novelty and Loss of 
Right}

\begin{enumerate}
    \item A person gets a patent unless:
    \begin{enumerate}
        \item (g)(1) during the course of an interference, another inventor (from 
        any WTO country) establishes that he got there first, or
        \item (g)(2) another U.S. inventor made the invention first.
    \end{enumerate}
    \item ``~.~.~.~priority generally goes to the first inventor to (1) reduce 
    an invention to practice, without (2) abandoning the 
    invention.''\footnote{Casebook p. 248.} An exception arises if the inventor 
    was not the first to reduce to practice but was reasonably diligent in doing 
    so.
    \item AIA sweeps away this complexity in favor of a first-to-file 
    system---see below. 
\end{enumerate}

\paragraph{Priority and the Reasonable Diligence Exception: \emph{Griffith v. 
Kanamaru}}
~\\\\
Delays to make an invention more marketable do not fall within the reasonable 
diligence exception.

\begin{enumerate}
    \item Griffith and Kanamaru both invented the same compound.
    \item Timeline:
    \begin{enumerate}
        \item June 30, 1981: Griffith established conception.
        \item November 17, 1982: Kanamaru filed for a patent.
        \item June 15, 1983 to September 13. 1983: Griffith was inactive.
        \item January 11, 1984: Griffith reduced to practice.
    \end{enumerate}
    \item Griffith argued that he was reasonably diligent in reducing the 
    invention to practice while (1) he sought funding, as a form of peer review, 
    and (2) awaited the matriculation of a particular grad student, even though 
    there were other students who could have done the work.\footnote{Casebook p. 
    250.}
    \item The court held that the reasonable diligence exception applies to 
    ``everyday problems and limitations'' (e.g., illness), but held that delays 
    in order to ``refine an invention to the most marketable and profitable form 
    have not been accepted as sufficient excuses for 
    inactivity.''\footnote{Casebook p. 250--51.}
    \item Held (affirming the PTO): Griffith failed to establish a prime facie 
    case of reasonable diligence or a sufficient excuse for 
    inactivity.\footnote{Casebook p. 251.}
\end{enumerate}

\paragraph{Prior User Rights}

\begin{enumerate}
    \item Under the 1952 act, a patent holder can exclude another who 
    independently developed the same invention from using it.
    \item The AIA addressed this problem by granting ``prior user rights'' to 
    non-patentees who were using the invention before someone else patented it, 
    with some limitations (e.g., non-transferability).\footnote{Casebook p. 252.}
\end{enumerate}

\paragraph{Novelty, Priority, and Statutory Bars under the AIA}

\begin{enumerate}
    \item The AIA introduced \textbf{three main changes regarding novelty and 
    priority}:\footnote{Casebook p. 253.}
    \begin{enumerate}
        \item The critical date is when a patent is first filed.
        \item Relevant prior art consists of all references available prior to 
        the filing date, with a one-year grace period.
        \item Priority contests will be determined by filing dates.
    \end{enumerate}
    \item \textbf{Structure}:
    \begin{enumerate}
        \item 102(a) Novelty; Prior Art: patent granted unless---
        \begin{enumerate}
            \item (1) Published before filing date.
            \item (2) Described in another previously filed patent.
            \item 
        \end{enumerate}
        \item (b) Exceptions---
        \begin{enumerate}
            \item (1) Disclosure made less than one year before filing is not 
            prior art if---
            \begin{enumerate}
                \item Made by the inventor.
                \item Based on information from the inventor.
            \end{enumerate}
            \item (2) Disclosure is not prior art under (a)(2) if---
            \begin{enumerate}
                \item Subject matter was obtained from the inventor.
                \item Subject matter had been disclosed by the inventor.
                \item Subject matter disclosed and invention were owned by the 
                same person.
            \end{enumerate}
        \end{enumerate}
    \end{enumerate}
    \item No \textbf{geographic restrictions} on prior 
    art.\footnote{Casebook p. 254.}
    \item \textbf{Novelty vs. priority}:
    \begin{enumerate}
        \item Novelty: inventor vs. inventor. Who invented it first?
        \item Priority: inventor vs. prior art. Is it new?
        \item The AIA only asks \textbf{who filed first} (including the one 
        year grace period).\footnote{Casebook pp. 244--45.}
    \end{enumerate}
\end{enumerate}

\paragraph{Grace period under the AIA}

\begin{enumerate}
    \item Generally, under the AIA, an application has to be filed before a 
    prior art event.
    \item But there's a grace period exception. A disclosure (i.e., any prior 
    art reference) is eligible for the grace period if made by the inventor or 
    if the inventor himself previously made a public 
    disclosure.\footnote{Casebook p. 255.}
    \begin{enumerate}
        \item Under the 1952 act, independent third-party disclosure 
        \emph{\textbf{did}} create a grace period. Not so under the 
        AIA.\footnote{Casebook p. 256.}
        \item ``In public use'' or ``on sale'' (AIA \S\ 102) can refer to 
        confidential activity. For instance, an inventor's \textbf{non-informing 
        public use} is a disclosure (but not third-party non-informing 
        uses).\footnote{Casebook p. 257.}
    \end{enumerate}
\end{enumerate}

% FIXME table comparing 1952 act and AIA

\subsubsection{Nonobviousness}

\paragraph{35 U.S.C. \S\ 103}
~\\\\
``A patent for a claimed invention may not be obtained, notwithstanding that the 
claimed invention is not identically disclosed as set forth in section 102, if 
the differences between the claimed invention and the prior art are such that 
the claimed invention as a whole would have been obvious before the effective 
filing date of the claimed invention to a person having ordinary skill in the 
art to which the claimed invention pertains. Patentability shall not be negated 
by the manner in which the invention was made.''

\paragraph{Testing for Obviousness: \emph{Graham v. John Deere Co.}}
~\\\\
\S\ 103 requires an ``inventive leap.''\footnote{Casebook p. 269.} Minor obvious 
improvements don't qualify.

\begin{enumerate}
    \item First case interpreting \S\ 103.
    \item History of the patent system: Jefferson; \enquote{limited monopoly 
    might serve to incite \enquote{ingenuity}~.~.~.~.}\footnote{Casebook p. 
    259--61.}
    \item First appearance of nonobviousness: \emph{Hotchkiss v. Greenwood}, 
    1851. Codified in \S\ 103.\footnote{Casebook p. 261--62.}
    \item \S\ 103 test:\footnote{Casebook pp. 262--63.}
    \begin{enumerate}
        \item Determine scope and content of the prior art.
        \item Determine differences between prior art and the claims.
        \item Determine the level of ordinary skill in the art.
        \item Evaluate obviousness.
    \end{enumerate}
    \item 1950: Graham got a tractor patent ('811).
    \item 1953: Graham modified the design and got a second patent ('798), at 
    issue here. Graham argued that John Deere infringed the '798 patent.
    \item Another patent, held by Glencoe, had all the elements of the '798 
    patent.
    \item The Court found that the improvement in the '798 patent was obvious 
    and therefore invalid. It also found that the Glencoe patent was 
    mechanically identical.\footnote{Casebook p. 268.}
\end{enumerate}

\paragraph{\emph{KSR International Co. v. Teleflex Inc.}}
~\\\\
The Federal Circuit's TSM test is too restrictive. The standard for 
nonobviousness is flexible.

\begin{enumerate}
    \item Engelgau patented a pedal assembly. Teleflex held the exclusive 
    license.\footnote{Casebook p. 269.}
    \item Teleflex sued KSR for infringement. KSR countered that the relevant 
    claim was invalid because its subject matter was obvious.
    \item The Federal Circuit had developed the ``teaching, suggestion, or 
    motivation'' (TSM) test, which held that a patent was nonobvious only if 
    ``some motivation or suggestion to combine the prior art teachings'' could 
    be found in prior art, the nature of the problem, or the knowledge of a 
    person having ordinary skill in the art.\footnote{Casebook p. 270.}
    \item The PTO rejected one of Engelbau's other claims on the basis of 
    nonobviousness. It cited two other patents: an adjustable pedal (Redding) 
    and a method for mounting an electronic sensor on a pedal's support 
    structure (Smith). The PTO rejected the claim that simply put the two 
    together, but it allowed a variation on Redding using a fixed pivot 
    point.\footnote{Casebook p. 272--73.}
    \item The District Court found that the claim was obvious. It granted 
    summary judgment in favor of KSR. The Federal Circuit reversed because the 
    District Court had improperly applied its TSM test.\footnote{Casebook p. 
    273.}
    \item The Supreme Court here found that the TSM test was too formalistic and 
    restrictive. It argued for a broader, more flexible test. It found that 
    Engelbau patent was obvious because any person with ordinary skill in the 
    art would have thought to combine the other patents. 
    Reversed.\footnote{Casebook pp. 274--77.}
\end{enumerate}
\newpage % TODO remove
\paragraph{Nonobviousness and the AIA}

\begin{enumerate}
    \item % FIXME 289
\end{enumerate}

\subsection{Administrative Procedures at the PTO}

% TODO 290-95

\subsection{Infringement}

\subsubsection{Claim Interpretation}

% TODO 290-322

\subsubsection{Literal Infringement}

% TODO 323-328

\subsubsection{The Doctrine of Equivalents}

% TODO 328-345

% TODO 350-355

\subsubsection{The ``Reverse'' Doctrine of Equivalents}

% TODO 355-61

\subsection{Defenses}

\subsubsection{Inequitable Conduct}

% TODO 375-82

\subsubsection{Exhaustion of Patent Rights}

% TODO 382-84

% TODO [bowman supp 12-16]

\subsection{Remedies}

% TODO 35 usc 285, 286, 287

% TODO 399-421
