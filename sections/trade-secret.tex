\section{Trade Secret}

\subsection{Introduction}

\subsubsection{History}

\begin{enumerate}
    \item Acto servi corrupti: corrupting a slave, i.e., bribing to get secret 
    information.\footnote{Casebook p. 33.}
\end{enumerate}

\subsection{Overview}

\begin{enumerate}
    \item Traditionally a common law tort.\footnote{Casebook p. 35.}
    \item 1979: Uniform Trade Secrets Act (UTSA), adopted by 44 states and 
    DC.\footnote{Casebook p. 36.}
    \item Structure of a trade secret claim:\footnote{Casebook p. 37.
    }
    \begin{enumerate}
        \item Subject matter.
        \item Reasonable precautions to keep it secret.
        \item Misappropriation.
    \end{enumerate}
\end{enumerate}

\subsection{Theory}

\begin{enumerate}
    \item Utilitarian: protecting secrets against theft encourages investment 
    in proprietary information.\footnote{Casebook p. 37.}
    \item Tort theory: prevent and punish wrongful acts---``the maintenance of 
    commercial morality.''\footnote{Casebook p.  37--38.}
\end{enumerate}

\subsection{Subject Matter}

\subsubsection{Defining Trade Secrets: \emph{Metallurgical Industries, Inc. v. 
Fourtek, Inc.}}

To determine whether information is a trade secret, courts should consider the 
level of confidentiality, the value to the secret holder, and the cost of 
developing the secret.

\begin{enumerate}
    \item Metallurgical developed improvements to the zinc recovery process. 
    It alleged that these improvements were trade secrets and that former 
    employees stole them and brought them to Fourtek.\footnote{Casebook p. 40.}
    \item The district court held that the modification process was not 
    protectable as a trade secret.
    \item What is a trade secret?
    \begin{enumerate}
        \item Metallurgical's improvements were unknown to the industry. It 
        took steps to protect this information (e.g., NDAs), suggesting it was 
        secret.\footnote{Casebook p. 41.}
        \item Courts should \emph{consider} these factors, but they are not 
        \emph{required}:\footnote{Casebook p. 41--42.}
        \begin{enumerate}
            \item Confidentiality.
            \item Value to the secret holder.
            \item Cost of developing the secret.
        \end{enumerate}
    \end{enumerate}
    \item Here, Metallurgical had a trade secret.
\end{enumerate}
% TODO notes 43-48

\subsubsection{Reasonable Efforts to Maintain Secrecy: \emph{Rockwell Graphic 
Systems, Inc. v. DEV Industries, Inc.}}

% TODO add takeaway

\begin{enumerate}
    \item Rockwell manufactured printing presses. It sometimes subcontracted 
    the manufacture of replacement parts, which involved providing 
    subcontractors with ``piece part drawings'' of the parts to be 
    made.\footnote{Casebook p. 49.}
    \item Some of Rockwell's employees left for DEV, a competitor. They 
    allegedly copied some of the drawings.
    \item DEV argued that the drawings were not trade secrets because Rockwell 
    ``made only perfunctory efforts to keep them secret.''\footnote{Casebook 
    p. 50.} Rockwell did take some measures to preserve secrecy, but DEV 
    argues that it could have done more.
    \item Posner described two conceptions of requirement that the owner take 
    reasonable precautions to preserve secrecy:\footnote{Casebook p. 51.}
    \begin{enumerate}
        \item \emph{Deterrence}: prevent actions that redistribute wealth from 
        one firm to another. The plaintiff must prove that the defendant 
        obtained the secret through a wrongful act.
        \item \emph{Utilitarian}: encourage investment. The owner's actions 
        have evidentiary significance, but mainly to show that the secret has 
        value. It matters less how the defendant acquired the information.
    \end{enumerate}
    \item Under both conceptions, efforts to preserve secrecy also matter for 
    remedies, because if the owner had let the secret fall into the public 
    domain, he would receive a windfall by enforcing exclusive ownership of 
    the secret.\footnote{Casebook p. 52.}
    \item Here, whether Rockwell's efforts were ``reasonable'' is a question 
    of fact. Motion for summary judgment denied.
    \item Perfect security can be inefficient, so it is not 
    required---``perfect security is not optimum security.'' The standard is 
    reasonableness.
\end{enumerate}

% TODO remove
\newpage

\subsubsection{Disclosure of Trade Secrets}

% TODO 58-65

\subsection{Misappropriation of Trade Secrets}

\subsubsection{Improper Means}

% TODO 66-70

\subsubsection{Confidential Relationship}

% TODO 70-76

\subsubsection{Reverse Engineering}

% TODO 76-83

\subsubsection{The Special Case of Departing Employees}

% TODO 83-106

\subsection{Agreements to Keep Secrets}

% TODO 106-111

\subsection{Remedies}

% TODO 111-122
